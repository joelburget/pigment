\section{Features}

This module should import all the feature modules. This module
should be imported by all the functionality modules. This module
thus functions as exactly the list of features included in the
current version of the system.

\section{A universe of descriptions: |Desc|}

The |mkLazyDescDef| function lazily eliminates a desc value (i.e. |d| such that
|desc :>: CON d|. If the tag is canonical, it calls the corresponding case in
the dispatch table with the relevant arguments; otherwise, it cannot compute,
so it returns a term on the |Left|. Note that finite sums are handled using the
case for sigma.

< type DescDispatchTable = (VAL,
<                       VAL -> VAL,
<                       VAL -> VAL -> VAL,
<                       VAL -> VAL -> VAL,
<                       VAL -> VAL -> VAL)

<   mkLazyDescDef :: VAL -> DescDispatchTable -> Either NEU VAL
<   mkLazyDescDef arg (idCase, constCase, prodCase, sigmaCase, piCase) =
<       let args = arg $$ Snd in
<         case arg $$ Fst of
<           IDN     -> Right $ idCase
<           CONSTN  -> Right $ constCase  (args $$ Fst)
<           SUMN    -> Right $ sigmaCase  (ENUMT (args $$ Fst)) (args $$ Snd $$ Fst)
<           PRODN   -> Right $ prodCase   (args $$ Fst) (args $$ Snd $$ Fst)
<           SIGMAN  -> Right $ sigmaCase  (args $$ Fst) (args $$ Snd $$ Fst)

<           PIN     -> Right $ piCase     (args $$ Fst) (args $$ Snd $$ Fst)
<           N t     -> Left t
<           _       -> error "mkLazyDescDef: invalid constructor!"

< type EnumDispatchTable = (VAL, VAL -> VAL -> VAL)

< mkLazyEnumDef :: VAL -> EnumDispatchTable -> Either NEU VAL
< mkLazyEnumDef arg (nilECase, consECase) = let args = arg $$ Snd in
<     case arg $$ Fst of
<         NILN   -> Right nilECase
<         CONSN  -> Right $ consECase (args $$ Fst) (args $$ Snd $$ Fst)
<         N t    -> Left t
<         _      -> error "mkLazyEnumDef: invalid constructor!"

\section{Equality}

In the display syntax, a blue equality can be between arbitrary DExTms,
rather than ascriptions. To allow this, we add a suitable constructor |DEqBlue|
to DInTm, along with appropriate elaboration and distillation rules.

\section{FreeMonad}

\question{What should the coercion rule be for |COMPOSITE|?}

\section{IDesc}

\subsection{Extending the display language}

We introduce a special DIMu for display purposes. While its definition
is the same than |IMu|, its "typing" is not: the label of an |IMu| is
presented as a lambda-bound anchor. When we are displaying a
particular |IMu|, we precisely know at which index we are considering
it. Therefore, a |DIMu| takes an anchor directly. The distillation
rule takes care of taking applying the lambda-bound anchor to the
index of |IMu| to make a fully applied anchor |DIMu|.

\subsection{Plugging Canonical terms in}

<  ("iinduction", [iI,d,i,v,bp,p]) -> App (Var "__iinduction") [d, p, i, v]
<  ("imapBox", [iI,d,x,bp,p,v]) -> App (Var "__imapBox") [d, p, v]

\section{Labelled Types}

\question{The following is all commented out. Is it detritus?}

<   canTy chev (ty :>: Call c tm) = do
<      tytv@(ty :=>: tyv) <- chev (SET :>: ty)
<      ccv@(c :=>: cv) <- chev (ty :>: c)
<      tmtv@(tm :=>: tmv) <- chev (LABEL cv ty :>: tm)
<      return (Call ccv tmtv)

\section{Quotients}

|equivalenceRelation A R| is the proposition that |R| is an equivalence
relation over |A|.
