\documentclass[authoryear]{sigplanconf}

\usepackage[T1]{fontenc}
\usepackage{pslatex}

\usepackage{amsmath}
\usepackage{amssymb}
\usepackage{verbatim}
\usepackage{pig}
\usepackage{url}
\usepackage{xspace}
\usepackage{listings}
\usepackage{flushend}

\include{macros}

%% Haskell-mode for listings
\lstnewenvironment{code}{\lstset{language=Haskell,
                                 basicstyle=\small\ttfamily,
                                 xleftmargin=1cm}}{}

%% Structure
%\newenvironment{structure}{\footnotesize\verbatim}{\endverbatim}
\newenvironment{structure}{\comment}{\endcomment}

%% Written bits of Structure
\newenvironment{wstructure}{\comment}{\endcomment}

%% Comments
%% \setlength{\marginparwidth}{0.7in}
%% \newcommand{\note}[1]{\-\marginpar[\raggedright\footnotesize #1]%
%%                                   {\raggedright\footnotesize #1}}
\newcommand{\note}[1]{}

%% Syntax
\newcommand{\bind}{\emph{bind}\xspace}
\newcommand{\return}{\emph{return}\xspace}

\begin{document}

\ColourEpigram
%\MonochromeEpigram

\conferenceinfo{ICFP'10,} {September 27--29, 2010, Baltimore, Maryland, USA.}
\CopyrightYear{2010}
\copyrightdata{978-1-60558-794-3/10/09}

%\titlebanner{banner above paper title}        % These are ignored unless
%\preprintfooter{short description of paper}   % 'preprint' option specified.



%%%%%%%%%%%%%%%%%%%%%%%%%%%%%%%%%%%%%%%%%%%%%%%%%%%%%%%%%%%%%%%%
%% Title
%%%%%%%%%%%%%%%%%%%%%%%%%%%%%%%%%%%%%%%%%%%%%%%%%%%%%%%%%%%%%%%%


\title{The Gentle Art of Levitation}


%% Alphabetical ordering.
\authorinfo{James Chapman}
           {Institute of Cybernetics, Tallinn University of Technology}
           {james@cs.ioc.ee}
\authorinfo{Pierre-\'{E}variste Dagand \\ Conor McBride}
           {University of Strathclyde}
           {\{dagand,conor\}@cis.strath.ac.uk}
\authorinfo{Peter Morris}
           {University of Nottingham}
           {pwm@cs.nott.ac.uk}


\maketitle


%%%%%%%%%%%%%%%%%%%%%%%%%%%%%%%%%%%%%%%%%%%%%%%%%%%%%%%%%%%%%%%%
%% Abstract
%%%%%%%%%%%%%%%%%%%%%%%%%%%%%%%%%%%%%%%%%%%%%%%%%%%%%%%%%%%%%%%%


\begin{abstract}
  We present a closed dependent type theory whose inductive types
  are given not by a scheme for generative declarations,
  but by encoding in a \emph{universe}. Each inductive datatype arises
  by interpreting its \emph{description}---a
  first-class value in a datatype of descriptions. Moreover, the
  latter itself has a description. Datatype-generic programming thus
  becomes ordinary programming. We show some of the resulting generic
  operations and deploy them in particular, useful ways on the
  datatype of datatype descriptions itself. Simulations in existing
  systems suggest that this
  apparently self-supporting setup is achievable without paradox or
  infinite regress.
\end{abstract}

%\category{CR-number}{subcategory}{third-level}
\category{D.1.1}{Programming Techniques}{Applicative (Functional) Programming}
\category{D.3.3}{Language Constructs and Features}{Data types and structures}

\terms
Design, Languages, Theory



%%%%%%%%%%%%%%%%%%%%%%%%%%%%%%%%%%%%%%%%%%%%%%%%%%%%%%%%%%%%%%%%
%% Introduction
%%%%%%%%%%%%%%%%%%%%%%%%%%%%%%%%%%%%%%%%%%%%%%%%%%%%%%%%%%%%%%%%

\section{Introduction}

Dependent datatypes, such as the ubiquitous vectors (lists indexed by
length) express \emph{relative} notions of data validity. They allow
us to function in a complex world with a higher standard of basic
hygiene than is practical with the context-free datatypes of ML-like
languages. Dependent type systems, as found in
Agda~\cite{norell:agda}, Coq~\cite{coq},
Epigram~\cite{mcbride.mckinna:view-from-the-left}, and contemporary
Haskell~\cite{cheney:gadt, xi:gadt}, are beginning to make themselves useful. As
with rope, the engineering benefits of type indexing sometimes
outweigh the difficulties you can arrange with enough of it.

%Dependent types are an appealing technique for building safer and more
%reliable software. By giving types more expressive power, the
%developer is able to encode more precise invariants in the types. As a
%result, more bugs are caught automatically, during
%type-checking. Because of this benefit, dependently-typed systems have
%flourished, such as Generalized Abstract Data-Types (GADT) in
%Haskell~\cite{spj:gadt}, Agda~\cite{norell:agda},
%Ynot~\cite{morrisett:ynot}, or Epigram~\cite{pigs:epigram}, to name
%but a few.

\begin{wstructure}
<- Describe the problem
    <- Data-types in dependent-type theory
        <- Much more precise
            <- More powerful type-system
            -> Stronger safety guarantees
\end{wstructure}

%In this paper, we will focus on data-types in such systems. Indeed,
%the expressive power of the type-system has a direct impact on
%data-types. Because types can \emph{depend} on terms, our data-types
%can be made more precise. The typical example is vectors, which type
%depend on the size of the vector. Having more precision about
%data-types, we can write safer code: taking the $\CN{head}$ of a
%vector is \emph{ensured} to succeed whenever its type states that it
%is a non-empty vector. This property is automatically enforced by the
%type-checker.

\begin{wstructure}
        <- Equipped with elimination principle
            <- Defining functions over them
            <- Making proofs over them
\end{wstructure}

%Moreover, in total programming systems, such as Agda, Ynot, or
%Epigram, data-types come equipped with an elimination principle: while
%a data-type definition introduces new type formers in the theory, we
%need an eliminator to dispose of them. Looking through the
%Curry-Howard lenses, the elimination principle corresponds to an
%induction principle associated with the data-type. To program over our
%data-types, we rely on their induction principle, guaranteeing the
%well-foundedness of our definition.

\begin{wstructure}
    <- Agda standard library [Nisse file]
        <- x implementations of natural numbers
        <- y implementations of lists
        -> Painful duplication of code and functionality
            <- Types are (slightly) different
                -> Same functions need to be re-implemented 
        -> Crucial need for ``genericity''
\end{wstructure}

The blessing of expressing just the right type for the job can also be
a curse. Where once we might have had a small collection of basic
datatypes and a large library, we now must cope with a cornucopia of
finely confected structures, subtly designed, subtly different. The
basic vector equipment is much like that for lists, but we implement
it separately, often retyping the same code. The Agda standard
library~\cite{nisse:asl}, for example, sports a writhing mass of
list-like structures, including vectors, bounded-length lists,
difference lists, reflexive-transitive closures---the list is
petrifying. Here, we seek equipment to tame this gorgon's head with
\emph{reflection}.

\begin{wstructure}
        /> Coq, Agda: external notion
            <- Not first-class citizen
            -> Cannot "compute" with them
            -> No reflection for data-types (?)
\end{wstructure}

The business of belonging to a datatype is itself a notion
relative to the type's \emph{declaration}. Most typed functional
languages, including those with dependent types, feature a datatype
declaration construct, external to and extending the language for
defining values and programs. However, dependent type systems also
allow us to reflect types as the image of a function from a set of
`codes'---a \emph{universe construction}~\cite{martin-lof:itt}. 
Computing with codes, we expose operations on and
relationships between the types they reflect. Here, we adopt
the universe as our guiding design principle. We abolish the
datatype declaration construct, by reflecting it as a datatype of
datatype descriptions which, moreover, \emph{describes itself}. This
apparently self-supporting construction is a trick, of course, but
we shall show the art of it. We contribute


%As in the simply-typed world, the definition of
%data-types is processed by a meta-theoretical engine, before being
%reifed by extending the type theory with the corresponding type
%formers and elimination principle. Because of this external apparatus,
%data-type definition is not \emph{first-class}: we cannot compute with
%them, such as making new data-types from previous data-types. 
%
%This is a rather harsh limitation, in particular in a
%dependently-typed system. Indeed, reflection~\cite{allen:reflection,
%  gregoire:ring-solver} is at the heart of many dependently-typed
%programming techniques. Not having first-class data-type definitions,
%we have to give up reflection for data-types.

\begin{wstructure}
    <- Dependent types offer new programming techniques
        <- Eg.: universe construction
        /> State of the art haunted by the simply-typed paradigm
            -> Generative
            -> Non reflective
\end{wstructure}

%However, we do not think that we are condemned to such fate. The
%external presentation of data-types is an heritage of the simply-typed
%paradigm. Dependently-typed systems have more to offer. Indeed, new
%programming techniques, unavailable in a simply-typed setting,
%arises. One of them is \emph{universe
%  construction}~\cite{martin-lof:itt}. We shall see how this technique
%help us overcoming the limitations of the standard, non reflective and
%generative presentation of data-types.

\begin{wstructure}
<- State contributions
    <- Closed presentation of data-types 
        -> No generativity requires
        -> Subsuming standard inductive families 
            /> Some popular extensions excluded for now
    <- Descriptions of data-types are first-class 
        <- Self-encoded [Section sec:desc-levitate]
    <- ``generic programming is just programming''
        <- Ability to inspect data-type definition
            -> Write program over them
        <- A generic program works over a class of data-types (???)
            -> Capture this class by common structure
            -> Write a program over this common code
    <- Design a language for generic programming
        -> First serious attempt
            /> except possibly Lisp
                <- ???
\end{wstructure}

%In this paper, we propose a new approach to building data-types in a
%dependent-type theory. Our contributions are the following:

\begin{itemize}
\item a \emph{closed} type theory, extensible only \emph{definitionally},
  nonetheless equipped with a universe of inductive families of datatypes;
\item a \emph{self-encoding} of the universe codes as a datatype in the
  universe---datatype generic programming is just programming;
\item a bidirectional \emph{type propagation} mechanism to conceal
  artefacts of the encoding, restoring
  a convenient presentation of data;
\item examples of generic operations and constructions over our universe,
  notably the \emph{free monad} construction;
\item datatype generic programming delivered \emph{directly},
  not via some isomorphic
  model or `view' of declared types.
\end{itemize}

We study two universes as a means to explore this novel way to equip a
programming language with its datatypes. We warm up with a universe of
\emph{simple} datatypes, just sufficient to describe itself. Once we
have learned this art, we scale up to \emph{indexed} datatypes,
encompassing the inductive families~\cite{dybjer:families,luo:utt}
found in Coq and Epigram, and delivering experiments in generic
programming with applications to the datatype of codes itself.

We aim to deliver proof of concept, showing that a closed theory with
a self-encoding universe of datatypes can be made practicable, but we
are sure there are bigger and better universes waiting for a similar
treatment. Benke, Dybjer and
Jansson~\cite{benke:universe-generic-prog} provide a useful survey of
the possibilities, including extension to inductive-recursive
definition, whose closed-form presentation~\cite{dybjer:axiom-ir,
  dybjer:ir-initial-algebra} is both an inspiration for the present
enterprise, and a direction for future study.

The work of Morris, Altenkirch and
Ghani~\cite{morris:PhD,morris:spf,alti:lics09} on
(indexed) containers has informed our style of encoding and the
equipment we choose to develop, but the details here reflect pragmatic
concerns about intensional properties which demand care in
practice. We have thus been able to implement our work as the basis
for datatypes in the Epigram 2 prototype~\cite{pigs:epigram}. We
have also developed a \emph{stratified} model of our coding scheme
in Agda and Coq\footnote{This model is available at \\
\url{http://personal.cis.strath.ac.uk/~dagand/levitate.tar.gz}}.




%\item We present a basic type-theory and extend it with a universe of
%  finite sets (Section~\ref{sec:type-theory}). We show how coding can
%  be made practical by putting types at work ;
%\item We give a closed presentation of inductive data-types, through a
%  universe of descriptions (Section~\ref{sec:universe-desc}). This
%  first universe has the expressive power of simple inductive
%  types. Being closed, this presentation does not require
%  generativity, hence the type theory remains unchanged when
%  data-types are introduced ;
%\item We present a self-encoding of the universe of description inside
%  itself (Section~\ref{sec:desc-levitate}). As a consequence,
%  description of data-types appears as first-class object in the type
%  theory. We illustrate the benefit of a first-order presentation by
%  implementing a generic catamorphism as well as a generic free monad
%  construction, together with its monadic operations ;
%\item We index the universe of descriptions, to subsume standard
%  inductive families (Section~\ref{sec:indexing-desc}). In this
%  setting, we develop several examples of dependently-typed
%  data-structure and some generic operations over them ;
%\item We have implemented this technology in the Epigram programming
%  language. This is, we believe, the first attempt to design a
%  language for generic programming, Lisp having opened the way. We
%  propose and demonstrate with several examples that generic
%  programming is just programming. Because data-types are described by
%  code, we can finally program with them. As a consequence, generic
%  programs are implemented as functions built from the data-type
%  definition.
%\end{itemize}

%%%%%%%%%%%%%%%%%%%%%%%%%%%%%%%%%%%%%%%%%%%%%%%%%%%%%%%%%%%%%%%%
%% The Type Theory
%%%%%%%%%%%%%%%%%%%%%%%%%%%%%%%%%%%%%%%%%%%%%%%%%%%%%%%%%%%%%%%%

\section{The Type Theory}

\subsection{Base theory}

\begin{structure}
<- Presentation of the formalism
    <- Standard presentation
        -> No novelty here
    <- 3 judgments [equation]
        -> Context validity
        -> Typing judgements
        -> Equality judgements
\end{structure}

\[
\begin{array}{ll}
\Gamma\vdash\Valid & \mbox{\(\Gamma\) is a valid context, giving types to variables} \\
\Gamma\vdash \Bhab{t}{T} & \mbox{term \(t\) has type \(T\) in context \(\Gamma\)} \\
\Gamma\vdash \Bhab{s \equiv t}{T} & \mbox{\(s\) and \(t\) are equal at type \(T\) in context \(\Gamma\)} \\
\end{array}
\]

\begin{structure}
    <- Invariants [equation]
        -> By induction on derivations
\end{structure}

The systems of inference rules will be formulated to ensure that that the
following implications always hold by induction on derivations.
\[
\begin{array}{l@{\;\;\Rightarrow\;\;}l}
\Gamma\vdash \Bhab{t}{T}            & \Gamma\vdash\Valid \;\wedge\; \Gamma\vdash\Type{T} \\
\Gamma\vdash s \equiv \Bhab{t}{T}   & \Gamma\vdash \Bhab{s}{T} \;\wedge\; \Gamma\vdash \Bhab{t}{T} \\
\Gamma;\xS;\Delta \vdash J          & \Gamma\vdash \Bhab{s}{S} \;\Rightarrow\; 
                                           \Gamma;\Delta[s/x] \vdash J[s/x] \\
\end{array}
\]


\begin{structure}
    <- Judgemental equality
        <- Presentation independant of particular implementation choice
        -> Model in Agda, intensional
        -> Used in Epigram, OTT
\end{structure}

\begin{structure}
<- Context validity [figure]
    <- Not much to be said
\end{structure}

\begin{figure}

\[
%% Empty context validity
\Axiom{\vdash \Valid}
\qquad
%% Extend context
\Rule{\Gamma       \vdash \Type{S}}
     {\Gamma ; \xS \vdash \Valid}\;x\not\in\Gamma
\]

\caption{Context validity}
\label{fig:context-validity}
\end{figure}


\begin{structure}
<- Typing judgements [figure]
    <- Set in Set
        -> For simplicity of presentation
        -> Assume that a valid stratification can be inferred
            <- Harper-Pollack, Luo, Courant
        -> See later discussion
    <- Standard presentation of Pi and Sigma types
\end{structure}

\begin{figure}

\[\stkc{
%% Context
\Rule{\Gamma ; \xS ; \Delta \vdash \Valid}
     {\Gamma ; \xS ; \Delta \vdash \Bhab{\V{x}}{\V{S}}}
\qquad
%% Conversion
\Rule{\Gamma \vdash \Bhab{\V{s}}{\V{S}} \quad 
      \Gamma \vdash \Type{\V{S} \equiv \V{T}}}
     {\Gamma \vdash \Bhab{\V{s}}{\V{T}}}
\\
%% Girard's favorite
\Rule{\Gamma \vdash \Valid}
     {\Gamma \vdash \Type{\Set}}
\qquad
%% Pi-Sigma
\Rule{\Gamma       \vdash \Type{\V{S}} \quad
      \Gamma ; \xS \vdash \Type{\V{T}}}
     {\Gamma \vdash \Type{\PIS{\xS} \V{T}, \SIGMAS{\xS} \V{T}}}
\\
%% %% Prop
%% \Rule{\Gamma \vdash \Bhab{q}{\Prop}}
%%      {\Gamma \vdash \Type{\prf{q}}}
%% \qquad
%% %% True
%% \Rule{\Gamma \vdash \Valid}
%%      {\Gamma \vdash \Bhab{\True}{\Prop}}
%% \\
%% Unit
\Rule{\Gamma \vdash \Valid}
     {\Gamma \vdash \Bhab{\Unit}{\Set}}
\qquad
%% Void
\Rule{\Gamma \vdash \Valid}
     {\Gamma \vdash \Bhab{\Void}{\Unit}}
\\
%% Lambda
\Rule{\stkl{\Gamma       \vdash \Type{\V{S}} \\
            \Gamma ; \xS \vdash \Bhab{\V{t}}{\V{T}}}}
     {\Gamma \vdash \Bhab{\PLAM{\x}{\V{S}} \V{t}}{\PIS{\xS} \V{T}}}
\qquad
%% Application
\Rule{\stkl{\Gamma \vdash \Bhab{\V{f}}{\PIS{\xS} \V{T}} \\
            \Gamma \vdash \Bhab{\V{s}}{\V{S}}}}
     {\Gamma \vdash \Bhab{\V{f}\: \V{s}}{\V{T}[\V{s}/\V{x}]}} 
\\
%% Pair
\Rule{\Gamma       \vdash \Bhab{\V{s}}{S} \quad 
      \Gamma ; \xS \vdash \Bhab{\V{T}}{\Set}    \quad
      \Gamma       \vdash \Bhab{\V{t}}{\V{T}[\V{s}/\V{x}]}}
     {\Gamma \vdash \Bhab{\pair{\V{s}}{\V{t}}{\V{x}.\V{T}}}{\SIGMAS{\xS} \V{T}}}
\\
%% First projection
\Rule{\Gamma \vdash \Bhab{\V{p}}{\SIGMAS{\xS} \V{T}}}
     {\Gamma \vdash \Bhab{\fst{\V{p}}}{\V{S}}} 
\qquad
%% Second projection
\Rule{\Gamma \vdash \Bhab{\V{p}}{\SIGMAS{\xS} \V{T}}}
     {\Gamma \vdash \Bhab{\snd{\V{p}}}{\V{T}[\fst{\V{p}}/\V{x}]}}
\\
}\]


\caption{Typing judgements}
\label{fig:typing-judgements}

\end{figure}


\begin{structure}
<- Judgemental equality [figure]
    <- symmetry, reflexivity, and transitivity
    <- beta-rules for lambda and pair
    <- xi-rule for functions
    -> Agnostic in the notion of equality
        <- Doesn't rely on a ``propositional'' equality
\end{structure}


\begin{figure}

\[\stkc{
%% %% Reflexivity
%% \Rule{\Gamma \vdash \Bhab{x}{T}}
%%      {\Gamma \vdash \Bhab{x \equiv x}{T}}
%% \qquad
%% %% Symmetry
%% \Rule{\Gamma \vdash \Bhab{x \equiv y}{T}}
%%      {\Gamma \vdash \Bhab{y \equiv x}{T}}
%% \qquad
%% %% Transitivity
%% \Rule{\stkl{\Gamma \vdash \Bhab{z \equiv y}{T} \\
%%             \Gamma \vdash \Bhab{y \equiv z}{T} }}
%%      {\Gamma \vdash \Bhab{x \equiv z}{T}}
%% \\
%% Beta-reduction
\Rule{\stkl{\Gamma       \vdash \Type{\M{S}} \quad
            \Gamma ; \xS \vdash \Bhab{\M{t}}{\M{T}} \\
            \Gamma       \vdash \Bhab{\M{s}}{\M{S}}}}
     {\Gamma \vdash \Bhab{(\PLAM{\x}{\M{S}} \M{t})\:\M{s} \equiv \M{t}[\M{s}/\x]}{\M{T}[\M{s}/\x]}}
\\
%% Xi rule
%% \Rule{\Gamma       \vdash \Type{S} \quad
%%       \Gamma ; \xS \vdash \Bhab{t \equiv t'}{T}}
%%      {\Gamma \vdash \Bhab{(\PLAM{\x}{S} t) \equiv (\PLAM{\x}{S} t')}{\PIS{\xS} T}}
%% \\
%% Projections
\Rule{\stkl{\Gamma                 \vdash \Bhab{\M{s}}{\M{S}} \quad
            \Gamma ; \xS           \vdash \Bhab{\M{T}}{\Set} \\
            \Gamma ; \Bhab{\M{s}}{\M{S}}   \vdash \Bhab{\M{t}}{\M{T}[\M{s}/\x]}}}
     {\Gamma \vdash \Bhab{\fst{(\pair{\M{s}}{\M{t}}{\x.\M{T}})} \equiv \M{s}}{\M{S}}}
\qquad
\Rule{\stkl{\Gamma               \vdash \Bhab{\M{s}}{\M{S}} \quad
            \Gamma ; \xS         \vdash \Bhab{\M{T}}{\Set} \\
            \Gamma ; \Bhab{\M{s}}{\M{S}} \vdash \Bhab{\M{t}}{\M{T}[\M{s}/\x]}}}
     {\Gamma \vdash \Bhab{\snd{(\pair{\M{s}}{\M{t}}{\x.\M{T}})} \equiv \M{t}}{\M{T}[\M{s}/\x]}}
}\]


\caption{Judgemental equality}
\label{fig:judgemental-equality}

\end{figure}



\begin{structure}
!!! Need Help !!!
<- Meta-theoretical properties
    <- Assuming a stratified discipline
    <> The point here is to reassert that dependent types are not evil, 
       there is no non-terminating type-checker, or such horrible lies <>
    -> Strongly normalizing
        -> Every program terminates
    -> Type-checking terminates
    ???
\end{structure}

\subsection{Finite sets}

\begin{structure}
<- Motivation
    <- Finite sets could be encoded with Unit and Bool
        /> Hinder the ability to name things
    -> For convenience
        <- Named elements
        <- Referring by name instead of code
        -> Types as coding presentation
            /> Also as coding representation!
\end{structure}

\begin{structure}
<- Implementation [figure]      
    <- Tags
        -> Purely informational token
    <- EnumU
        -> Universe of finite sets
    <- EnumT e
        -> Elements of finite set e
\end{structure}

\begin{figure}

\[\stkc{
%% UId
\Rule{\Gamma \vdash \Valid}
     {\Gamma \vdash \Type{\UId}}
\qquad
%% Tag
\Rule{\Gamma \vdash \Valid}
     {\Gamma \vdash \Bhab{\Tag{s}}{\UId}}\;s \mbox{ unique identifier}
\\
%% EnumU
\Rule{\Gamma \vdash \Valid}
     {\Gamma \vdash \Type{\EnumU}} 
\qquad
%% EnumT
\Rule{\Gamma \vdash \Bhab{e}{\EnumU}}
     {\Gamma \vdash \Type{\EnumT{e}}} 
\\
%% NilE
\Rule{\Gamma \vdash \Valid}
     {\Gamma \vdash \Bhab{\NilE}{\EnumU}} 
\qquad
%% ConsE
\Rule{\Gamma \vdash \Bhab{t}{\UId} \quad
      \Gamma \vdash \Bhab{e}{\EnumU}}
     {\Gamma \vdash \Bhab{\ConsE{t}{e}}{\EnumU}}
\\
%% Ze
\Rule{\Gamma \vdash \Valid}
     {\Gamma \vdash \Bhab{\Ze}{\EnumT{\ConsE{t}{e}}}} 
\qquad
%% Su
\Rule{\Gamma \vdash \Bhab{n}{\EnumT{e}}}
     {\Gamma \vdash \Bhab{\Su{n}}{\EnumT{\ConsE{t}{e}}}}
}\]


\caption{Typing rules for finite sets}
\label{fig:typing-finite-set}

\end{figure}


\begin{structure}
<- Equipment
    <- \spi operator
        <- Equivalent of Pi on finite sets
        <- First argument: (finite) domain
        <- Second argument: for each element of the domain, a co-domain
        -> Inhabitant of \spi: right-nested tuple of solutions
            <- Skip code for space reasons
    <- switch operator
        <- case analyses over x
        <- index into the \spi tuple to retrieve the corresponding result
\end{structure}

\begin{figure}

\[\stkl{
%% spi
\F{$\pi$} : \PITEL{\V{e}}{\EnumU}
            \PITEL{\V{P}}{\EnumT{e} \To \Set} \To \Set 
\\
%% switch
\begin{array}{@{}ll}
\F{switch} : & \PITEL{\V{e}}{\EnumU}
               \PITEL{\V{P}}{\EnumT{e} \To \Set} \\
             & \PITEL{\V{b}}{\spi{e}{P}}
               \PITEL{\V{x}}{\EnumT{e}} \To P~x
\end{array}
}\]

\caption{Operators on finite sets}
\label{fig:operator-finite-set}

\end{figure}


\begin{structure}
<- Equivalent to having a function space over finite sets
    /> Made non-obvious by low-level encodings
        <- General issue with codes
             -> Need to provide an attractive presentation to the user
    -> Types seem to obfuscate our reading
        <- Provide ``too much'' information
        /> False impression: information is actually waiting to be used more widely
        -> See next Section
\end{structure}

\subsection{Type propagation}

\begin{structure}
<- Bidirectional type-checking [ref. Turner,Pierce]
    -> Separating type-checking from type synthesis
    <- Type checking: push terms into types
        <- Example: |Pi S T :>: \ x . t| allows us to drop annotation on lambda
    <- Type inference: pull types out of terms
        <- Example: |x : S l- x :<: S| gives us the type of x
\end{structure}

\begin{structure}
<- Formalization: type propagation
    <- Motivation
        -> High-level syntax
            -> exprIn: types are pushed in
                <- Subject to type *checking*
            -> exprEx: types are pulled from
                <- Subject to type *synthesis*
        -> Translated into our low-level type theory
        -> Presented as judgements
    -> Presentation mirors typing rule of [figure] 
        -> Ignore identitical judgements
\end{structure}

\begin{structure}
<- Type checking [figure]
    <- Push a type in an exprIn
    <- Result in a full term
    -> *Use* the type to build the term!
        -> Domain and co-domain propagation for Pi and Sigma
        -> Translation of 'tags into EnumTs
        -> Translation of ['tags ...] into EnumUs
        -> Finite function space into switch
\end{structure}

\begin{figure}

\[\stkc{
%% Form
\boxed{\Gamma \Vdash \propag{\push{\CN{exprIn}}{\CN{type}}}{\CN{term}}} 
\\
\\
%% Set in Set
%% \Axiom{\Gamma \Vdash \propag{\push{\Set}{\Set}}
%%                             {\Set}}
%% \\
%% Prop
%% \Rule{\Gamma \Vdash \propag{\push{q}{\Prop}}
%%                            {q'}}
%%      {\Gamma \Vdash \propag{\push{\prf{q}}{\Set}}
%%                            {\prf{q'}}}
%% \qquad
%% True
%% \Axiom{\Gamma \Vdash \propag{\push{\True}{\Prop}}
%%                             {\True}}
%% \\
%% Pi
%% \Rule{\Gamma \Vdash \propag{\push{S}{\Set}}
%%                            {S'} \quad
%%       \Gamma \Vdash \propag{\push{T}{S' \To \Set}}
%%                            {T'}}
%%      {\Gamma \Vdash \propag{\push{\PiTy{S}{T}}{\Set}}
%%                            {\PiTy{S'}{T'}}} 
%% \\
%% %% Sigma
%% \Rule{\Gamma \Vdash \propag{\push{S}{\Set}}
%%                            {S'} \quad
%%       \Gamma \Vdash \propag{\push{T}{S' \To \Set}}
%%                            {T'}}
%%      {\Gamma \Vdash \propag{\push{\SigmaTy{S}{T}}{\Set}}
%%                            {\SigmaTy{S'}{T'}}}
%% \\
%% Lambda
\Rule{\Gamma ; \xS \Vdash \propag{\push{\V{t}}{\V{T}}}
                                 {\V{t'}}}
     {\Gamma \Vdash \propag{\push{\LAM{\x} \V{t}}{\PIS{\xS}{\V{T}}}}
                           {\PLAM{\x}{\V{S}} \V{t'}}} 
\\
%% Pair
\Rule{\stkl{ \Gamma \Vdash \propag{\push{\V{s}}{\V{S}}}
                                  {\V{s'}} \\
             \Gamma \Vdash \propag{\push{\V{t}}{\V{T}[\V{s'}/\V{x}]}}
                                  {\V{t'}}}}
     {\Gamma \Vdash \propag{\push{\pair{\V{s}}{\V{t}}{}}{\SIGMAS{\xS}{\V{T}}}}
                           {\pair{\V{s'}}{\V{t'}}{\V{T}}}}
\\
%% EnumU
%% \Axiom{\Gamma \Vdash \propag{\push{\EnumU}{\Set}}
%%                             {\EnumU}} 
%% \qquad
%% %% EnumT
%% \Rule{\Gamma \Vdash \propag{\push{e}{\EnumU}}
%%                            {e'}}
%%      {\Gamma \Vdash \propag{\push{\EnumT{e}}{\Set}}
%%                            {\EnumT{e'}}}
%% \\
%% Tag
\Axiom{\Gamma \Vdash \propag{\push{\etag{t_0}}{\EnumT{(\ConsE{\etag{t_1}}{\V{e}})}}}
                            {\Ze}}\;\etag{t_0} = \etag{t_1}
\\
\Rule{\Gamma \Vdash \propag{\push{\etag{t_0}}{\EnumT{\V{e}}}}
                           {\V{n}}}
     {\Gamma \Vdash \propag{\push{\etag{t_0}}{\EnumT{(\ConsE{\etag{t_1}}{\V{e}})}}}
                            {\Su{\V{n}}}}\;\etag{t_0} \neq \etag{t_1}
\\
%% EnumU
\Axiom{\Gamma \Vdash \propag{\push{[]}{\EnumU}}
                            {\NilE}}
\\
\Rule{\Gamma \Vdash \propag{\push{\V{ts}}{\EnumU}}
                                 {\V{cs}}}
     {\Gamma \Vdash \propag{\push{[ \etag{t_1},\: \V{ts} ]}{\EnumU}}
                                 {\ConsE{\etag{t_1}}{\V{cs}}}}
\\
%% Switch
\Rule{\Gamma \Vdash \propag{\push{\V{t}}{\spi{\V{e}}{\V{P}}}}
                           {\V{t'}}}
     {\Gamma \Vdash \begin{array}{@{}l} 
                        \propag{\push{\V{t}}{\PI{\V{x}}{\EnumT{\V{e}}} \V{P}\:\V{x}}}
                               {\\ \PLAM{\V{x}}{(\EnumT{\V{e}})} \switch{\V{e}}{\V{P}}{\V{t'}}{\V{x}}}
                    \end{array}}\;\mbox{$\V{t}$ is $[]$ or $[\_,\_]$}
%% \\
%% Conversion
%% \Rule{\Gamma \Vdash \propag{s}
%%                            {\pull{s'}{S}} \quad 
%%       \Gamma \Vdash \push{S \equiv T}{\Set}}
%%      {\Gamma \Vdash \propag{\push{s}{T}}
%%                            {s'}}
}\]


\caption{Type checking}
\label{fig:type-checking}

\end{figure}


\begin{structure}
<- Type synthesis [figure]
    <- Pull a type out of an exprEx
    <- Result in a full term, together with its type
    -> Do *not* need to specify types
        -> Extracting a term from the context
        -> Function application
        -> Projections
\end{structure}

\begin{figure}

\[\stkc{
%% Form
\boxed{\Gamma \Vdash \propag{\exprEx}{\pull{\CN{term}}{\CN{type}}}}
\\
\\
%% Reversal
\Rule{\Gamma \Vdash \propag{\push{T}{\Set}}
                           {T'} \quad
      \Gamma \Vdash \propag{\push{t}{T'}}
                           {t'}}
     {\Gamma \Vdash \propag{(\Bhab{t}{T})}
                           {\pull{t'}{T'}}} 
\\
%% Context
\Rule{\Gamma ; \xS ; \Delta \vdash \Valid}
      {\Gamma ; \xS ; \Delta \Vdash \propag{\x}
                                           {\pull{\x}{\M{S}}}}
\qquad
%% Application
\Rule{\stkl{\Gamma \Vdash \propag{\M{f}}
                                 {\pull{\M{f}\M{'}}{\PIS{\xS}{\M{T}}}} \\
            \Gamma \Vdash \propag{\push{\M{s}}{\M{S}}}
                                 {\M{s'}}}}
     {\Gamma \Vdash \propag{\M{f}\: \M{s}}{\pull{\M{f'}\: \M{s'}}{\M{T} [\M{s'}/\x]}}} 
\\
%% First projection
\Rule{\Gamma \Vdash \propag{\M{p}}
                           {\pull{\M{p'}}{\SIGMAS{\xS}{\M{T}}}}}
     {\Gamma \Vdash \propag{\fst{\M{p}}}
                           {\pull{\fst{\M{p'}}}{\M{S}}}} \qquad 
%% Second projection
\Rule{\Gamma \Vdash \propag{\M{p}}
                           {\pull{\M{p'}}{\SIGMAS{\xS}{\M{T}}}}}
     {\Gamma \Vdash \propag{\snd{\M{p}}}
                           {\pull{\snd{\M{p'}}}{\M{T} [\fst{\M{p'}}/\x]}}}
}\]


\caption{Type synthesis}
\label{fig:type-synthesis}

\end{figure}


\begin{structure}
<- Summary
    -> Not a novel technique [refs?]
        /> Used as a boilerplate scrapper
    -> Make dealing with codes *practical*
        <- Example: Finite sets/finite function space
        -> We should not restrain ourself in using codes
            <- We know how to present them to the user
-> Will extend this machinery in further sections
\end{structure}


%%%%%%%%%%%%%%%%%%%%%%%%%%%%%%%%%%%%%%%%%%%%%%%%%%%%%%%%%%%%%%%%
%% A Universe of simple data-types
%%%%%%%%%%%%%%%%%%%%%%%%%%%%%%%%%%%%%%%%%%%%%%%%%%%%%%%%%%%%%%%%

\section{A universe of simple data-types}

\begin{wstructure}
<- Why starting with simple data-types
    <- For pedagogical purposes
        <- Data-types as we know them every day
        /> Target dependent types
    -> Cut down version of Induction Recursion
        -> Presentation evolves independently as we go to dependent types
\end{wstructure}

In this section, we describe an alternative implementation of
inductive types, as we know them in simply-typed languages such as
Haskell or OCaml. This choice is motivated by pedagogical
considerations: by describing the layman data-types in our setting, we
build on our intuition of data-types we are familiar with. However,
our goal is and remains to discuss data-types as we find them in
dependently-typed languages.

Our proposal is based on induction-recursion~\cite{dybjer:general-ir,
  dybjer:axiom-ir, dybjer:ir-initial-algebra, dybjer:iir}. However, in
the context of simple data-types, the full power of
induction-recursion is not necessary. Therefore, we will present a
stripped-down version. As we extend the scope of our universe of
data-types, our presentation will diverge from induction-recursion. We
shall compare both approaches in Section~\ref{sec:discussion}

\subsection{The power of $\Sigma$}

\begin{wstructure}
<- The duality of Sigma
    <- Sigma generalises sum over arbitrary arities
        -> \Sigma A B == \Sigma_{x : A} B x
    <- Sigma generalises product to have a dependant second component
        -> \Sigma A B == (x : A) \times (B x)
\end{wstructure}

In dependently-typed language, $\Sigma$-types can be read in two
rather different ways. This duality is actually reflected in the
notation we can find in the literature, depending on the sensibility
of the author. Hence, the type $\SigmaTy{A}{B}$ is sometimes written
$\Sigma_{x : A} (B x)$. This notation stresses the fact that
$\Sigma$-types are a generalisation of sums over arbitrary
arities. When simply-typed languages have finite sums,
dependently-typed languages have sums of indexed by any set.

On the other hand, $\SigmaTy{A}{B}$ is sometimes written
$\SIGMA{\V{x}}{A} (B x)$. Under this perspective, $\Sigma$-types can
be read as a generalisation of products, where the second component
can depend on the first one. When simply-typed languages pack data
into non-dependent tuples, dependently-types languages tolerate this
non-dependent usage. However, they also give the rather novel ability
for data to influence further data.

\begin{wstructure}
<- Data-types in the simply-typed world
    -> "sums-of-product"
        <- Sum of constructors
        <- Product of arguments
<- Data-types in the dependently-typed world
    -> "sigmas-of-sigmas"
    /> Need ability to manipulate these sigmas
        -> Define a Code for data-types
        -> Together with a sigma-based Interpretation
\end{wstructure}

In the simply-typed world, the essence of data-type definition can be
summed up by the term ``sum-of-product''. A data-type is defined by a
finite sum, choice, of constructors, each of them composed by a
product, a tuple, of arguments. Therefore, the grammar of data-types
definition in simply-typed languages, such as Haskell or OCaml, follow
this form.

To model these data-types, we are simply left with capturing this
grammar in a dependently-typed setting. Hence, the notion of
``sum-of-product'' naturally translates into
\emph{sigmas-of-sigmas}. However, beyond this intuition, it should be
clear that a direct encoding of data-types through raw $\Sigma$-types
is not a viable option. Indeed, just as for finite sets, encoding
throw away information when we crucially need it. In the realm of
data-types, things become even more tougher, as one could legitimately
expect an induction principle, for instance. Anonymous in a see of
$\Sigma$s, our data-types have little chance of survival.


\subsection{A universe of descriptions}

\begin{wstructure}
<- Introduction to Universe construction
    <- Define a Code
        -> Name objects
    <- Define an Interpretation of codes into the type theory
        -> Give a semantics to objects
    -> Ability to manipulate code
    -> Ability to compute with these objects
\end{wstructure}

\note{Check Martin-Lof reference}

For our sigma-of-sigmas to be known, we crucially need the ability to
name them. To this end, we will use a standard technique in
dependently-typed programming: we construct an ad-hoc universe. This
technique dates back to Martin-L\"of definition of type
theory~\cite{martin-lof:itt}. Since then, it has been fruitful as a
programming technique~\cite{who?} \note{I would need some other
  example of universe construction for practical purposes}. We refer
the reader to Agda's tutorial~\cite{norell:agda-tutorial} for a
pedagogical presentation of universe construction.

The key idea behind universe construction is our ability to make names
by defining new types. These names are called \emph{codes}. By
defining a set of codes, we somehow define the syntax of a
language. However, as such, a system of code is useless as it lacks a
semantics. Instead of equipping the universe of codes with some
computational behaviour, we pragmatically chose to \emph{interpret}
these codes back into the standard type theory. Hence, codes act only
as labels, while the type theory provides the computational machinery.

Codes being simple labels, we have the ability to inspect them, hence
taking advantage of their structural information. Being able to
inspect them, we are therefore free to compute over them: deriving new
codes from previous ones, or even new functions on them. In several
occasions, we will have the opportunity to witness the power of
universes.

\note{Is that really a motivating motivation for the usage of universe
  construction? }

\begin{wstructure}
<- Justification of the code 
    <- [both figures]: cannot be read separately
    <- Mimic the standard grammar of data-types description
        -> Just as we already know it
        <- '\Sigma for making sigmas-of-sigmas
        <- 'indx for exhibiting the functoriality
            -> For recursive arguments
        <- '1 for end of description
\end{wstructure}

Hence, we propose to embed inductive types as a universe in our
dependent type theory, the universe of \emph{descriptions}
(Figure~\ref{fig:desc_universe}). As expected, the code of this
universe mimics the standard grammar of data-types definitions in
simply-typed languages. Hence, we have a $\DSigma{}{}$ code,
interpreted as a $\Sigma$-type, to build the
sigmas-of-sigmas. Descriptions are terminated by $\DUnit$, which
contains no useful payload. The functoriality of the data-types is
introduced by $\DIndx$. When we tie the knot with a fix-point, the
hole left open by $\DIndx$ will be turned into a standard recursive
argument. We notice that this functoriality appears in the type of
$\descop{D}{}$ itself, for a given data-type definition $D$. This
corresponds to the morphism part of the functor described by $D$.

\begin{figure*}

\[
\begin{array}{ll}
\stk{
\data \Desc : \Set \where \\
\;\;\begin{array}{@{}l@{\::\:}l@{\quad}l}
    \DUnit          & \Desc \\
    \DSigma         & \PI{\V{S}}{\Set} \PIS{S \To \Desc} \Desc \\
    \DIndx          & \Desc \To \Desc
\end{array}
}
&
\stk{
\descop{\_\:}{} : \Desc \To \Set \To \Set \\
\begin{array}{@{}ll@{\:=\:\:}ll}
\descop{\DUnit}{& X}        &  \Unit                                       \\
\descop{\DSigma{S}{D}}{& X} &  \SIGMAS{\V{s} : S}{\descop{D~s}{X}}         \\
\descop{\DIndx{D}}{& X}     &  \TIMES{X}{\descop{D}{X}}
\end{array}
}
\end{array}
\]


\caption{Universe of Descriptions}
\label{fig:desc_universe}

\end{figure*}

To give some intuition on this universe of descriptions, we now turn
to some examples. For obvious pedagogical reasons, we will manually
build these descriptions. However, it should be clear that, in
practice, these definitions can be automatically constructed an
Haskell-like $\data$ definition.

\subsection{Examples}
\label{sec:desc-examples}

\begin{wstructure}
<- Nat
    <- Sum of zero, suc
    <- zero case: done
    <- suc case: leave open and done
    -> NatD Z = 1 + Z
\end{wstructure}

Our first example is the natural numbers, or rather its carrier
functor. Our code is presented in the high-level expression language
of Section~\ref{sec:type-propagation}. The translation back to the raw
terms is laborious but should not pose any difficulty. The code is the
following:

\[\stk{
\NatD : \Desc \\
\NatD \mapsto \DSigma{(\EnumT [ \NatZero, \NatSuc{} ])}
                     {[ \DUnit \quad (\DIndx{\DUnit}) ]}
}\]

Let us explain its construction. First, we use $\DSigma{}{}$ to build
a sum between $\NatZero$ and $\NatSuc{}$. In the $\NatZero$ case, we
reach the end of the description: there is no useful payload in that
case. In the $\NatSuc{}$ case, we left one hole open for the recursive
argument, and we close the description.

In a more synthetic notation, we have simply implemented the following
functor:

\[    \NatD\: Z = 1 + Z    \]



\begin{wstructure}
<- List
    <- Sum of nil, cons
    <- nil case: done
    <- cons case: product of X with leave open and done
    -> ListD X Z = 1 + X * Z
\end{wstructure}

With a small change to the definition of $\NatD$, we obtain the
carrier of lists:

\[\stk{
\ListD : \Set \To \Desc \\
\ListD \: X \mapsto \DSigma{(\EnumT [ \ListNil, \ListCons ])}
                           {[ \DUnit \quad (\DSigma{X}{\LAM{\_} \DIndx{\DUnit}}) ]}
}\]

The modification consists in turning the $\NatSuc{}$ into a proper
$\ListCons$ taking an argument in X followed by an inductive
argument. In this case, we use $\DSigma{}{}$ in its product
interpretation: we pack an element of $X$ together with the recursive
argument. Easily, one sees that this code actually defines the
following functor:

\[    \ListD\: X Z = 1 + X \times Z     \]

\begin{wstructure}
<- Tree
    <- sum of leaf, node
    <- leaf case: done
    <- node case: product of X with two leave open and done
    -> TreeD X Z = 1 + X * Z * Z
\end{wstructure}

Finally, we are not limited to one recursive argument. This is
demonstrated by our description of a binary tree functor below:

\[\stk{
\TreeD : \Set \To \Desc \\
\begin{array}{@{}ll}
\TreeD \: X \mapsto \DSigma{ & (\EnumT [ \TreeLeaf, \TreeNode ]) \\}
                           { & [ \DUnit \quad (\DSigma{X}{\LAM{\_} \DIndx{(\DIndx{\DUnit})}}) ]}
\end{array}
}\]

Again, we are at one evolutionary step away from $\ListD$. However,
instead of single appeal to the induction code, we call it twice. The
interpretation of this code corresponds to the following functor:

\[    \TreeD\: X Z = 1 + X \times Z \times Z     \]


\begin{wstructure}
<- Tagged description
    <- Form TDesc = List (UId x Desc) [equation]
    <- Follow usual sums-of-product presentation of data-type
        <- Finite set of constructors
        <- Then whatever you want
    -> Any Desc data-type can be turned into this form
        -> No loss of expressive power
        /> Guarantee a ``constructor form''
\end{wstructure}

From the examples above, we observe that data-types are defined by a
$\DSigma$ which first argument is a finite set of constructor. The
descriptions fitting into this pattern are called \emph{tagged}
description. Formally, we have:

\note{We have talked about a tagged description based on $\List (\UId
  \times \Desc)$. It makes the free monad easier to construct but,
  imho, considerably clutters anything else. I've reverted back to a
  more sigma-friendly presentation.}


\[
 \TagDesc \mapsto \SIGMA{\V{E}}{\EnumU} \spi{\V{E}}{\LAM{\_} \Desc}
\]

Again, this is a clear reminiscence of the ``sum-of-products'' style:
we have a finite sum of constructors. Of course, every description can
be expressed into this style, by using a singleton as single
constructor.

\note{I think we need a function mapping a $\TagDesc$ to the
  corresponding $\Desc$. This function should go here.}

\begin{wstructure}
<- Fictive object [figure 'data Desc']
    -> Must be read as a type signature
    -> See further for its actual implementation
        <- Subject to our levitation exercise
\end{wstructure}

So far, for convenience, we have taken as granted the existence of
$\Desc$, as presented in Figure~\ref{fig:desc_universe}. In
particular, we have considered its code, $\DSigma{}{}$, $\DIndx{}$,
and $\DUnit$ as type formers, extending the basic type theory. Hence,
in an implementation of this type theory, we would have extended it
with these constructors and their typing rules. Although it makes no
conceptual difference, the code of description should rather be read
as a specification. We promise the existence of such objects,
satisfying the typing rules. It will be the subject of
Section~\ref{sec:desc-levitate} to fulfil this promise, by actually
implementing the specification.

\subsection{Fix-point}
\label{sec:desc-fix-point}

\begin{wstructure}
<- Build the fix-point of functors
    <- See examples: need to build their initial algebra
    -> Extend the type theory with Mu/Con [figure]
        <- Straightforward definition of a fix-point
            <- Interpret D with (Mu D) as sub-objects
\end{wstructure}

\note{ Strictly positive types anyone? }

So far, we have used our universe of descriptions to build
functors. To illustrate its usage, we have implemented the signature
functors of natural numbers, lists, and binary trees. However, the
class of functors expressible with descriptions enjoy another
property: they all admit a fix-point.

\note{ I'm afraid that this section is killed by my categorical gibberish. }

Let us constructively prove this. First of all, for an object $D$
describing a signature functor, we define its fix-point by the type
former $\Mu{D}$. Inhabitants of the fix-point are materialised by the
structure map: $\Con{x}$ inhabits $\Mu{D}$ if $x$ belongs to the
interpretation of $D$ with the carrier of the initial algebra:

\[
\Rule{\Gamma \vdash \Bhab{D}{\Desc}}
     {\Gamma \vdash \Bhab{\Mu{D}}{\Set}} \qquad
\Rule{\Gamma \vdash \Bhab{D}{\Desc} \quad 
      \Gamma \vdash \Bhab{x}{\descop{D}{(\Mu{D})}}}
     {\Gamma \vdash \Bhab{\Con{x}}{\Mu{D}}}
\]

For the less categorically inclined reader, we can simply read this
construction as \emph{tying the knot}: the holes left open by the
functor are filled by its own recursive definition. Hence, we have
recovered the data-types we are usually able to define in simply-typed
languages.

\begin{wstructure}
<- Elimination on Mu
    <- We are used to foldD : \forall X. (desc D X -> X) -> mu D -> X
        /> Not dependent
        -> Cannot express some (which one again?) properties
    -> Develop a dependent induction
        <- Everywhere/All
        <- Induction
    -> *Generic*
    ???
\end{wstructure}

Together with the type formers defined above, we would legibly expect
an elimination principle. Following the categorical intuition, we
could be tempted to provide it as a catamorphism:

\[
\F{cata} : \PITEL{D}{\Desc}
           \PI{T}{\Set}
           (\descop{D}{T} \To T) \To 
           \Mu{D} \To T 
\]

Whereas this definition is amply sufficient in the simply-typed world,
it comes short of its promises in the dependently-typed realm. Indeed,
this type is absolutely non dependent. As this operator will be the
corner stone of any inductive definition, we are better off sharping
it before use.

The first step consists in turning $T$ into a predicate
$\Bhab{P}{\Mu{D} \To \Set}$, hence introducing a dependency on the
descriptions. Consequently, we strengthen the algebra $\descop{D}{T}
\To T$ by a similar process. We start with turning the arrow into a
dependently arrow, hence offering $\Bhab{xs}{\descop{D}{(\Mu{D})}}$ to
further analysis. 

Then, we have to translate the notion of algebra in our dependent
setting. This is achieved by the $\All{D}{X}{P}{x}$ predicate that,
intuitively, states that $\Bhab{P}{X \To \Set}$ holds everywhere in
the sub-structures of $x$. Based on that knowledge, we ought to be
able to prove that $P$ holds for $\Con{xs}$. We close our elimination
principle by taking an argument $x$ in $\Mu{D}$ and claiming that the
motive $P$ is realised on $x$.

\[
\begin{array}{lcll}
\F{induction} & : & \multicolumn{2}{l}{\PITEL{D}{\Desc}}                   \\
              &   & \multicolumn{2}{l}{\PITEL{P}{\Mu{D} \To \Set}}         \\
              &   & \PITEL{m}{& \PI{xs}{\descop{D}{(\Mu{D})}}              \\
              &   &           & \All{D}{(\Mu{D})}{P}{xs} \To P (\Con{xs})} \\
              &   & \multicolumn{2}{l}{\PI{x}{\Mu{D}} P x}
\end{array}
\]

At this stage, we are still left with implementing $\F{All}$ and the
$\F{induction}$ principle. For space reason, we refer the reader to
our technical report for the actual code. Having this dependent
catamorphism, we are empowered with the ability to $\emph{compute}$
over descriptions. Note that this induction principle is the first
manifestation of a generic operation over descriptions. Thanks to
$\F{induction}$, any data-type we define automatically comes equipped
with an induction principle.

\note{Shall we give the code of $\All$?}

\begin{wstructure}
<- Extending type propagation
    <- Data-type declaration turns into definitions
        -> Straightforward translation to Desc
        -> Creation of a variable referring to the structure
    <- Labelled Mu
        /> Just mention the possibility of labelling, no details required
        -> For the user, objects have names rather than Mu of codes
    <- Push Mu to an applied name [figure]
        -> Direct integration into the type propagation machinery
    -> Coded presentation is practical
        <- The user never see a code
\end{wstructure}


In this section, we have introduced a universe of description. We have
shown how to manipulate these codes to build data-types. We have also
presented a fix-point construction and its associated dependent
catamorphism. However, one could argue that programming with these
objects is not practical. 

Whereas simply-typed languages feature a convenient grammar for
defining data-types and manipulating them, our proposal could be seen
as a step backward. In our setting, the user would have to write a
description instead of a sugared data-type definition. Then, to
recognise an element of $\Nat$ for instance, she would have to
identify $\Mu{}$ of a description as indeed the fix-point of the
functor signature of natural numbers. Finally, she would have to write
codes instead of the natural data-type constructor.

As for finite sets previously, these symptoms clearly indicate that we
suffer from too much type information. Just as for finite, the cure
lies in rationalising this information. Let us first tackle the issue
of data-type declaration: unsurprisingly, the grammar of data-types we
are used can be straightforwardly translated to the
``sigmas-of-sigmas'' paradigm. Hence, a data-type definition is turned
into the corresponding description. Then, at the level of definitions,
we associate the name of the data-type to the fix-point of the
description. The induction principle as well as other generic
operations follow for free.

Doing so, we seal data-type definition behind a convenient
abstraction. We take advantage of this abstraction barrier to tamper
the second issue. Namely, our user would rather be presented a name
rather that a fix-point of a signature functor. Rather surprisingly,
it is sufficient to \emph{label} the $\Mu{}$ constructor with the
user-provided name. When pretty-printing this object, we just expose
the label, instead of the definition in full.

Regaining the ability to write data-type constructor is simply a
matter of type propagation. We therefore extends the type-checking
framework of Figure~\ref{fig:type-checking} with the following
inference rule:

\note{ This is rather drafty. I should explain what $\vec{e}$ and $\widetilde{e'}$ are. }

\[
\Rule{\forall e_i \in \vec{e}, \propag{\push{e_i}
                                            {\descop{f\: i}{(\Mu{(\DSigma{(\Enum\: b)}{f})})}}}
                                      {e'_i}}
     {\propag{\push{C\: \vec{e}}{\Mu{(\DSigma{(\Enum\: b)}{f})}}}
             {\Con{(\etag{C}, \widetilde{e'})}}}\;\etag{C} \in b
\]

Hence, in the high-level expression language, the user types a
familiar type constructors, applied to a telescope of arguments. In
the type propagation setting, this expression will be pushed to a
$\Mu{}$ type of a tagged description. It is then straightforward to
look-up the tag and turn the constructor form into the appropriate
code.

This concludes our presentation of the universe of descriptions. In
this section, we have seen how data-types from the simply-typed world
can be expressed in our dependent-type setting. By relying on a
universe construction technique, we benefit from the ability to
inspect and manipulate these codes. This opens some appealing
opportunities for generic programming. Finally, we have demonstrated
that programming with codes is practical. To this end, we have
extended the type propagation machinery. In this setting, we can
define, interact, and build data-types as transparently as in the
simply-typed world.


%%%%%%%%%%%%%%%%%%%%%%%%%%%%%%%%%%%%%%%%%%%%%%%%%%%%%%%%%%%%%%%%
%% Levitating the universe of descriptions
%%%%%%%%%%%%%%%%%%%%%%%%%%%%%%%%%%%%%%%%%%%%%%%%%%%%%%%%%%%%%%%%

\section{Levitating the universe of descriptions}

\subsection{Implementing finite sets}

\begin{structure}
<- Recall typing rules of 1st section
    -> Make clear they were just promises
    -> Can be implemented now
        <- Simply List UId
\end{structure}

\[\stk{
\EnumU : \Set \\
\EnumU \mapsto \Mu{(\List~\UId)}
}\]


\begin{structure}
<- Consequences
    -> Type theory doesn't need to be extended with EnumU, NilE, and ConsE
        <- EnumU == Mu EnumUD
        <- NilE, ConsE are just tags
    -> Do not need a specific \spi eliminator
        <- \spi is an instance of the generic eliminator
            <- Code?
    -> Anything else remains the same (switch, EnumT, 0, 1+)
\end{structure}

\begin{structure}
<- Summary of the operation
    <- The content of the type theory is exactly the same
        <- before == after
    /> type naming scheme condenses
        <- Replace named constructors by codes in the universe of data-types
    -> Our next step is a similar move (in essence)
        /> Condenses the entire naming scheme of data-types
\end{structure}

\subsection{Implementing descriptions}

\begin{structure}
<- Realizing our promises
    <- We are going to implement Desc
    /> Desc is itself a data-type
        <- Same situation as EnumU
            <- We want to benefit from generic operations
        -> It ought to be encoded in itself
\end{structure}

\subsubsection{First try}

\begin{structure}
<- A partial implementation
    <- '1 and 'indx are easy
    <- 'sigma is partially doable
        /> lack the ability to do an higher-order inductive call
    -> Show partial code [figure]
\end{structure}

\[\stk{
\DescD : \Desc \\
\begin{array}{@{}ll}
\DescD \mapsto \DSigma{}{} & (\EnumT [ \DUnit, \DSigma{}{}, \DIndx{} ]) \\
                           & \left[\begin{array}{l}
                                   \DUnit                                \\
                                   \DSigma{\Set}{(\LAM{\V{X}} ???)} \\
                                   \DIndx{\DUnit}                                  \\
                                   \end{array}
                             \right]
\end{array}
}\]

\subsubsection{Second try}

\begin{structure}
<- Extending the universe of description
    -> With higher-order induction
    <- Intuition: index elements in X by H, and go on reading
        -> indx is isomorph to hindx for H = 1
    /> Keep indx
        <- First order!
        -> Extensionally equal to hindx 1
        /> Practically, definitional equality on Sigma/Pi cannot cope with it
    -> Show DescD code
\end{structure}


\[
\stk{
\data \Desc : \Set \where \\
\;\;\begin{array}{@{}l@{\::\:\:}l@{\quad}l}
    \ldots          & \:\:\ldots \\
    \DHindx         & \PI{H}{Set} \Desc \To \Desc
\end{array}
}
\]

\[\stk{
\descop{\_\:}{} : \Desc \To \Set \To \Set \\
\begin{array}{@{}l@{\:=\:\:}ll}
\ldots                        &  \ldots \\
\descop{\DHindx{H}{D}}{X}     &  \TIMES{(H \To X)}{\descop{D}{X}}
\end{array}
}\]


\[\stk{
\DescD : \Desc \\
\begin{array}{@{}ll}
\DescD \mapsto \DSigma{}{} & (\EnumT [ \DUnit, \DSigma{}{}, \DIndx{}, \DHindx{}{} ]) \\
                           & \left[\begin{array}{l}
                                   \DUnit                                \\
                                   \DSigma{\Set}{(\LAM{\V{X}} \DHindx{X}{\DUnit})} \\
                                   \DIndx{\DUnit}                                  \\
                                   \DSigma{\Set}{(\LAM{\_} \DIndx{\DUnit})}
                                   \end{array}
                             \right]
\end{array}
}
\]


\subsubsection{Final move}

\begin{structure}
<- Subtlety: translation of [ ... ]
    -> Let us do it manually
        -> Code with problem for the motive of switch
\end{structure}

\begin{structure}
<- The magician trick
    <- Our problem is to give a motive for switch
        /> We perfectly know what it ought to be: \_ -> DescD
    -> Solution: extend the type theory with a special purpose switchD
        -> Only extension required to the type theory!
        -> Hidden away to the user by the syntactic sugar
            -> Sufficient to ensure unavailability as a raw operator
            <- Another instance of type propagation
\end{structure}

\begin{structure}
<- Generic programming now!
    <- Desc is just data
        -> Can be manipulated
    <- Free induction scheme on Desc
        -> Ability to inspect data-types
        -> Ability to program on data-types
\end{structure}

\subsubsection{Desc, atomically}

\begin{structure}
<- Adding hindx have introduced some duplication
    <- indx == hindx 1
    -> We can factor out commonalities 
        /> Obtain an equivalent presentation
        /> Still embeddable (refer to the Agda model)
\end{structure}

\begin{structure}
<- Give new presentation [figure]
    <- hindx have introduced the notion of function space: 'Pi
    <- hindx and indx are both composed by a binary product and a left open term: 'x and 'id 
    -> Straightforward translation to the new system [equation]
\end{structure}

\[
\stk{
\data \Desc : \Set \where \\
\;\;\begin{array}{@{}l@{\::\:}l@{\quad}l}
    \DId            & \Desc                                   \\
    \DUnit          & \Desc                                   \\
    \DProd{}{}      & \PI{\V{D}, \V{D'}}{\Desc} \Desc         \\
    \DSigma         & \PI{\V{S}}{\Set} \PIS{S \To \Desc} \Desc \\
    \DPi            & \PI{\V{S}}{\Set} \PIS{S \To \Desc} \Desc 
\end{array}
}
\]

\[\stk{
\descop{\_\:}{} : \Desc \To \Set \To \Set \\
\begin{array}{@{}l@{\:=\:\:}ll}
\descop{\DId}{X}          &  X                                           \\
\descop{\DUnit}{X}        &  \Unit                                       \\
\descop{\DProd{D}{D'}}{X} &  \TIMES{\descop{D}{X}}{\descop{D'}{X}}       \\
\descop{\DSigma{S}{D}}{X} &  \SIGMA{\V{s}}{S} \descop{D\: s}{X}                \\
\descop{\DPi{S}{D}}{X}    &  \PI{\V{s}}{S} \descop{D\: s}{X}            
\end{array}
}\]

\[\begin{array}{l@{\:\mapsto\:\:}l}
\DIndx{D}         & \DProd{\DId}{D}                      \\
\DHindx{H}{D}     & \DProd{(\DPi{H}{(\LAM{\_} \DId)})}{D}
\end{array}
\]


\subsection{The generic catamorphism}

\begin{structure}
<- Making cata
    <- Present the type signature
    <- Starts with a call to generic induction
        <- induction on Desc!
        /> Show types at hand
        -> Explain how to use inductive hypothesis
    <- Implement the 'replace' function
    -> Dependent-typeless catamorphism 
\end{structure}

This \(\F{induction}\) operator is the natural dependent elimination
principle, but we might also benefit from the traditional \emph{catamorphism}
or `fold operator' which accompanies a (weakly) initial algebra. We should
like to have
\[\stk{
\F{cata} : \PITEL{D}{\Desc}
           \PI{T}{\Set}
           (\descop{D}{T}\To T) \To 
           \Mu{D} \To T \\
\F{cata}\: D\: T\: f \mapsto
  \F{induction}\: D\: (\LAM{\_}T)\: (\LAM{xs\:ts} f\: ?)
}\]
but what should \(?\) be? We have \(xs:\descop{D}{\Mu{D}}\)
and \(ts:\All{D}{(\Mu{D})}{(\LAM{\_}T)}{xs}\), so surely we can construct
an element of \(\descop{D}{T}\) by replacing each recursive component from
\(xs\) with its counterpart from \(ts\).

\[\stk{
\F{replace} : \stk{\PITEL{D}{\Desc}
                   \PITEL{X,Y}{\Set}\\
                   \PI{xs}{\descop{D}{X}} 
                   \All{D}{X}{(\LAM{\_}Y)}{xs} \To
                   \descop{D}{Y}} \\
\F{replace}\: \DUnit\:          X\: Y\: \Void\:          \Void          \mapsto \Void \\
\F{replace}\: (\DSigma{S}{D})\: X\: Y\: \pair{s}{xs}{}\: ys             \mapsto
    \pair{s}{\F{replace}\: {D~s}\: X\: Y\: xs\: ys}{}         \\
\F{replace}\: (\DIndx{D})\:     X\: Y\: \pair{x}{xs}{}\: \pair{y}{ys}{} \mapsto
    \pair{y}{\F{replace}\: D\: X\: Y\: xs\: ys}{}
}\]



\begin{structure}
<- Deriving generic functions
    <- Taking a Desc and computing a function
        <- Desc comes equipped with an induction principle
        -> Ability to compute more functions from it
            -> More generic functions
    <- Inspecting data-types
        <- All described by a Desc code
        -> Ability to explore the code
            <- Desc equipped with an induction principle
            -> Build new objects based on that structure
\end{structure}

\subsection{The generic Free Monad}

\begin{structure}
!!! EARLIER !!!
<- Tagged description
    <- Form TDesc = List (UId x Desc) [equation]
    <- Follow usual sums-of-product presentation of data-type
        <- Finite set of constructors
        <- Then whatever you want
    -> Any Desc data-type can be turned into this form
        -> No loss of expressive power
        /> Garantee a ``constructor form''
\end{structure}

A \emph{tagged} description is given by an inhabitant of
\[
 \F{TagDesc} \mapsto \List{\TIMES{\UId}{\Desc}}
\]


Datatypes specified in the conventional `sum of products' style naturally give rise to tagged descriptions. Of course, every description can be dorced into this style with a singleton choice of tag.


\begin{structure}
<- A generic program: the free monad construction
    <- Recall free monad construction in Haskell
        -> Based on a functor F
    <- Note that the free monad construction is itself defined by a functor
        -> Extract it
    <- Encode it in the Desc world [equation]
        <- F is the Desc we start with
        <- The free monad functor is what we have just defined
        <- [\_]* : Desc -> Set -> Desc
           [\_]* D X = 'cons ['var ('sigma X (\_ -> '1))] D
        -> Mu does the fix-point
\end{structure}


We may then implement the \emph{free monad} construction as a
transformation on \(\F{TagDesc}\).
\[\stk{
\FreeMonad{\_} : \F{TagDesc} \To \Set \To \F{TagDesc} \\
\FreeMonad{\pair{E}{D}{}}\:X \mapsto
\pair{\ListCons{\:\DVar{}}{E}}{\pair{\DConst{X}}{D}{}}{}
}\]
Of course, we must equip the resulting datatypes with operations delivering a monadic interface. As usual, \(\LAM{\x}\DVar{\x}\) performs the r\^ole of `return', embedding variables into terms. We shall also need a generic \emph{substitution} operator.


\begin{structure}
<- A generic program: monadic substitution [equation]
    <- subst : \forall T X Y. mu ([T]* X) -> (X -> mu ([T]* Y)) -> mu ([T]* Y)
        -> Using Fold
    -> Consequences
        <- We have free monad data-type
            <- Term + variables
        <- We have monad operations
            <- Return / var
            <- Substitution / bind
\end{structure}

\begin{structure}
<- Deriving new data-structure and functions on them
    <- Computing the Free Monad of a data-type
        <- Derive new data-structure from previous one
            <- It is just code
        /> New data-structure comes with some equipment
    <- Computing new functions on computed data-types
        <- If data comes with structure, we ought to be able to capture it
            <- Induction on Desc
            -> Ability to compute over data
\end{structure}


%%%%%%%%%%%%%%%%%%%%%%%%%%%%%%%%%%%%%%%%%%%%%%%%%%%%%%%%%%%%%%%%
%% Indexing descriptions
%%%%%%%%%%%%%%%%%%%%%%%%%%%%%%%%%%%%%%%%%%%%%%%%%%%%%%%%%%%%%%%%

\section{A Universe of Inductive Families}
\label{sec:indexing-desc}

\newcommand{\vtup}[2]{\bigRedBracket{\begin{array}{@{}#1@{}}#2\end{array}}}

\begin{wstructure}
!!! Need Help !!!
<- Motivation
    <- Desc: expressivity of simply-typed datatypes: inductive types
        <- Values do not influence types
    /> Example: Vectors
        <- Cannot be defined by just induction
            <- Vectors of all size need to be defined at the *same* time
            -> Defined as a *family* of types
                -> Index
        -> I -> IDesc I: Inductive family
    ???
\end{wstructure}

So far, we have explored the realm of inductive types, building on
intuition from ML-like datatypes, using type dependency as a
descriptive tool in $\Desc$ and its interpretation. Let us now make
dependent types the object as well as the means of our study.

Dependent datatypes provide a way to work at higher level of
precision \emph{a priori}, reducing the sources of failure we
might otherwise need to manage. For the perennial
example, consider \emph{vectors}---lists indexed by length. By
making length explicit in the type, we can prevent hazardous
operations (the type of `head' demands vectors of length
$\NatSuc{\V{n}}$) and offer stronger guarantees (pointwise
addition of $\V{n}$-vectors yields an $\V{n}$-vector).

However, these datatypes are not \emph{individually} inductive. For
instance, we have to define the whole \emph{family} of vectors
mutually, in one go. In dependently typed languages, the basic grammar
of datatypes is that of inductive families. To capture this grammar,
we must account for \emph{indexing}.

%% \subsection{Desc, atomically}
%% \label{sec:idesc-atomic-desc}

%% \begin{wstructure}
%% [Outdated: type former presentation instead]
%% <- Adding hindx have introduced some duplication
%%     <- indx == hindx 1
%%     -> We can factor out commonalities 
%%         /> Obtain an equivalent presentation
%%         /> Still embeddable (refer to the Agda model)
%% \end{wstructure}

%% \begin{wstructure}
%% <- Also replacing '1 by 'const  [figure]
%%     <- For convenience
%%         <- 'const X equivalent to 'sigma X (\_ -> '1)
%%         /> Easier to abstract
%%             <- Extensionally same
%%             /> 'const more useful in practice
%% \end{wstructure}

%% Before moving on to indexed descriptions, we have to carry out some
%% maintenance work on descriptions. We presented $\Desc$ as the grammar
%% of inductive types. Hence, the codes closely follow this grammar. In
%% the following, we adopt an alternative presentation. With
%% $\DSigma{\!}{\!}$, we are actually \emph{quoting} a standard
%% type-former, namely

%% $$\Bhab{\Sigma}{\PI{\V{S}}{\Set} \PI{\V{S}}{\Set} \Set}$$

%% In the alternative presentation, we go further and present all our
%% codes as quotations of standard type-formers. This presentation is
%% shown in Figure~\ref{fig:type-former-desc}. The reader will notice
%% that we replace $\DUnit$ by a more general $\DConst{\!}$ code. Whereas
%% $\DUnit$ was interpreted as the unit set, $\DConst{\V{X}}$ is
%% interpreted as $\V{X}$, for any $\Bhab{\V{X}}{\Set}$. Extensionally,
%% $\DConst{\V{X}}$ and $\DSigma{\V{X}}{\DUnit}$ are equivalent. However,
%% $\DConst{\!}$ is more succinct. More importantly, $\DConst$ is
%% \emph{first-order}, unlike its equivalent encoding. From a
%% definitional perspective, we are giving more opportunities to the
%% type-system, hence reducing the burden on the programmer. For the same
%% reason, we introduce $\DProd{\!}{\!}$ that overlaps with
%% $\DSigma{\!}{\!}$.

%% This reorganisation is strictly equivalent to the previous one
%% (Fig.~\ref{fig:hindx_desc}). Just as the previous version, it is also
%% self-descriptive. We refer the reader to the companion technical
%% report for details. In this finer-grained presentation, we can define
%% $\DIndx{\!}$ and $\DHindx{\!}{\!}$ as follow:

%% \[\begin{array}{l@{\:\mapsto\:\:}l}
%% \DIndx{\V{D}}         & \DProd{\DId}{\V{D}}                      \\
%% \DHindx{\V{H}}{\V{D}}     & \DProd{(\DPi{\V{H}}{(\LAM{\_} \DId)})}{\V{D}}
%% \end{array}
%% \]

%% Consequently, the examples previously developed can be
%% straightforwardly translated into this new presentation. For example,
%% here is the new definition of $\NatD$:

%% \[\stk{
%% \NatD : \Desc \\
%% \NatD \mapsto \DSigma{(\EnumT{[ \NatZero, \NatSuc{\!} ]})}
%%                      {[ \DUnit \quad \DId ]}
%% }\]


%% In the following, we adopt this last version as our de
%% facto universe of inductive types. In particular, we are going to
%% evolve this presentation into an indexed one.

%% \note{Shall we talk about the Type Theory being Desc Zero? or such story?}

%% \begin{figure}

%% \[\stk{
%% \begin{array}{ll}
%% \stk{
%% \data \Desc : \Set \where                                      \\
%% \;\;\begin{array}{@{}l@{\::\:\:}l@{\quad}l}
%%     \DId            & \Desc                                    \\
%%     \DConst{\!}     & \Set \To \Desc                           \\
%%     \DProd{\!}{\!}  & \PI{\V{D}, \V{D'}}{\Desc} \Desc          \\
%%     \DSigma{\!}{\!} & \PI{\V{S}}{\Set} \PIS{\V{S} \To \Desc} \Desc \\
%%     \DPi{\!}{\!}    & \PI{\V{S}}{\Set} \PIS{\V{S} \To \Desc} \Desc 
%% \end{array}
%% }
%% \vspace{0.2in}
%% \\
%% \stk{
%% \descop{\_\:}{} : \Desc \To \Set \To \Set \\
%% \begin{array}{@{}l@{\:=\:\:}ll}
%% \descop{\DId}{\V{X}}          &  \V{X}                                           \\
%% \descop{\DConst{\V{Z}}}{\V{X}}    &  \V{Z}                                           \\
%% \descop{\DProd{\V{D}}{\V{D'}}}{\V{X}} &  \TIMES{\descop{\V{D}}{\V{X}}}{\descop{\V{D}\V{'}}{\V{X}}}       \\
%% \descop{\DSigma{\V{S}}{\V{D}}}{\V{X}} &  \SIGMA{\V{s}}{\V{S}} \descop{\V{D}\: \V{s}}{\V{X}}          \\
%% \descop{\DPi{\V{S}}{\V{D}}}{\V{X}}    &  \PI{\V{s}}{\V{S}} \descop{\V{D}\: \V{s}}{\V{X}}            
%% \end{array}
%% }
%% \end{array}
%% }\]

%% \caption{Universe of descriptions based on Type-formers}
%% \label{fig:type-former-desc}

%% \end{figure}

\subsection{The universe of indexed descriptions}

\begin{wstructure}
<- Labelling Id
    <- We had: data Desc : Set -> Set
    -> We want: data IDesc : (I -> Set) -> Set
        <- Indexed functor (?)
        -> It is sufficient to label Id
            <- Where the functor is built
\end{wstructure}


\newcommand{\DotTo}{\mathop{\blue{\dot{\rightarrow}}}}

We presented the $\Desc$ universe as a
grammar of strictly positive endofunctors on $\Set$ and developed
inductive types by taking a fixpoint. To describe inductive families
indexed by some $\Bhab{\V{I}}{\Set}$, we play a similar game with
endofunctors on the category $\Set^{\V{I}}$,
families of sets \(\V{X},\V{Y}:\V{I}\To\Set\) for objects, and for morphisms,
families of functions in \(\V{X}\DotTo\V{Y}\), defined pointwise:
\[
\V{X}\DotTo\V{Y} \mapsto \PI{\V{i}}{\V{I}}\V{X}\:\V{i}\To\V{Y}\:\V{i}
\]

An \emph{indexed functor} in $\Set^{\V{I}}\To\Set^{\V{J}}$ has the
flavour of a device driver, characterising `responses' to a given
request in \(\V{J}\) where we may in turn make `subrequests' at indices
chosen from \(\V{I}\). When we use indexed functors to define inductive
families of datatypes,
\(\V{I}\) and \(\V{J}\) coincide: we explain how to make a node fit a given
index, including subnodes at chosen indices. E.g., if we are asked for a
vector of length 3, we choose to ask in turn for a tail of length 2.

To code up valid notions of response to a given request, we introduce
$\SYMBIDesc$ and its interpretation:
%
\[\stk{
\IDesc{(\Bhab{\V{I}}{\Set})} : \Set \smallskip \\
\idescop{\_}{}{} : _{\PI{\V{I}}{\Set}} \IDesc{\V{I}} \To (\V{I} \To \Set) \To \Set    \\
}\]

An \(\IDesc{\V{I}}\) specifies just \emph{one} response, but a
request-to-response \emph{function},
$\V{R}:\V{I} \To \IDesc{\V{I}}$, yields a strictly positive endofunctor
\[
  \LAM{\V{X}} \LAM{\V{i}} \idescop{\V{R}\:\V{i}}{\V{I}}{\V{X}} :
  \Set^{\V{I}} \To \Set^{\V{I}}
\]
whose fixpoint we then take:
%
\[\stkl{
\Rule{\Gamma \vdash \Bhab{\V{I}}{\Set} \qquad
      \Gamma \vdash \Bhab{\V{R}}{\V{I} \To \IDesc{\V{I}}}}
     {\Gamma \vdash \Bhab{\SYMBIMu_{\V{I}}{\V{R}}}{\V{I}\To\Set}} \qquad
\\
\Rule{\begin{array}{l@{\qquad}l}
          \Gamma \vdash \Bhab{\V{I}}{\Set} &
          \Gamma \vdash \Bhab{\V{R}}{\V{I} \To \IDesc{\V{I}}} \\
          \Gamma \vdash \Bhab{\V{i}}{\V{I}} &
          \Gamma \vdash \Bhab{\V{x}}{\idescop{\V{R}\:\V{i}}{\V{I}}{(\SYMBIMu_{\V{I}}{\V{R}})}}
      \end{array}}
     {\Gamma \vdash \Bhab{\Con{\V{x}}}{\IMu{\V{I}}{\V{R}}{\V{i}}}}
}\]

\newcommand{\upgrade}{\F{upgrade}}
\newcommand{\inductionI}{\F{indI}}
\newcommand{\cataI}{\F{cataI}}

We define the $\SYMBIDesc$ grammar in Figure~\ref{fig:idesc},
delivering only \emph{strictly positive} families. As well as
indexing our descriptions, we have refactored a little, adopting
a more compositional algebra of codes, where $\Desc$ is
biased towards the right-nested tuples. We now have
\(\DVar{i}\) for recursive `subrequests' at a chosen index \(i\),
with tupling by right-associative
\(\DProd{}{}\) and higher-order branching
by \(\DPi{\!}{\!}\).  Upgrade your old $\Desc$
to a trivially indexed $\IDesc{\Unit}$ as follows!
\[\begin{array}{@{}ll}
\upgrade :\Desc & \To\IDesc{\Unit} \\
\upgrade\:\DUnit & \mapsto \DConst{\Unit} \\
\upgrade\:(\DSigma{\V{S}}{\V{D}}) &
   \mapsto \DSigma{\V{S}}{\LAM{\V{s}}\upgrade\:(\V{D}\:\V{s})} \\
\upgrade\:(\DIndx{\V{D}}) & \mapsto
  \DProd{\DVar{\Void}}{\upgrade\:\V{D}} \\
\upgrade\:(\DHindx{\V{H}}{\V{D}}) & \mapsto
  \DProd{(\DPi{\V{H}}{\LAM{\_}\DVar{\Void}})}{\upgrade\:\V{D}} \\
\end{array}\]

To deliver induction for indexed datatypes, we need the `holds everywhere'
machinery. We present $\SYMBAllI$ and $\SYMBallI$ in
Figure~\ref{fig:allI-predicates}, with a twist---where
$\Desc$ admits the $\SYMBall$ construction, $\SYMBIDesc$ is \emph{closed}
under it! The $\SYMBAllI$
operator for a description indexed on \(\V{I}\) is strictly positive in
turn, and has a description indexed on
 some \(\SIGMA{\V{i}}{\V{I}}{\V{X}\: \V{i}}\).
Induction on indexed descriptions is then hardwired thus:
%
\[\stk{
\begin{array}{@{}ll}
\inductionI : & _{\PI{\V{I}}{\Set}}
                   \PITEL{\V{R}}{\V{I} \To \IDesc{\V{I}}}
                   \PI{\V{P}}{(\SIGMA{\V{i}}{\V{I}}{\IMu{\V{I}}{\V{R}}{\V{i}}}) \To \Set} \\
                 & (      \PITEL{\V{i}}{\V{I}} 
                          \PI{\V{xs}}{\idescop{\V{R}\: \V{i}}{\V{I}}{(\SYMBIMu_{\V{I}}{\V{R}})}} \\
                 & \   \idescop{\AllI{}
                                     {(\V{R}\: \V{i})}
                                     {(\SYMBIMu_{\V{I}}{\V{R}})}
                                     {\V{xs}}}
                               {}
                               {\V{P}} \To
                       \V{P}\: \pair{\V{i}}{\Con{\V{xs}}}{}) \To \\
                 & \PITEL{\V{i}}{\V{I}}
                   \PI{\V{x}}{\IMu{\V{I}}{\V{R}}{\V{i}}}
                   \V{P}\: \pair{\V{i}}{\V{x}}{}
\end{array} \\
\inductionI\: \V{R}\: \V{P}\: \V{m}\: \V{i}\: (\Con{\V{xs}}) \mapsto 
    \V{m}\: \V{i}\: \V{xs}\: (\allI{}
                                    {\V{R}\: \V{i}}
                                    {(\SYMBIMu_{\V{I}}{\V{R}})}
                                    {\V{P}}
                                    {(\spl{\LAM{\V{i}}\LAM{\V{xs}} \inductionI\: \V{R}\: \V{P}\: \V{m}})}
                                    {\V{xs}})
}\]
%
The generic catamorphism, $\cataI$, is constructed from $\inductionI$
as before. Its type becomes more elaborated, to deal with the
indexing:
%
\[
\begin{array}{@{}l@{}l}
\cataI :& \PITEL{\V{I}}{\Set}
          \PITEL{\V{R}}{\V{I} \To \IDesc{\V{I}}} \\
        & \PI{\V{T}}{\V{I} \To \Set}
          (\PI{\V{i}}{\V{I}}{\idescop{\V{R}\: \V{i}}{}{\V{T}} \To \V{T}\: \V{i}}) \To
          \SYMBIMu_{\V{I}}{\V{R}} \DotTo \V{T}
\end{array}
\]


\begin{figure*}

\[
\begin{array}{ll}
%%
\stk{
\begin{array}{@{}ll}
\SYMBAllI : & _{\PI{\V{I}}{\Set}}
              \PITEL{\V{D}}{\IDesc{\V{I}}}
              \PI{\V{X}}{\V{I} \To \Set} \\
            & \idescop{\V{D}}{\V{I}}{\V{X}} \To
              \IDesc{(\SIGMA{\V{i}}{\V{I}}{\V{X}\: \V{i}})}
\end{array} \\
\begin{array}{@{}l@{}l@{\:\mapsto\:\:}l}
\AllI{\:}{(\DVar{\V{i}})}{& \V{X}}{\V{x}} &
    \DVar{\pair{\V{i}}{\V{x}}{}} \\
\AllI{\:}{(\DConst{\V{K}})}{& \V{X}}{\V{k}} &
    \DConst{\Unit} \\
\AllI{\:}{(\DProd{\V{D}}{\V{D'}})}{& \V{X}}{\pair{\V{d}}{\V{d'}}{}} &
    \DProd{\AllI{}{\V{D}}{\V{X}}{\V{d}}}{\AllI{}{\V{D'}}{\V{X}}{\V{d'}}} \\
\AllI{\:}{(\DSigma{\V{S}}{\V{D}})}{& \V{X}}{\pair{\V{s}}{\V{d}}{}} &
    \AllI{}{(\V{D}\: \V{s})}{\V{X}}{\V{d}} \\
\AllI{\:}{(\DPi{\V{S}}{\V{D}})}{& \V{X}}{\V{f}} &
    \DPi{\V{S}}{\LAM{\V{s}} \AllI{}{(\V{D}\: \V{s})}{\V{X}}{(\V{f}\: \V{s})}}
\end{array}
}
&
%%
\stk{
\begin{array}{@{}ll}
\SYMBallI : & _{\PI{\V{I}}{\Set}}
              \PITEL{\V{D}}{\IDesc{\V{I}}}
              \PITEL{\V{X}}{\V{I} \To \Set} 
              \PI{\V{P}}{(\SIGMA{\V{i}}{\V{I}}{\V{X}\: \V{i}}) \To \Set} \\
            & (\PI{\V{x}}{\SIGMA{\V{i}}{\V{I}}{\V{X}\: \V{i}}} \V{P}\: \V{x}) \To
              \PI{\V{xs}}{\idescop{\V{D}}{\V{I}}{\V{X}}} 
              \idescop{\AllI{}{\V{D}}{\V{X}}{\V{xs}}}{}{\V{P}}
\end{array} \\
\begin{array}{@{}l@{}l@{\:\mapsto\:\:}l}
\allI{\:}{(\DVar{\V{i}})}{& \V{X}}{\V{P}}{\V{p}}{\V{x}} &
    \V{p}\: \pair{\V{i}}{\V{x}}{} \\
\allI{\:}{(\DConst{\V{K}})}{& \V{X}}{\V{P}}{\V{p}}{\V{k}} &
    \void \\
\allI{\:}{(\DProd{\V{D}}{\V{D'}})}{& \V{X}}{\V{P}}{\V{p}}{\pair{\V{d}}{\V{d'}}{}} &
    \pair{\allI{}{\V{D}}{\V{X}}{\V{P}}{\V{p}}{\V{d}}}
         {\allI{}{\V{D'}}{\V{X}}{\V{P}}{\V{p}}{\V{d'}}}{} \\
\allI{\:}{(\DSigma{\V{S}}{\V{D}})}{& \V{X}}{\V{P}}{\V{p}}{\pair{\V{s}}{\V{d}}{}} &
    \allI{}{(\V{D}\: \V{s})}{\V{X}}{\V{P}}{\V{p}}{\V{d}} \\
\allI{\:}{(\DPi{\V{S}}{\V{D}})}{& \V{X}}{\V{P}}{\V{p}}{\V{f}} &
    \LAM{\V{a}}\allI{}{(\V{D}\: \V{a})}{\V{X}}{\V{P}}{\V{p}}{(\V{f}\: \V{a})}
\end{array}
\end{array}
}
\]

\caption{Indexed induction predicates}
\label{fig:allI-predicates}

\end{figure*}


\begin{figure}

\[\stk{\begin{array}{@{}ll}
\IDesc{(\Bhab{\V{I}}{\Set})} &: \Set \\
\DVar{(\Bhab{\V{i}}{\V{I}})} &: \IDesc{\V{I}} \\
\DConst{(\Bhab{\V{A}}{\Set})} &:\IDesc{\V{I}}       \\
\DProd{(\Bhab{\V{D}}{\IDesc{\V{I}}})}{(\Bhab{\V{D}}{\IDesc{\V{I}}})}
  & :\IDesc{\V{I}}       \\
\DSigma{(\Bhab{\V{S}}{\Set})}{(\Bhab{\V{D}}{\V{S}\To\IDesc{\V{I}}})}
& : \IDesc{\V{I}}  \\
\DPi{(\Bhab{\V{S}}{\Set})}{(\Bhab{\V{D}}{\V{S}\To\IDesc{\V{I}}})}
& : \IDesc{\V{I}}  \\
\end{array}\smallskip \\
\idescop{\_\:}{}{} :_{\PI{\V{I}}{\Set}} \IDesc{\V{I}} \To (\V{\V{I}} \To \Set) \To \Set                  \\
\begin{array}{@{}l@{\V{X}}@{\:\mapsto\:\:}ll}
\idescop{\DVar{\V{i}}}{\V{I}}{&}      &  \V{X}\: \V{i}                                           \\
\idescop{\DConst{\V{K}}}{\V{I}}{&}    &  \V{K}                                                   \\
\idescop{\DProd{\V{D}}{\V{D'}}}{\V{I}}{&} &  \TIMES{\idescop{\V{D}}{\V{I}}{\V{X}}}{\idescop{\V{D'}}{\V{I}}{\V{X}}}       \\
\idescop{\DSigma{\V{S}}{\V{D}}}{\V{I}}{&} &  \SIGMA{\V{s}}{\V{S}} \idescop{\V{D}\: \V{s}}{\V{I}}{\V{X}}                    \\
\idescop{\DPi{\V{S}}{\V{D}}}{\V{I}}{&}    &  \PI{\V{s}}{\V{S}} \idescop{\V{D}\: \V{s}}{\V{I}}{\V{X}}            
\end{array}
}
\]

\caption{Universe of indexed descriptions}
\label{fig:idesc}

\end{figure}



\subsection{Examples}
\label{sec:idesc-examples}

\paragraph{Natural numbers:}

\begin{wstructure}
<- Nat
    -> [equation]
    <- Non-indexed types lives in IDesc 1
        -> This applies to all previous examples
\end{wstructure}

For basic reassurance, we \(\upgrade\:\NatD\):
%
\[\stk{
\upgrade\:\NatD : \IDesc{\Unit} \\
\upgrade\:\NatD \mapsto \DSigma{(\EnumT{\sqr{\NatZero\: \SYMBNatSuc}})}
                     {\sqr{(\DConst{\Unit}) \; 
                           (\DProd{\DVar{\Void}}{\DConst{\Unit}})}}
}\]
%
Note that trailing \(\Unit\)'s keep our right-nested, \(\void\)-terminated
tuple structure, and with it our elaboration machinery.
We can similarly \(\upgrade\) any inductive type.
Moreover, \(\IDesc{I}\) can now code a bunch of mutually
inductive types, if \(I\) enumerates the
bunch~\cite{paulin:habilitation, yakushev:mutual-def}.



\paragraph{Indexed descriptions:}

\begin{wstructure}
<- Levitation [figure]
    <- Following Desc encoding
        /> Note: simple datatype
            -> Live in IDesc 1
    -> Behind the scene, relies on the special purpose switchD
\end{wstructure}

Note that $\IDesc{\V{I}}$ is a plain inductive type, parametrised
by \(\V{I}\), but indexed trivially.
%
\[\stk{
\IDescD : \PI{\V{I}}{\Set} \IDesc{\Unit} \\
\IDescD\: \V{I} \mapsto \SYMBDSigma \\
\quad
 \EnumT\vtup{r}{\SYMBDVar\\
                \SYMBDConst\\
                \DProd{}{}\\
                \SYMBDSigma\\
                \SYMBDPi}
         \; \vtup{l@{}l}{
  (\DProd{\DConst{\V{I}} &}{\DConst{\Unit}})                  \\
  (\DProd{\DConst{\Set}  &}{\DConst{\Unit}})                  \\
  (\DProd{\DVar{\Void}}{\DProd{\DVar{\Void}&}{\DConst{\Unit}}})  \\
  (\DSigma{\Set}{\LAM{\V{S}}
     \DProd{( \DPi{\V{S}}{\LAM{\_} \DVar{\Void}}) &}{\DConst{\Unit}}})     \\
  (\DSigma{\Set}{\LAM{\V{S}}
     \DProd{( \DPi{\V{S}}{\LAM{\_} \DVar{\Void}}) &}{\DConst{\Unit}}})     \\
                                   }
}\]

Therefore, this universe is self-describing and can be
levitated. As before, we rely on a special purpose $\F{switchID}$
operator to build the finite function $\bigRedBracket{\ldots}$
without mentioning \(\SYMBIDesc\).

\paragraph{Vectors:}

\newcommand{\VecD}{\F{VecD}}
\newcommand{\VecNil}{\etag{\CN{vnil}}}
\newcommand{\SYMBVecCons}{\etag{\CN{vcons}}\xspace}
\newcommand{\VecCons}[2]{\SYMBVecCons\:#1\:#2}

So far, our examples live in $\IDesc{\Unit}$, with no interesting
indexing. Let us at least have vectors. Recall
that the constructors $\VecNil$ and $\SYMBVecCons$ are defined only for
$\NatZero$ and $\NatSuc$ respectively:
%
\[
\stk{
\data \D{Vec}\: \PITEL{\V{X}}{\Set} : \PI{\V{i}}{\Nat} \Set \where \\
\;\;\begin{array}{@{}l@{\::\:\:}l@{\quad}l}
    \VecNil          & \D{Vec}\:\V{X}\:{\NatZero}   \\
    \SYMBVecCons & _{\PI{\V{n}}{\Nat}}\V{X} \To \D{Vec}\:{\V{X}}\:{\V{n}} \To \D{Vec}\:{\V{X}}\:{(\NatSuc{\V{n}})}
\end{array}
}
\]

One way to code constrained datatypes is to appeal to a suitable
notion of propositional equality \(\PropEq\) on indices. The
constraints are expressed as `Henry Ford' equations in the datatype.
For vectors:
%
\[\stk{
\VecD : \Set \To \Nat \To \IDesc{\Nat} \\
\VecD\: \V{X}\: \V{i} \mapsto \SYMBDSigma\\
\quad
\EnumT{\vtup{r}{\VecNil\\ \SYMBVecCons}}
\; \vtup{r}{
                            (\DConst{(\NatZero\PropEq\V{i})}) \\
 ( \DSigma{\Nat}{\LAM{\V{n}}
   \DProd{\DConst{\V{X}}}
     {\DProd{\DVar{\V{n}}}{\DConst{(\NatSuc{\V{n}}\PropEq\V{i})}})}}
                          }
}\]

You may choose $\VecNil$ for any index you like as long as it is
$\NatZero$; in the $\SYMBVecCons$ case, the length of the tail is
given explicitly, and the index $\V{i}$ must be one more. Our previous
\(\Unit\)-terminated tuple types can now be seen as the trivial case
of constraint-terminated tuple types, with elaboration supplying the
witnesses when trivial.

In this paper, we remain anxiously agnostic about
propositional equality. Any will do, according to
conviction; many variations are popular. The
homogeneous identity type used in Coq is ill-suited to
dependent types, but its heterogeneous variant (forming equations
regardless of type) allows the translation of pattern
matching with structural recursion to
\(\F{indI}\)~\cite{goguen:pattern-matching}. The
extensional equality of \citet{altenkirch:ott} also sustains the translation.

\begin{wstructure}
!!! Need Help !!!
<- Brady optimisation: forcing
    <- Source to source translation
    <- Able to remove some constraints
    -> Example: Fin [figure]
    ??? More technical detail needed
\end{wstructure}

However, sometimes, the equations are redundant. 
Looking back at $\D{Vec}$, we find that the equations constrain
the choice of constructor and stored tail index retrospectively.
But \emph{inductive families need not store their
  indices}~\cite{brady:index-inductive-families}!  If we
analyse the incoming index, we can tidy our description of $\D{Vec}$
as follows:
%
\[\stk{
\VecD \:\PITEL{\V{X}}{\Set} : \Nat \To \IDesc{\Nat} \\
\begin{array}{@{}lll}
\VecD\:\V{X}\: \NatZero     & \mapsto & \DConst{\Unit} \\
\VecD\:\V{X}\: (\NatSuc{\V{n}}) & \mapsto &
 \DProd{\DConst{\V{X}}}{\DVar{\V{n}}}
\end{array}
                                       
}\]
%
The constructors and equations have simply disappeared. A similar
example is $\SYMBFin$ (bounded numbers), specified by:
%
\[
\stk{
\data \SYMBFin : \PI{\V{n}}{\Nat} \Set \where \\
\;\;\begin{array}{@{}l@{\::\:\:}l@{\quad}l}
    \FinZero      & _{\PI{\V{n}}{\Nat}}\Fin{(\NatSuc{\V{n}})}   \\
    \SYMBFinSuc   & _{\PI{\V{n}}{\Nat}}\Fin{\V{n}} \To \Fin{(\NatSuc{\V{n}})}
\end{array}
}\]
%
In this case, we can eliminate equations but not constructors, since both
$\FinZero$ and $\SYMBFinSuc$ both target $\SYMBNatSuc$:
%
\[\stk{
\FinD : \Nat \To \IDesc{\Nat} \\
\begin{array}{@{}lll}
\FinD\: \NatZero         & \mapsto & \DSigma{\EnumT{\Void}}{\Void} \\
\FinD\: (\NatSuc{\V{n}}) & \mapsto & \DSigma{\EnumT{\sqr{\FinZero\: \SYMBFinSuc}}}
                                            {\sqr{(\DConst{\Unit})\: (\DVar{\V{n}})}}
\end{array}
}\]

This technique of extracting information by case analysis on indices
applies to descriptions exactly where Brady's `forcing' and
`detagging' optimisations apply in compilation. They eliminate just
those constructors, indices and constraints which are redundant even
in \emph{open} computation. In \emph{closed} computation, where proofs
can be trusted, all constraints are dropped.


\paragraph{Tagged indexed descriptions:}

\newcommand{\SYMBmuide}{\D{\({\mu}^{\!+}\)}\xspace}
\newcommand{\muide}[2]{\SYMBmuide\!\!_{#1}\:#2}

Let us reflect this index analysis technique.
We can divide a description of tagged indexed data in two: first, the
constructors that do not depend on the index; then, the constructors
that do. The non-dependent part mirrors the definition for non-indexed
descriptions. The index-dependent part simply indexes the choice of
constructors by $\V{I}$. Hence, by inspecting the index, it is
possible to vary the `menu' of constructors.
%
\[
\begin{array}{@{}l@{\:\mapsto\:\:}l}
 \TagIDesc{\V{I}}  & \TIMES{\ATagIDesc{\V{I}}}{\ITagIDesc{\V{I}}} \\
 \ATagIDesc{\V{I}} & \SIGMA{\V{E}}{\EnumU} \PI{\V{i}}{\V{I}} \spi{\V{E}}{\LAM{\_} \IDesc{\V{I}}} \\
 \ITagIDesc{\V{I}} & 
     \SIGMA{\V{F}}{\V{I} \To \EnumU} \PI{\V{i}}{\V{I}} \spi{(\V{F}\: \V{i})}{\LAM{\_} \IDesc{\V{I}}} 
\end{array}
\]

\begin{wstructure}
<- Vectors
    Do we treat them in the end? 
    What can we say here we haven't with Fin?
\end{wstructure}

In the case of a tagged $\D{Vec}$, for instance, for the index
$\NatZero$, we would only propose the constructor
$\ListNil$. Similarly, for $\NatSuc{n}$, we would only propose the
constructor $\SYMBListCons$.

We write $\toIDesc{\V{D}}\:\V{i}$ to denote the $\IDesc{\V{I}}$
computed from the tagged indexed description $\V{D}$ at index
$\V{i}$. Its expansion is similar to the definition of \(\SYMBtoDesc\)
for tagged descriptions, except that it must also append the two parts.
We again write $\muide{\V{I}}{\V{D}}$ for
$\IMu{\V{I}}{(\toIDesc{\V{D}})}$.

\paragraph{Typed expressions:}

\begin{wstructure}
<- Hutton's razor
    <- Types
        <- 'Nat
        <- 'Bool
    <- Term [figure]
        <- val : Val 'a -> 'a  for Val : Ty -> Set, mapping to Nat and Bool
        <- cond : 'Bool -> a -> a -> a
        <- plus : 'Nat -> 'Nat -> 'Nat
        <- le : 'Nat -> 'Nat -> 'Bool
\end{wstructure}

%% Types
\newcommand{\Ty}{\D{Ty}}
\newcommand{\Ebool}{\etag{\CN{bool}}}
\newcommand{\Enat}{\etag{\CN{nat}}}

%% Constructors
\newcommand{\SYMBEval}{\etag{\CN{val}}\xspace}
\newcommand{\SYMBEvar}{\etag{\CN{var}}\xspace}
\newcommand{\Eval}[1]{\SYMBEval\:#1}
\newcommand{\SYMBEcond}{\etag{\CN{cond}}\xspace}
\newcommand{\Econd}[3]{\SYMBEcond\:#1\:#2\:#3}
\newcommand{\SYMBEplus}{\etag{\CN{plus}}\xspace}
\newcommand{\Eplus}[2]{\SYMBEplus\:#1\:#2}
\newcommand{\SYMBEle}{\etag{\CN{le}}\xspace}
\newcommand{\Ele}[2]{\SYMBEle\:#1\:#2}

%% Index mapper (terminology?)
\newcommand{\SYMBVal}{\F{Val}\xspace}
\newcommand{\Val}[1]{\SYMBVal\:#1}
\newcommand{\SYMBVar}{\F{Var}\xspace}
\newcommand{\Var}[2]{\SYMBVar\: #1\: #2}

%% Hutton expressions
\newcommand{\HExprD}{\F{ExprD}}
\newcommand{\HExprAD}{\F{ExprAD}}
\newcommand{\HExprID}{\F{ExprID}}
\newcommand{\HExprVarD}[1]{\C{ExprD}_{\F{Var},#1}}
\newcommand{\HExprFreeD}{\C{ExprD}^{\C{Free}}}
\newcommand{\HExprAFreeD}{\C{ExprAD}^{\C{Free}}}

We are going to define a syntax for a small language with
two types, natural numbers and booleans:
%
\[
\Ty \mapsto \EnumT{\sqr{\Enat\: \Ebool}}
\]

\newcommand{\plusHost}{\mathop{\green{+_{\mathrm{H}}}}}
\newcommand{\leHost}{\mathop{\green{\leq_{\mathrm{H}}}}}


This language has values, conditional expression, addition and
comparison. Informally, their types are:
%
\[
\begin{array}{l@{\::\:\:}l}
\SYMBEval            & \Val{\V{ty}} \To \V{ty} \\
\SYMBEcond           & \Ebool \To \V{ty} \To \V{ty} \To \V{ty}  \\ 
\end{array}
\qquad
\begin{array}{l@{\::\:\:}l}
\SYMBEplus           & \Enat \To \Enat \To \Enat                           \\
\SYMBEle             & \Enat \To \Enat \To \Ebool                          \\
\end{array}
\]
%
The function $\SYMBVal$ interprets object language types in the
host language, so that arguments to $\SYMBEval$ fit their
expected type.
%
\[\stk{
\SYMBVal : \Ty \To \Set \\
\begin{array}{@{}l@{\:\mapsto\:\:}l}
\Val{\Enat}   & \Nat \\
\Val{\Ebool}  & \Bool
\end{array}
}\]
%
We take $\Nat$ and $\Bool$ to represent natural numbers and Booleans
in the host language, equipped with addition $\plusHost$ and
comparison $\leHost$.

We express our syntax as a tagged indexed description, indexing over
object language types $\Ty$. We note that some constructors are always
available, namely $\SYMBEval$ and $\SYMBEcond$. On the other hand,
$\SYMBEplus$ and $\SYMBEle$ constructors are index-dependent, with
$\SYMBEplus$ available just when building a $\Enat$, $\SYMBEle$ just
for $\Ebool$. The code, below, reflects this intuition, with the first
component uniformly offering $\SYMBEval$ and $\SYMBEcond$, the second
selectively offering $\SYMBEplus$ or $\SYMBEle$.
%
%%% \begin{figure}
%
\[\stk{
\stk{
\HExprD : \TagIDesc{\Ty} \\
\HExprD \mapsto \sqr{ \HExprAD , \HExprID } \\
} \smallskip\\
\stk{
\HExprAD : \ATagIDesc{\Ty} \\
\HExprAD \mapsto \vtup{l}{
   {\vtup{r}{\SYMBEval\\ \SYMBEcond \,}} \red{,} \;
      \LAM{\V{ty}}
      \vtup{l@{\:}l}{
      \DProd{\DConst{(\Val{\V{ty}})}&}{\DConst{\Unit}} \\
      \DProd{\DProd{\DVar{\Ebool}}{\DProd{\DVar{\V{ty}}}{\DVar{\V{ty}}}}&}
        {\DConst{\Unit}} \\
     }
   }
\smallskip\\
} 
\\
\stk{
\HExprID : \ITagIDesc{\Ty} \\
\HExprID \mapsto \vtup{l}{
                   \vtup{r}{\sqr{\SYMBEplus} \\ \sqr{\SYMBEle}} \red{,} \;
  \LAM{\_} \sqr{\DProd{\DProd{\DVar{\Enat}}{\DVar{\Enat}}}{\DConst{\Unit}}}
                   }
}
}\]

%%%\caption{Syntax of typed expressions}
%%%\label{fig:hexpr-full}

%%%\end{figure}

\newcommand{\evalH}{\F{eval}_{\green{\Downarrow}}}
\newcommand{\evalOne}{\F{eval}_{\green{\downarrow}}}

Given the syntax, let us supply the semantics. We implement an
evaluator as a catamorphism:
%
\[\stk{
\evalH : \PI{\V{ty}}{\Ty} 
         \muide{\Ty}{\HExprD}\: \V{ty} \To
         \Val{\V{ty}} \\
\evalH\: \V{ty}\: \V{term} \mapsto \cataI_{\Ty} \:
                                 (\toIDesc{\HExprD})\: 
                                 \SYMBVal\: 
                                 \evalOne\: 
                                 \V{ty}\: 
                                 \V{term}
}\]
%
To finish the job, we must supply the algebra which implements a single
step of evaluation, given subexpressions evaluated already.
%
\[\stk{
\evalOne : \PI{\V{ty}}{\Ty}
 \idescop{(\toIDesc{\HExprD})\:\V{ty}}{\Ty}{\SYMBVal}
           \To {\Val{\V{ty}}} \\
\begin{array}{@{}l@{}c@{}l@{\:\mapsto\:\:}l}
\evalOne\: & \_\: & (\SYMBEval\;\V{x})                                            & \V{x} \\
\evalOne\: & \_\: & (\SYMBEcond\:\BoolTrue\:\V{x}\:\_)   & \V{x} \\
\evalOne\: & \_\: & (\SYMBEcond\:\BoolFalse\:\_\:\V{y})  & \V{y} \\
\evalOne\: & \Enat\: & (\SYMBEplus\:\V{x}\:\V{y})   & \V{x} \plusHost \V{y} \\
\evalOne\: & \Ebool\: & (\SYMBEle\:\V{x}\:\V{y})  & \V{x} \leHost \V{y} 
\end{array}
}\]

\begin{wstructure}
    /> Closed term
        <- only constants and operations on them
        -> Extend Val with Var : Ty -> Set, mapping to EnumU
            -> Open term
            -> Language of well-typed terms
                <- By construction
\end{wstructure}

Hence, we have a type-safe syntax and a tagless interpreter for our
language, in the spirit
of~\citet{augustsson.carlsson:dependent.interpreter}, with help from
the generic catamorphism. However, so far, we are only able to define
and manipulate \emph{closed} terms. Adding variables, it is possible
to build and manipulate \emph{open} terms, that is, terms in a
context. We shall get this representation, for free, thanks to the
\emph{free indexed monad} construction.


\subsection{Free indexed monad}

\begin{wstructure}
<- Variation on a theme: free imonad construction
    <- Recall existence of generic free monad construction
    -> Present its generalisation to IDesc [equation]
        <- \I -> IDesc I as describing an indexed endofunctor
        <- Free monad construction
    -> Still a suitable, generic notion of substitution
        <- show type signature
        <- show implementation?? (space! space!)
\end{wstructure}

In Section~\ref{sec:desc-free-monad}, we have built a free monad
operation for simple descriptions. The process is similar in the
indexed world. Namely, given an indexed functor, we derive the indexed
functor coding its free monad: \note{pwm: Whoa there. Maybe we should
  say something about IMonads in general before we get to this point?}
%
\[\stk{
\begin{array}{ll}
\FreeIMonad{\_}{} : & _{\PI{\V{I}}{\Set}}
                     \PITEL{\V{R}}{\TagIDesc{\V{I}}} 
                     \PITEL{\V{X}}{\V{I} \To \Set}\To 
                      \TagIDesc{\V{I}}
\end{array} \\
\FreeIMonad{\pair{\V{E}}{\V{F}}{}}{\V{I}}{\V{R}} \mapsto
    \pair{\pair{\ListCons{\SYMBDVar}{(\fst{\V{E}})}} 
               {\LAM{\V{i}}
                \pair{\DConst{(\V{R}\: \V{i})}}
                     {(\snd{\V{E}})\: \V{i}}{}}{}}
         {\V{F}}{}
}\]


\newcommand{\substI}{\F{substI}}


Just as in the universe of descriptions, this construction comes with
an obvious \return and a substitution operation, the \bind. Its
definition is the following:
%
\[\stk{
\begin{array}{@{}ll}
\substI : & _{\PI{\V{I}}{\Set}}
            \PI{\V{X}, \V{Y}}{\V{I} \To \Set}
            \PITEL{\V{R}}{\TagIDesc{\V{I}}} \\
          & (\V{X} \DotTo 
             \muide{\V{I}}{(\FreeIMonad{\V{R}}{\V{I}}{\V{Y}})}) \To 
             \muide{\V{I}}{(\FreeIMonad{\V{R}}{\V{I}}{\V{X}})} \DotTo
             \muide{\V{I}}{(\FreeIMonad{\V{R}}{\V{I}}{\V{Y}})}
\end{array} \\
\substI\: \V{X}\: \V{Y}\: \V{R}\: \V{\sigma}\: \V{i}\: \V{t} \mapsto \\
\qquad    \cataI_{\V{I}}\: (\toIDesc{\FreeIMonad{\V{R}}{}{\V{X}}})\:
                      (\muide{\V{Y}}{(\FreeIMonad{\V{R}}{}{\V{Y}})})\:
                      (\F{applyI}\: \V{R}\: \V{X}\: \V{Y}\: \V{\sigma})\:
                      \V{i}\:
                      \V{t} 
}\]
% 
where  $\F{applyI}$ is defined as follows:
%
\[\stk{
\begin{array}{@{}ll}
\F{applyI} : & _{\PI{\V{I}}{\Set}}
            \PITEL{\V{R}}{\TagIDesc{\V{I}}}
            \PI{\V{X}, \V{Y}}{\V{I} \To \Set} \\
          & (\V{X} \DotTo \muide{\V{I}}{(\FreeIMonad{\V{R}}{\V{I}}{\V{Y}})}{}) \To \\
          & \idescop{\toIDesc{\FreeIMonad{\V{R}}
                                         {\V{I}}
                                         {\V{X}}}}
                    {\V{I}}
                    {\muide{\V{I}}{(\FreeIMonad{\V{R}}{\V{I}}{\V{Y}})}} \DotTo 
            \muide{\V{I}}{(\FreeIMonad{\V{R}}{\V{I}}{\V{Y}})}{}
\end{array} \\
\begin{array}{@{}l@{\:\mapsto\:\:}l}
\F{applyI}\: \V{R}\: \V{X}\: \V{Y}\: \V{\sigma}\: \V{i}\: \pair{\SYMBDVar}{\V{x}}{}   & \V{\sigma}\: \V{i}\: \V{x}                   \\
\F{applyI}\: \V{R}\: \V{X}\: \V{Y}\: \V{\sigma}\: \V{i}\: \pair{\V{c}}{\V{ys}}{} & \Con{\pair{\V{c}}{\V{ys}}{}}
\end{array}
}\]
 
The subscripted types corresponds to implicit arguments that can be
automatically inferred, hence do not have to be typed in. Let us now
consider two examples of free indexed monads.


\paragraph{Typed expressions:}

\begin{wstructure}
    /> Closed term
        <- only constants and operations on them
        -> Extend Val with Var : Ty -> Set, mapping to EnumU
            -> Open term
            -> Language of well-typed terms
                <- By construction
\end{wstructure}

\newcommand{\Ctxt}{\D{Context}}
\newcommand{\SYMBCtxtEmpty}{\C{[]}\xspace}
\newcommand{\CtxtEmpty}{\SYMBCtxtEmpty}
\newcommand{\SYMBCtxtSnoc}{\C{snoc}\xspace}
\newcommand{\CtxtSnoc}[2]{\SYMBCtxtSnoc\:#1\:#2}
\newcommand{\SYMBEnv}{\F{Env}}
\newcommand{\Env}[1]{\SYMBEnv\: #1}
\newcommand{\SYMBlookup}{\F{lookup}}
\newcommand{\lookup}[4]{\SYMBlookup\: #1\: #2\: #3\: #4}

In the previous section, we presented a language of closed
arithmetic expressions. Using the free monad construction, we are
going to extend this construction to open terms. An open term is
defined with respect to a context, represented by a snoc-list of
types:
%
\[
\begin{array}{@{}l@{\::\:\:}l@{\quad}l}
\Ctxt           & \Set \\
\SYMBCtxtEmpty  & \Ctxt \\
\SYMBCtxtSnoc   & \Ctxt \To \Ty \To \Ctxt
\end{array}
\]
%
An environment realises the context, packing a value for each type:
%
\[
\stk{
\SYMBEnv : \Ctxt \To \Set \\
\begin{array}{@{}l@{\:\:\mapsto\:\:}l}
\Env{\CtxtEmpty}                & \Unit \\
\Env{(\CtxtSnoc{\V{G}}{\V{S}})} & \TIMES{\Env{\V{G}}}{\Val{\V{S}}}
\end{array}
}\]
%
In this setting, we define type variables, $\SYMBVar$ by:
%
\[\stk{
\Var{}{} : \Ctxt \To \Ty \To \Set \\
\begin{array}{@{}ll@{\:\:\mapsto\:\:}l}
\Var{\CtxtEmpty}{& \V{T}}                & 
    \Void \\
\Var{(\CtxtSnoc{\V{G}}{\V{S}})}{& \V{T}} & 
    \SUM{(\Var{\V{G}}{\V{T}})}{(\V{S} \PropEq \V{T})}
\end{array}
}\]
%
While $\SYMBVal$ maps the type to the corresponding host type,
$\SYMBVar$ indexes a value in the context, obtaining a proof that the
types match. The $\SYMBlookup$ function precisely follow this
semantics:
%
\[\stk{
\SYMBlookup : \PI{\V{G}}{\Ctxt} 
          \Env{\V{G}} \To 
          \PI{\V{T}}{\Ty} 
          \Var{\V{G}}{\V{T}} \To
          \Val{\V{T}} \\
\begin{array}{@{}l@{}l@{}l@{}l@{}lll}
\lookup{& (\CtxtSnoc{\V{G}}{.T})}{& \pair{\V{g}}{\V{t}}{}}{& \V{T}}{& (\SumRight{\C{refl}})} & \mapsto & \V{t} \\
\lookup{& (\CtxtSnoc{\V{G}}{\V{S}})}{& \pair{\V{g}}{\V{t}}{}}{& \V{T}}{& (\SumLeft{\V{x}})} & \mapsto & \lookup{\V{G}}{\V{g}}{\V{T}}{\V{x}} 
\end{array}
}\]

\newcommand{\SYMBEmpty}{\F{Empty}\xspace}
\newcommand{\Empty}[1]{\SYMBEmpty\:#1}

\newcommand{\SYMBopenTerm}{\F{openTm}\xspace}
\newcommand{\openTerm}[1]{\SYMBopenTerm\: #1}
\newcommand{\closeTerm}{\F{closeTm}}

\newcommand{\update}{\F{update}}

Consequently, taking the free monad of \(\HExprD\) by \(\SYMBVar\:
\V{G}\), we obtain the language of open terms in a context \(\V{G}\):
%
\[
\openTerm{\V{G}} \mapsto \FreeIMonad{\HExprD}{\Ty}{(\SYMBVar\:\V{G})}
\]
%
In this setting, the language of closed terms corresponds to the free
monad assigning an empty set of values to variables
%
\[
\closeTerm \mapsto \FreeIMonad{\HExprD}{\Ty}{\SYMBEmpty}
\quad
\mbox{where}
\quad
\stk{
\SYMBEmpty : \Ty \To \Set \\
\begin{array}{@{}l@{\:\:\mapsto\:\:}l}
\Empty{\_}   & \Zero \\
\end{array}
}\]
%
Allowing variables from an empty set is much like forbidding variables,
so \(\closeTerm\) and \(\HExprD\) describe isomorphic
datatypes. Correspondingly, you can update an old \(\HExprD\) to a shiny
\(\closeTerm\):
%
\[\stk{
\update : \muide{\Ty}{\HExprD} \DotTo \muide{\Ty}{\closeTerm} \\
\begin{array}{@{}l@{}l}
\update\: \V{ty}\: \V{tm} \mapsto \cataI_{\Ty}\: & (\toIDesc{\HExprD})\:
                                                  (\muide{\Ty}{\closeTerm})\: \\
                                                & (\LAM{\_} \LAM{\pair{\V{tag}}{\V{tm}}{}} \Con{\pair{\Su{\V{tag}}}{\V{tm}}{}})\:
                                                  \V{ty}\:
                                                  \V{tm}
\end{array}
}\]
% 
The other direction of the isomorphism is straightforward, the
$\SYMBDVar$ case being impossible. Therefore, we are entitled to
reuse the $\evalH$ function to define the semantics of $\closeTerm$.

\newcommand{\discharge}{\F{discharge}}

Now we would like to give a semantics to the open term language. We
proceed in two steps: first, we substitute variables by their value in
the context; then, we evaluate the resulting closed term. Thanks to
$\evalH$, the second problem is already solved. Let us focus on
substituting variables from the context. Again, we can subdivide this
problem: first, discharging a single variable from the context; then,
applying this $\discharge$ function on every variables in the term.

The $\discharge$ function is relative to the required type and a
context of the right type. Its action is to map values to themselves,
and variables to their value in context. This corresponds to the
following function:
%
\[\stk{
\begin{array}{@{}ll}
\discharge : & \PI{\V{G}}{\Ctxt}
               \Env{\V{G}} \To 
               \Var{\V{G}}{} \DotTo
               \muide{\Ty}{\closeTerm}
\end{array} \\
\begin{array}{@{}l@{\:\mapsto\:\:}l}
\discharge\: \V{G}\: \V{g}\: \V{ty}\: \V{v} &
    \Con{\pair{\SYMBEval}{\lookup{\V{G}}{\V{g}}{\V{ty}}{\V{v}}}{}}
\end{array}
}\]

\begin{wstructure}
            /> Then, perform subst everywhere in the term
                -> Show type [code]
                /> This is a bind!?
                -> There is some more structure 
                    -> We should try to get it
\end{wstructure}

\newcommand{\substH}{\F{substExpr}}

We are now left with applying $\discharge$ over all variables of the
term.  We simply have to fill in the right arguments to $\substI$, the
type guiding us:
%
\[
\stk{
\begin{array}{@{}ll}
\substH  : & \PI{\V{G}}{\Ctxt} \\
           & (\Var{\V{G}}{} \DotTo
              \muide{\Ty}{\closeTerm}) \DotTo \\
          & \muide{\Ty}{(\openTerm{\V{G}})} \DotTo 
            \muide{\Ty}{\closeTerm}
\end{array} \\
\substH\: \V{G}\:
          \V{ty}\:          
          \V{g}\:
          \V{\sigma}\: 
          \V{tm} \mapsto  
\substI_{\Ty}\:
               (\SYMBVar\: \V{G})\: 
               \SYMBEmpty\:
               \HExprD\: 
               \V{\sigma}\:
               \V{ty}\:
               \V{tm}
}\]

Hence completing our implementation of the open terms
interpreter. Without much effort, we have described the syntax of a
well-typed language, together with its semantics.


\paragraph{Indexed descriptions:}

An interesting instance of free monad is $\SYMBIDesc$ itself. Indeed,
$\SYMBDVar$ is nothing but the \return. The remaining constructors form
the carrier functor, trivially indexed by $\Unit$. The signature functor
is described as follow:
%
\[\stk{
\IDescFreeD : \ATagIDesc{\Unit} \\
\begin{array}{@{}ll}
\IDescFreeD \mapsto \bigRedBracket{\begin{array}{l}
                                \sqr{\SYMBDConst\:\:
                                     \DProd\:\:
                                     \SYMBDSigma\:\:
                                     \SYMBDPi} \red{,}\\
                                  \LAM{\_}\bigRedBracket{\begin{array}{l}
                                        \DConst{\Set}               \\
                                        \DProd{\DVar{\Void}}{\DVar{\Void}}  \\
                                        \DSigma{\Set}{(\LAM{\V{S}} \DPi{\V{S}}{(\LAM{\_} \DVar{\Void})})} \\
                                        \DSigma{\Set}{(\LAM{\V{S}} \DPi{\V{S}}{(\LAM{\_} \DVar{\Void})})}
                                    \end{array}}\end{array}}
\end{array}
}\]
%
We get $\IDesc{\V{I}}$ by extending the signature with variables from \(\V{I}\):
%
\[\stk{
\IDescD : \PI{\V{I}}{\Set} \TagIDesc{\Unit} \\
\IDescD\: \V{I} \mapsto \FreeIMonad{\red{[}\IDescFreeD\red{,[}\LAM{\_}\sqr{}\red{,}\LAM{\_}\red{[]]]}}{\Unit}\LAM{\_}\V{I}
}\]

The fact that indexed descriptions are closed under substitution
is potentially of considerable utility, if we can exploit this fact:
\[
\idescop{\V{\sigma} \V{D}}{\V{J}}{\V{X}} 
    \mapsto 
        \idescop{\V{D}}
                {\V{I}}
                {\LAM{\V{i}}
                     {\idescop{\V{\sigma} \V{i}}
                              {\V{J}}
                              {\V{X}}}}
        \quad \mbox{where}\;\V{\sigma}:\V{I}\To\IDesc{\V{J}}
\]
By observing that a description can be decomposed via substitution, we
split its meaning into a superstructure of substructures, e.g. a
`database containing salaries', ready for traversal operations
preserving the former and targeting the latter.
 
%\newpage


%%%%%%%%%%%%%%%%%%%%%%%%%%%%%%%%%%%%%%%%%%%%%%%%%%%%%%%%%%%%%%%%
%% Discussion
%%%%%%%%%%%%%%%%%%%%%%%%%%%%%%%%%%%%%%%%%%%%%%%%%%%%%%%%%%%%%%%%

\section{Discussion}
\label{sec:discussion}

\subsection{Universe stratification}

\begin{wstructure}
!!! Need Help !!!
<- Universe stratification
    <- Stratified agda model
        <- Fully stratified
        <- Proof of iso between host and embedding
    ???
\end{wstructure}

As presented, our type theory suffers from a major weakness. Indeed,
we are subject to Girard's paradox, as we assume that $\Set$ lives in
$\Set$. We made that choice for presentational convenience, as
universe stratification is orthogonal to our work. Nonetheless, our
universe of description rather naturally leads itself to
stratification. Unsurprisingly, $\IDesc{\!}$ at level $l$ is of type
$\Set^{\blue{l+1}}$. Similarly, the interpretation of $\IDesc{\!}$ at
level $l$ is an object of type $\Set^{\blue{l}}$:

\[\stk{
\data \IDesc{\!}^{\blue{l}} (\Bhab{\V{I}}{\Set^{\blue{l+1}}}) : \Set^{\blue{l+1}} \where \\
\;\;\ldots \\
\\
\idescop{\_\:}{}{}^{\blue{l}} : \PI{\V{I}}{\Set^{\blue{l+1}}} \IDesc{{\!}^{\blue{l}}\V{I}} \To (\V{I} \To \Set^{\blue{l}}) \To \Set^{\blue{l}}    \\
\ldots
}\]

The operations and examples developed in this paper stratify just as
well. We refer the reader to our Agda model, which takes advantage of
set polymorphism to implement the universe of indexed descriptions at
any level. Further, we have coded $\IDesc{\!}$ in itself and have
proved the isomorphism between the host and the embedded universes.

\subsection{Related Work}

\begin{structure}
!!! Need Help !!!
<- Comparison with Induction Recursion
    ???
\end{structure}


\begin{wstructure}
!!! Need Help !!!
<- Related Work
    <- Generic in simply-typed functional languages
        <- PolyP \cite{jansson:polyp}
        <- Generic Haskell \cite{hinze:generic-haskell}
        <- Scratch your boilerplate \cite{spj:syb}
\end{wstructure}

Generic programming is a vast topic. We refer our reader to Garcia et
al.~\cite{garcia:generic-comparative-study} for a broad overview of
generic support in various languages. In the sole context of Haskell,
there is a myriad of proposals. These approaches are presented and
compared in Hinze et al.~\cite{hinze:generic-approach-comparative} and
Rodriguez et al.~\cite{rodriguez:generic-libs-comparative}.

In particular, our approach is similar in spirit with polytypic
programming, as initiated by PolyP~\cite{jansson:polyp}. Indeed,
generic functions, in our system, are built by induction on the
pattern functor. Unlike PolyP, we do not have to resort to pre-process
data-type definitions: our data-types are, natively, nothing but
codes.

Our approach also support the Generic
Haskell~\cite{hinze:generic-haskell} model. This model, based on
type-indexed data types, enables computing new data-types from
others. This is natural in our system, as data-types descriptions are
first-class.

Another generic programming framework is Scrap Your
Boilerplate~\cite{spj:syb} (SYB). Our proposal is different in various
ways. The corner stone of SYB is the \emph{spine} view of data-type
constructors. A data-type is a spine composed by a constructor applied
to some arguments. Further, this spine is equipped with some
combinators including, primarily, an iterator. In this setting,
generic programs are written by composing these combinators. This
relies on a $\CN{Typeable}$ type-class, allowing dynamic dispatch to
data-type specific operations. As a result, SYB is not reflexive: it
is restricted to data-types instanciating $\CN{Typeable}$. Moreover,
it forbids type-indexed data types: it can only define generic
functions.


\begin{wstructure}
    <- Generic in dependent types
        <- Norell \cite{norell:msc-thesis}
        <- Polytypic prog in Coq \cite{verbruggen:polytype-coq}
        <- Universes for generic prog \cite{benke:universe-generic-prog}
\end{wstructure}

The interest in generic programming for dependent types is not new
either. \citet{norell:msc-thesis} has given a formalization of
polytypic programming in the setting of Alfa, a precursor of
Agda. Similarly, Verbruggen et al.~\cite{verbruggen:polytype-prog-coq,
  verbruggen:polytype-coq} have developed a framework for polytypic
programming in the Coq theorem prover. However, these works aim at
\emph{modelling} PolyP or Generic Haskell in a dependently-typed
setting, for the purpose of proving correctness properties of Haskell
code. Our approach is different in that we aim at building a
foundation for data-types, in a dependently-typed theory, for a
dependently-typed system.

Closer to us is the work by Benke et
al.~\cite{benke:universe-generic-prog}. This seminal work introduced
the usage of universes for developing generic programs. Our own
universes are rather similar to theirs: our universe of descriptions
is similar to their universe of iterated induction, and our universe
of indexed descriptions is isomorphic to their universe of finitary
indexed induction. This is not surprising, as we share the same source
of inspiration, namely induction recursion.

However, we differ in several ways. First, they adopt a generative
perspective: each universe extends the base type theory with both type
formers and elimination rules. Thanks to the levitation, we only rely
on a generic induction and a specialised
$\switchD{\!}{\!}{\!}$. Second, the authors do not tackle the issue of
\emph{programming} with such codes: it is not obvious, to us, how one
can define, manipulate, and program over these coded data-types. With
descriptions, we have shown how to abstract away codes and present a
convenient and familiar presentation to the developer. Finally, the
authors often resort to an extensional notion of equality, while we
have given an equality-agnostic presentation. Beside, our presentation
is arranged so as to use definitional equality as much as
possible. Hence, in practice, the developer is relieved from many
proof obligations.


%%%%%%%%%%%%%%%%%%%%%%%%%%%%%%%%%%%%%%%%%%%%%%%%%%%%%%%%%%%%%%%%
%% Conclusion
%%%%%%%%%%%%%%%%%%%%%%%%%%%%%%%%%%%%%%%%%%%%%%%%%%%%%%%%%%%%%%%%

% \section{Conclusion}
% 
% \begin{wstructure}
% <- System developed in a reasonable theory
%     <- Pi, Sigma, Finite sets
%     /> No assumption about the equality
%     -> Low requirement / high applicability
% \end{wstructure}
% 
% \begin{wstructure}
% <- Formalize a rationnalized presentation of types
%     <- Working directly with codes is not practical
%     -> Bidirectional type-checking
%         <- Type information flows during type checking/type synthesis
%         -> Elaboration turns high-level expressions to low-level terms
%     -> Should not be afraid by codes
% \end{wstructure}
% 
% \begin{wstructure}
% <- Dependently-typed presentation of simple inductive types
%     <- Universe of descriptions
%         <- Based on the specificity of dependent types
%             <- Universe of codes
%             <- Sigma types
%         -> External fix-point and generic induction scheme
%     <- Rationalised by type propagation
%         -> Developer does not see the code
%     <- Self-describing
%         <- Step-by-step exposition
%         -> Minimal extension to the type theory
%             <- Just need fix-point and induction
%         -> Closed presentation of datatype
%             -> Non generative
%         -> datatype is just data
%     <- Generic programming is just programming
%         <- Generic catamorphism
%         <- Generic free monad
% \end{wstructure}
% 
% To introduce our approach, we have presented a universe of
% description. This universe has the expressive power of simple
% inductive types, as found in ML-like languages. Further, we have
% implemented this universe as a self-described object. Hence, for a
% minimal extension of the type-theory, we get a closed, self-describing
% presentation of datatypes, where datatypes are just data.
% 
% \begin{wstructure}
% <- Indexed descriptions for dependent datatypes
%     <- Presented as a slight generalisation of Desc
%         <- Just add indexing
%     <- Develop several examples of datatypes
%         <- Typed syntax
%         <- Constrained datatype a la GADT
%     <- Generic indexed programming
%         <- Indexed free monad
%         <- Substitution
% \end{wstructure}
% 
% To capture dependent datatypes, we generalise our presentation to
% support indexing. The universe of indexed descriptions thus built
% encompasses inductive families. Again, this universe is
% self-described. We have developed several examples of dependent
% datatypes and generic functions over them.
% 
% \begin{wstructure}
% <- All of this without cheating
%     <- Admit a correct stratification
%     <- Terminating
%     <- Strictly-positive types
% \end{wstructure}
% 
% We have presented a self-describing, self-hosted universe for
% datatypes. We have shown the benefit of such approach, by our ability
% to reflect datatypes in the type-theory. This fosters a new way of
% considering generic programming: just as programming. Moreover,
% despite its egg-and-chicken nature, this presentation is free of
% paradox: it has been formalised in Agda, admitting a correct
% stratification.
% 
% \paragraph{Future work:} As such, indexed descriptions do
% not cover several extensions of inductive families. One of them is
% induction-recursion. An interesting question is to locate indexed
% descriptions in the spectrum between inductive families and indexed
% induction-recursion. Another popular extension we plan to consider is
% to allow internal fixpoints and higher-kinded datatypes.
% 
% Also, we have presented a generic notion of syntax with variables,
% thanks to the free monad construction. We would like to explore a
% notion of syntax with binding. Interestingly, introducing internal
% fixpoints or kinds would turn our universe into such syntax with
% binding. Once again, levitation would reveal itself convenient by
% providing generic tools to handle binding.

%%%%%%%%%%%%%%%%%%%%%%%%%%%%%%%%%%%%%%%%%%%%%%%%%%%%%%%%%%%%%%%%
%% Appendices
%%%%%%%%%%%%%%%%%%%%%%%%%%%%%%%%%%%%%%%%%%%%%%%%%%%%%%%%%%%%%%%%

% \appendix
% \section{Appendix Title}

% This is the text of the appendix, if you need one.


%%%%%%%%%%%%%%%%%%%%%%%%%%%%%%%%%%%%%%%%%%%%%%%%%%%%%%%%%%%%%%%%
%% Acknowledgments
%%%%%%%%%%%%%%%%%%%%%%%%%%%%%%%%%%%%%%%%%%%%%%%%%%%%%%%%%%%%%%%%

\acks

We are grateful to Jos\'{e} Pedro Magalh\~{a}es for his helpful
comments on a draft of this paper. We are also grateful to the Agda
team, without which levitation would have been a much more perilous
exercise. J. Chapman was supported by the Estonian Centre of
Excellence in Computer Science, EXCS, financed by the European
Regional Development Fund. P.-\'{E}. Dagand, C. McBride and P. Morris
are supported by the Engineering and Physical Sciences Research
Council, Grants EP/G034699/1 and EP/G034109/1.


%%%%%%%%%%%%%%%%%%%%%%%%%%%%%%%%%%%%%%%%%%%%%%%%%%%%%%%%%%%%%%%%
%% Bibliography
%%%%%%%%%%%%%%%%%%%%%%%%%%%%%%%%%%%%%%%%%%%%%%%%%%%%%%%%%%%%%%%%


\bibliography{paper}
\bibliographystyle{abbrvnat}

% The bibliography should be embedded for final submission.
%\begin{thebibliography}{}
%\softraggedright
%\end{thebibliography}

\end{document}
