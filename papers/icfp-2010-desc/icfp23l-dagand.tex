\documentclass[authoryear]{sigplanconf}

\usepackage[T1]{fontenc}
\usepackage{pslatex}

\usepackage{amsmath}
\usepackage{amssymb}
\usepackage{verbatim}
\usepackage{pig}
\usepackage{url}
\usepackage{xspace}
\usepackage{listings}
\usepackage{flushend}

\include{macros}

%% Haskell-mode for listings
\lstnewenvironment{code}{\lstset{language=Haskell,
                                 basicstyle=\small\ttfamily,
                                 xleftmargin=1cm}}{}

%% Structure
%\newenvironment{structure}{\footnotesize\verbatim}{\endverbatim}
\newenvironment{structure}{\comment}{\endcomment}

%% Written bits of Structure
\newenvironment{wstructure}{\comment}{\endcomment}

%% Comments
%% \setlength{\marginparwidth}{0.7in}
%% \newcommand{\note}[1]{\-\marginpar[\raggedright\footnotesize #1]%
%%                                   {\raggedright\footnotesize #1}}
\newcommand{\note}[1]{}

%% Syntax
\newcommand{\bind}{\emph{bind}\xspace}
\newcommand{\return}{\emph{return}\xspace}

\begin{document}

%\ColourEpigram
\MonochromeEpigram

\conferenceinfo{ICFP'10,} {September 27--29, 2010, Baltimore, Maryland, USA.}
\CopyrightYear{2010}
\copyrightdata{978-1-60558-794-3/10/09}

%\titlebanner{banner above paper title}        % These are ignored unless
%\preprintfooter{short description of paper}   % 'preprint' option specified.



%%%%%%%%%%%%%%%%%%%%%%%%%%%%%%%%%%%%%%%%%%%%%%%%%%%%%%%%%%%%%%%%
%% Title
%%%%%%%%%%%%%%%%%%%%%%%%%%%%%%%%%%%%%%%%%%%%%%%%%%%%%%%%%%%%%%%%


\title{The Gentle Art of Levitation}


%% Alphabetical ordering.
\authorinfo{James Chapman}
           {Institute of Cybernetics, Tallinn University of Technology}
           {james@cs.ioc.ee}
\authorinfo{Pierre-\'{E}variste Dagand \\ Conor McBride}
           {University of Strathclyde}
           {\{dagand,conor\}@cis.strath.ac.uk}
\authorinfo{Peter Morris}
           {University of Nottingham}
           {pwm@cs.nott.ac.uk}


\maketitle


%%%%%%%%%%%%%%%%%%%%%%%%%%%%%%%%%%%%%%%%%%%%%%%%%%%%%%%%%%%%%%%%
%% Abstract
%%%%%%%%%%%%%%%%%%%%%%%%%%%%%%%%%%%%%%%%%%%%%%%%%%%%%%%%%%%%%%%%


\begin{abstract}
  We present a closed dependent type theory whose inductive types
  are given not by a scheme for generative declarations,
  but by encoding in a \emph{universe}. Each inductive datatype arises
  by interpreting its \emph{description}---a
  first-class value in a datatype of descriptions. Moreover, the
  latter itself has a description. Datatype-generic programming thus
  becomes ordinary programming. We show some of the resulting generic
  operations and deploy them in particular, useful ways on the
  datatype of datatype descriptions itself. Simulations in existing
  systems suggest that this
  apparently self-supporting setup is achievable without paradox or
  infinite regress.
\end{abstract}

%\category{CR-number}{subcategory}{third-level}
\category{D.1.1}{Programming Techniques}{Applicative (Functional) Programming}
\category{D.3.3}{Language Constructs and Features}{Data types and structures}

\terms
Design, Languages, Theory



%%%%%%%%%%%%%%%%%%%%%%%%%%%%%%%%%%%%%%%%%%%%%%%%%%%%%%%%%%%%%%%%
%% Introduction
%%%%%%%%%%%%%%%%%%%%%%%%%%%%%%%%%%%%%%%%%%%%%%%%%%%%%%%%%%%%%%%%

\section{Introduction}

Dependent datatypes, such as the ubiquitous vectors (lists indexed by
length) express \emph{relative} notions of data validity. They allow
us to function in a complex world with a higher standard of basic
hygiene than is practical with the context-free datatypes of ML-like
languages. Dependent type systems, as found in
Agda~\cite{norell:agda}, Coq~\cite{coq},
Epigram~\cite{mcbride.mckinna:view-from-the-left}, and contemporary
Haskell~\cite{cheney:gadt, xi:gadt}, are beginning to make themselves useful. As
with rope, the engineering benefits of type indexing sometimes
outweigh the difficulties you can arrange with enough of it.

%Dependent types are an appealing technique for building safer and more
%reliable software. By giving types more expressive power, the
%developer is able to encode more precise invariants in the types. As a
%result, more bugs are caught automatically, during
%type-checking. Because of this benefit, dependently-typed systems have
%flourished, such as Generalized Abstract Data-Types (GADT) in
%Haskell~\cite{spj:gadt}, Agda~\cite{norell:agda},
%Ynot~\cite{morrisett:ynot}, or Epigram~\cite{pigs:epigram}, to name
%but a few.

\begin{wstructure}
<- Describe the problem
    <- Data-types in dependent-type theory
        <- Much more precise
            <- More powerful type-system
            -> Stronger safety guarantees
\end{wstructure}

%In this paper, we will focus on data-types in such systems. Indeed,
%the expressive power of the type-system has a direct impact on
%data-types. Because types can \emph{depend} on terms, our data-types
%can be made more precise. The typical example is vectors, which type
%depend on the size of the vector. Having more precision about
%data-types, we can write safer code: taking the $\CN{head}$ of a
%vector is \emph{ensured} to succeed whenever its type states that it
%is a non-empty vector. This property is automatically enforced by the
%type-checker.

\begin{wstructure}
        <- Equipped with elimination principle
            <- Defining functions over them
            <- Making proofs over them
\end{wstructure}

%Moreover, in total programming systems, such as Agda, Ynot, or
%Epigram, data-types come equipped with an elimination principle: while
%a data-type definition introduces new type formers in the theory, we
%need an eliminator to dispose of them. Looking through the
%Curry-Howard lenses, the elimination principle corresponds to an
%induction principle associated with the data-type. To program over our
%data-types, we rely on their induction principle, guaranteeing the
%well-foundedness of our definition.

\begin{wstructure}
    <- Agda standard library [Nisse file]
        <- x implementations of natural numbers
        <- y implementations of lists
        -> Painful duplication of code and functionality
            <- Types are (slightly) different
                -> Same functions need to be re-implemented 
        -> Crucial need for ``genericity''
\end{wstructure}

The blessing of expressing just the right type for the job can also be
a curse. Where once we might have had a small collection of basic
datatypes and a large library, we now must cope with a cornucopia of
finely confected structures, subtly designed, subtly different. The
basic vector equipment is much like that for lists, but we implement
it separately, often retyping the same code. The Agda standard
library~\cite{nisse:asl}, for example, sports a writhing mass of
list-like structures, including vectors, bounded-length lists,
difference lists, reflexive-transitive closures---the list is
petrifying. Here, we seek equipment to tame this gorgon's head with
\emph{reflection}.

\begin{wstructure}
        /> Coq, Agda: external notion
            <- Not first-class citizen
            -> Cannot "compute" with them
            -> No reflection for data-types (?)
\end{wstructure}

The business of belonging to a datatype is itself a notion
relative to the type's \emph{declaration}. Most typed functional
languages, including those with dependent types, feature a datatype
declaration construct, external to and extending the language for
defining values and programs. However, dependent type systems also
allow us to reflect types as the image of a function from a set of
`codes'---a \emph{universe construction}~\cite{martin-lof:itt}. 
Computing with codes, we expose operations on and
relationships between the types they reflect. Here, we adopt
the universe as our guiding design principle. We abolish the
datatype declaration construct, by reflecting it as a datatype of
datatype descriptions which, moreover, \emph{describes itself}. This
apparently self-supporting construction is a trick, of course, but
we shall show the art of it. We contribute


%As in the simply-typed world, the definition of
%data-types is processed by a meta-theoretical engine, before being
%reifed by extending the type theory with the corresponding type
%formers and elimination principle. Because of this external apparatus,
%data-type definition is not \emph{first-class}: we cannot compute with
%them, such as making new data-types from previous data-types. 
%
%This is a rather harsh limitation, in particular in a
%dependently-typed system. Indeed, reflection~\cite{allen:reflection,
%  gregoire:ring-solver} is at the heart of many dependently-typed
%programming techniques. Not having first-class data-type definitions,
%we have to give up reflection for data-types.

\begin{wstructure}
    <- Dependent types offer new programming techniques
        <- Eg.: universe construction
        /> State of the art haunted by the simply-typed paradigm
            -> Generative
            -> Non reflective
\end{wstructure}

%However, we do not think that we are condemned to such fate. The
%external presentation of data-types is an heritage of the simply-typed
%paradigm. Dependently-typed systems have more to offer. Indeed, new
%programming techniques, unavailable in a simply-typed setting,
%arises. One of them is \emph{universe
%  construction}~\cite{martin-lof:itt}. We shall see how this technique
%help us overcoming the limitations of the standard, non reflective and
%generative presentation of data-types.

\begin{wstructure}
<- State contributions
    <- Closed presentation of data-types 
        -> No generativity requires
        -> Subsuming standard inductive families 
            /> Some popular extensions excluded for now
    <- Descriptions of data-types are first-class 
        <- Self-encoded [Section sec:desc-levitate]
    <- ``generic programming is just programming''
        <- Ability to inspect data-type definition
            -> Write program over them
        <- A generic program works over a class of data-types (???)
            -> Capture this class by common structure
            -> Write a program over this common code
    <- Design a language for generic programming
        -> First serious attempt
            /> except possibly Lisp
                <- ???
\end{wstructure}

%In this paper, we propose a new approach to building data-types in a
%dependent-type theory. Our contributions are the following:

\begin{itemize}
\item a \emph{closed} type theory, extensible only \emph{definitionally},
  nonetheless equipped with a universe of inductive families of datatypes;
\item a \emph{self-encoding} of the universe codes as a datatype in the
  universe---datatype generic programming is just programming;
\item a bidirectional \emph{type propagation} mechanism to conceal
  artefacts of the encoding, restoring
  a convenient presentation of data;
\item examples of generic operations and constructions over our universe,
  notably the \emph{free monad} construction;
\item datatype generic programming delivered \emph{directly},
  not via some isomorphic
  model or `view' of declared types.
\end{itemize}

We study two universes as a means to explore this novel way to equip a
programming language with its datatypes. We warm up with a universe of
\emph{simple} datatypes, just sufficient to describe itself. Once we
have learned this art, we scale up to \emph{indexed} datatypes,
encompassing the inductive families~\cite{dybjer:families,luo:utt}
found in Coq and Epigram, and delivering experiments in generic
programming with applications to the datatype of codes itself.

We aim to deliver proof of concept, showing that a closed theory with
a self-encoding universe of datatypes can be made practicable, but we
are sure there are bigger and better universes waiting for a similar
treatment. Benke, Dybjer and
Jansson~\cite{benke:universe-generic-prog} provide a useful survey of
the possibilities, including extension to inductive-recursive
definition, whose closed-form presentation~\cite{dybjer:axiom-ir,
  dybjer:ir-initial-algebra} is both an inspiration for the present
enterprise, and a direction for future study.

The work of Morris, Altenkirch and
Ghani~\cite{morris:PhD,morris:spf,alti:lics09} on
(indexed) containers has informed our style of encoding and the
equipment we choose to develop, but the details here reflect pragmatic
concerns about intensional properties which demand care in
practice. We have thus been able to implement our work as the basis
for datatypes in the Epigram 2 prototype~\cite{pigs:epigram}. We
have also developed a \emph{stratified} model of our coding scheme
in Agda and Coq\footnote{This model is available at \\
\url{http://personal.cis.strath.ac.uk/~dagand/levitate.tar.gz}}.




%\item We present a basic type-theory and extend it with a universe of
%  finite sets (Section~\ref{sec:type-theory}). We show how coding can
%  be made practical by putting types at work ;
%\item We give a closed presentation of inductive data-types, through a
%  universe of descriptions (Section~\ref{sec:universe-desc}). This
%  first universe has the expressive power of simple inductive
%  types. Being closed, this presentation does not require
%  generativity, hence the type theory remains unchanged when
%  data-types are introduced ;
%\item We present a self-encoding of the universe of description inside
%  itself (Section~\ref{sec:desc-levitate}). As a consequence,
%  description of data-types appears as first-class object in the type
%  theory. We illustrate the benefit of a first-order presentation by
%  implementing a generic catamorphism as well as a generic free monad
%  construction, together with its monadic operations ;
%\item We index the universe of descriptions, to subsume standard
%  inductive families (Section~\ref{sec:indexing-desc}). In this
%  setting, we develop several examples of dependently-typed
%  data-structure and some generic operations over them ;
%\item We have implemented this technology in the Epigram programming
%  language. This is, we believe, the first attempt to design a
%  language for generic programming, Lisp having opened the way. We
%  propose and demonstrate with several examples that generic
%  programming is just programming. Because data-types are described by
%  code, we can finally program with them. As a consequence, generic
%  programs are implemented as functions built from the data-type
%  definition.
%\end{itemize}

%%%%%%%%%%%%%%%%%%%%%%%%%%%%%%%%%%%%%%%%%%%%%%%%%%%%%%%%%%%%%%%%
%% The Type Theory
%%%%%%%%%%%%%%%%%%%%%%%%%%%%%%%%%%%%%%%%%%%%%%%%%%%%%%%%%%%%%%%%


\section{The Type Theory}
\label{sec:type-theory}

One challenge in writing this paper is to extricate our account of datatypes from what else is new in Epigram 2. In fact, we demand relatively little from the setup, so we shall start with a `vanilla' theory and add just what we need. The reader accustomed to dependent types will recognise the basis of her favourite system; for those less familiar, we try to keep the presentation self-contained.

\subsection{Base theory}

\begin{wstructure}
<- Presentation of the formalism
    <- Standard presentation
        -> No novelty here
    <- 3 judgments [equation]
        -> Context validity
        -> Typing judgements
        -> Equality judgements
\end{wstructure}

We adopt a traditional presentation for our type
theory, with three mutually defined systems of judgments:
\emph{context validity}, \emph{typing}, and \emph{equality},
with the following forms:
%
\[
\begin{array}{ll}
\G \vdash \Valid                & \mbox{\(\G\) is a valid context, giving types to variables} \\
\G \vdash \Bhab{\M{t}}{\M{T}}           & \mbox{term \(\M{t}\) has type \(\M{T}\) in context \(\G\)} \\
\G \vdash \Bhab{\M{s} \equiv \M{t}}{\M{T}}  & \mbox{\(\M{s}\) and \(\M{t}\) are equal at type \(\M{T}\) in context \(\G\)} \\
\end{array}
\]

\begin{wstructure}
    <- Invariants [equation]
        -> By induction on derivations
\end{wstructure}

The rules are formulated to ensure that the
following `sanity checks' hold by induction on derivations
%
\[
\begin{array}{l@{\;\Rightarrow\;\;}l}
\G            \vdash \Bhab{\M{t}}{\M{T}}            
    & \G \vdash \Valid\; \wedge\; \G\vdash\Type{\M{T}} \\
\G            \vdash s \equiv \Bhab{\M{t}}{\M{T}}   
    & \G \vdash \Bhab{s}{T} \;\wedge\; \G\vdash \Bhab{\M{t}}{\M{T}}
\end{array} \]
%
and that judgments \(\M{J}\) are preserved by well-typed instantiation.
%
\[
\G ; \xS ; \Delta \vdash \M{J}            \;\Rightarrow\;      
    \G \vdash \Bhab{\M{s}}{\M{S}} \;\Rightarrow\; 
          \G ; \Delta[\M{s}/\x] \vdash \M{J}[\M{s}/\x] 
\]

%We are not going to prove the validity of these invariants. They
%follow rather straightforwardly from the induction rules. For formal
%proofs, we refer the reader to standard presentations of type theory,
%such as Luo's seminal work \cite{luo:utt}.

\begin{wstructure}
    <- Judgemental equality
        <- Presentation independent of particular implementation choice
        -> Model in Agda, intensional
        -> Used in Epigram, OTT
\end{wstructure}

We specify equality as a judgment, leaving open the details of its implementation, requiring only a congruence including ordinary computation (\(\beta\)-rules), decided, e.g., by testing \(\alpha\)-equivalence of \(\beta\)-normal forms~\cite{DBLP:journals/jfp/Adams06}. Coquand and Abel feature prominently in a literature of richer equalities, involving \(\eta\)-expansion, proof-irrelevance and other attractions~\cite{DBLP:journals/scp/Coquand96,DBLP:conf/tlca/AbelCP09}. Agda and Epigram 2 support such features, Coq currently does not, but they are surplus to requirements here.

% Therefore, we are not tied to a particular implementation
%choice. In particular, our system has been modelled in Agda, which
%features an intensional equality. On the other hand, it is used in
%Epigram, whose equality has a slightly extensional
%flavor~\cite{altenkirch:ott}. We expect users of fully extensional
%systems to also find their way through this presentation.

\begin{wstructure}
<- Context validity [figure no longer]
    <- Not much to be said
\end{wstructure}

Context validity ensures that variables inhabit well-formed
sets.
%
\[
%% Empty context validity
\Axiom{\vdash \Valid}
\qquad
%% Extend context
\Rule{\G       \vdash \Type{\M{S}}}
     {\G ; \xS \vdash \Valid}\;\x\not\in\G
\]
%
\begin{wstructure}
<- Typing judgements [figure]
    <- Set in Set
        -> For simplicity of presentation
        -> Assume that a valid stratification can be inferred
            <- Harper-Pollack, Luo, Courant
        -> See later discussion
    <- Standard presentation of Pi and Sigma types
\end{wstructure}
%
The basic typing rules
% (Fig.~\ref{fig:typing-judgements})
for tuples and functions are also standard, save that we locally adopt
\(\Set:\Set\) for presentational purposes. Usual techniques to resolve
this \emph{typical ambiguity} apply~\cite{harper:implicit-universe,
  luo:utt, courant:explicit-universe}. A formal treatment of
stratification for our system is a matter of
ongoing work.
%%  putting presentation before paradox~\cite{girard:set-in-set}.
%% The usual remedies apply, \emph{stratifying}
%% \(\Set\)~\cite{harper:implicit-universe, luo:utt,
%%   courant:explicit-universe}.
%%
\[\stkc{
%% Context
\Rule{\Gamma ; \xS ; \Delta \vdash \Valid}
     {\Gamma ; \xS ; \Delta \vdash \Bhab{\V{x}}{\V{S}}}
\qquad
%% Conversion
\Rule{\Gamma \vdash \Bhab{\V{s}}{\V{S}} \quad 
      \Gamma \vdash \Type{\V{S} \equiv \V{T}}}
     {\Gamma \vdash \Bhab{\V{s}}{\V{T}}}
\\
%% Girard's favorite
\Rule{\Gamma \vdash \Valid}
     {\Gamma \vdash \Type{\Set}}
\qquad
%% Pi-Sigma
\Rule{\Gamma       \vdash \Type{\V{S}} \quad
      \Gamma ; \xS \vdash \Type{\V{T}}}
     {\Gamma \vdash \Type{\PIS{\xS} \V{T}, \SIGMAS{\xS} \V{T}}}
\\
%% %% Prop
%% \Rule{\Gamma \vdash \Bhab{q}{\Prop}}
%%      {\Gamma \vdash \Type{\prf{q}}}
%% \qquad
%% %% True
%% \Rule{\Gamma \vdash \Valid}
%%      {\Gamma \vdash \Bhab{\True}{\Prop}}
%% \\
%% Unit
\Rule{\Gamma \vdash \Valid}
     {\Gamma \vdash \Bhab{\Unit}{\Set}}
\qquad
%% Void
\Rule{\Gamma \vdash \Valid}
     {\Gamma \vdash \Bhab{\Void}{\Unit}}
\\
%% Lambda
\Rule{\stkl{\Gamma       \vdash \Type{\V{S}} \\
            \Gamma ; \xS \vdash \Bhab{\V{t}}{\V{T}}}}
     {\Gamma \vdash \Bhab{\PLAM{\x}{\V{S}} \V{t}}{\PIS{\xS} \V{T}}}
\qquad
%% Application
\Rule{\stkl{\Gamma \vdash \Bhab{\V{f}}{\PIS{\xS} \V{T}} \\
            \Gamma \vdash \Bhab{\V{s}}{\V{S}}}}
     {\Gamma \vdash \Bhab{\V{f}\: \V{s}}{\V{T}[\V{s}/\V{x}]}} 
\\
%% Pair
\Rule{\Gamma       \vdash \Bhab{\V{s}}{S} \quad 
      \Gamma ; \xS \vdash \Bhab{\V{T}}{\Set}    \quad
      \Gamma       \vdash \Bhab{\V{t}}{\V{T}[\V{s}/\V{x}]}}
     {\Gamma \vdash \Bhab{\pair{\V{s}}{\V{t}}{\V{x}.\V{T}}}{\SIGMAS{\xS} \V{T}}}
\\
%% First projection
\Rule{\Gamma \vdash \Bhab{\V{p}}{\SIGMAS{\xS} \V{T}}}
     {\Gamma \vdash \Bhab{\fst{\V{p}}}{\V{S}}} 
\qquad
%% Second projection
\Rule{\Gamma \vdash \Bhab{\V{p}}{\SIGMAS{\xS} \V{T}}}
     {\Gamma \vdash \Bhab{\snd{\V{p}}}{\V{T}[\fst{\V{p}}/\V{x}]}}
\\
}\]


\paragraph{Notation.} We subscript information needed for type synthesis
but not type checking, e.g., the domain of a \(\LAMBINDER\)-abstraction,
and suppress it informally where clear. Square brackets denote tuples,
with a LISP-like right-nesting convention: \(\sqr{a\;b}\) abbreviates
\(\pair{a}{\pair{b}{\void}{}}{}\).

%We recognise the standard presentation of $\Pi$ and $\Sigma$
%types, respectively inhabited by lambda terms and dependent
%pairs. Naturally, there are rules for function application and
%projections of $\Sigma$-types. Equal types can be substituted, thanks
%to the conversion rule.

%For the sake of presentation, we postulate a $\Set$ in $\Set$
%rule. Having this rule makes our type theory inconsistent, by Girard's
%paradox~\cite{girard:set-in-set}. However, it has been
%shown~\cite{harper:implicit-universe, luo:utt,
%  courant:explicit-universe} that a valid stratification can be
%inferred, automatically or semi-automatically. In the remaining of our
%presentation, we will assume that such a stratification exists, even
%though we will keep it implicit. We shall discuss this assumption in
%Section~\ref{sec:discussion}.

%\begin{figure}
%
%\[\stkc{
%% Context
\Rule{\Gamma ; \xS ; \Delta \vdash \Valid}
     {\Gamma ; \xS ; \Delta \vdash \Bhab{\V{x}}{\V{S}}}
\qquad
%% Conversion
\Rule{\Gamma \vdash \Bhab{\V{s}}{\V{S}} \quad 
      \Gamma \vdash \Type{\V{S} \equiv \V{T}}}
     {\Gamma \vdash \Bhab{\V{s}}{\V{T}}}
\\
%% Girard's favorite
\Rule{\Gamma \vdash \Valid}
     {\Gamma \vdash \Type{\Set}}
\qquad
%% Pi-Sigma
\Rule{\Gamma       \vdash \Type{\V{S}} \quad
      \Gamma ; \xS \vdash \Type{\V{T}}}
     {\Gamma \vdash \Type{\PIS{\xS} \V{T}, \SIGMAS{\xS} \V{T}}}
\\
%% %% Prop
%% \Rule{\Gamma \vdash \Bhab{q}{\Prop}}
%%      {\Gamma \vdash \Type{\prf{q}}}
%% \qquad
%% %% True
%% \Rule{\Gamma \vdash \Valid}
%%      {\Gamma \vdash \Bhab{\True}{\Prop}}
%% \\
%% Unit
\Rule{\Gamma \vdash \Valid}
     {\Gamma \vdash \Bhab{\Unit}{\Set}}
\qquad
%% Void
\Rule{\Gamma \vdash \Valid}
     {\Gamma \vdash \Bhab{\Void}{\Unit}}
\\
%% Lambda
\Rule{\stkl{\Gamma       \vdash \Type{\V{S}} \\
            \Gamma ; \xS \vdash \Bhab{\V{t}}{\V{T}}}}
     {\Gamma \vdash \Bhab{\PLAM{\x}{\V{S}} \V{t}}{\PIS{\xS} \V{T}}}
\qquad
%% Application
\Rule{\stkl{\Gamma \vdash \Bhab{\V{f}}{\PIS{\xS} \V{T}} \\
            \Gamma \vdash \Bhab{\V{s}}{\V{S}}}}
     {\Gamma \vdash \Bhab{\V{f}\: \V{s}}{\V{T}[\V{s}/\V{x}]}} 
\\
%% Pair
\Rule{\Gamma       \vdash \Bhab{\V{s}}{S} \quad 
      \Gamma ; \xS \vdash \Bhab{\V{T}}{\Set}    \quad
      \Gamma       \vdash \Bhab{\V{t}}{\V{T}[\V{s}/\V{x}]}}
     {\Gamma \vdash \Bhab{\pair{\V{s}}{\V{t}}{\V{x}.\V{T}}}{\SIGMAS{\xS} \V{T}}}
\\
%% First projection
\Rule{\Gamma \vdash \Bhab{\V{p}}{\SIGMAS{\xS} \V{T}}}
     {\Gamma \vdash \Bhab{\fst{\V{p}}}{\V{S}}} 
\qquad
%% Second projection
\Rule{\Gamma \vdash \Bhab{\V{p}}{\SIGMAS{\xS} \V{T}}}
     {\Gamma \vdash \Bhab{\snd{\V{p}}}{\V{T}[\fst{\V{p}}/\V{x}]}}
\\
}\]

%
%\caption{Typing judgements}
%\label{fig:typing-judgements}
%
%\end{figure}


\begin{wstructure}
<- Judgemental equality [figure]
    <- symmetry, reflexivity, and transitivity
    <- beta-rules for lambda and pair
    <- xi-rule for functions
    -> Agnostic in the notion of equality
        <- Doesn't rely on a ``propositional'' equality
        -> Key: wide applicability of our proposal
\end{wstructure}

The judgmental equality comprises the computational rules below,
closed under reflexivity,
symmetry, transitivity and structural congruence, even under binders.
We omit the mundane rules which ensure these closure properties for
reasons of space.
\[\stkc{
%% %% Reflexivity
%% \Rule{\Gamma \vdash \Bhab{x}{T}}
%%      {\Gamma \vdash \Bhab{x \equiv x}{T}}
%% \qquad
%% %% Symmetry
%% \Rule{\Gamma \vdash \Bhab{x \equiv y}{T}}
%%      {\Gamma \vdash \Bhab{y \equiv x}{T}}
%% \qquad
%% %% Transitivity
%% \Rule{\stkl{\Gamma \vdash \Bhab{z \equiv y}{T} \\
%%             \Gamma \vdash \Bhab{y \equiv z}{T} }}
%%      {\Gamma \vdash \Bhab{x \equiv z}{T}}
%% \\
%% Beta-reduction
\Rule{\stkl{\Gamma       \vdash \Type{\M{S}} \quad
            \Gamma ; \xS \vdash \Bhab{\M{t}}{\M{T}} \\
            \Gamma       \vdash \Bhab{\M{s}}{\M{S}}}}
     {\Gamma \vdash \Bhab{(\PLAM{\x}{\M{S}} \M{t})\:\M{s} \equiv \M{t}[\M{s}/\x]}{\M{T}[\M{s}/\x]}}
\\
%% Xi rule
%% \Rule{\Gamma       \vdash \Type{S} \quad
%%       \Gamma ; \xS \vdash \Bhab{t \equiv t'}{T}}
%%      {\Gamma \vdash \Bhab{(\PLAM{\x}{S} t) \equiv (\PLAM{\x}{S} t')}{\PIS{\xS} T}}
%% \\
%% Projections
\Rule{\stkl{\Gamma                 \vdash \Bhab{\M{s}}{\M{S}} \quad
            \Gamma ; \xS           \vdash \Bhab{\M{T}}{\Set} \\
            \Gamma ; \Bhab{\M{s}}{\M{S}}   \vdash \Bhab{\M{t}}{\M{T}[\M{s}/\x]}}}
     {\Gamma \vdash \Bhab{\fst{(\pair{\M{s}}{\M{t}}{\x.\M{T}})} \equiv \M{s}}{\M{S}}}
\qquad
\Rule{\stkl{\Gamma               \vdash \Bhab{\M{s}}{\M{S}} \quad
            \Gamma ; \xS         \vdash \Bhab{\M{T}}{\Set} \\
            \Gamma ; \Bhab{\M{s}}{\M{S}} \vdash \Bhab{\M{t}}{\M{T}[\M{s}/\x]}}}
     {\Gamma \vdash \Bhab{\snd{(\pair{\M{s}}{\M{t}}{\x.\M{T}})} \equiv \M{t}}{\M{T}[\M{s}/\x]}}
}\]

Given a suitable stratification of \(\Set\), the
computation rules yield a terminating evaluation procedure, ensuring
the decidability of equality and thence type checking.

%Finally, we define the rules governing judgemental equality in
%Figure~\ref{fig:judgemental-equality}. We implicitly assume that
%judgemental equality respects symmetry, reflexivity, and
%transitivity. We capture the computational behavior of the language
%through the $\beta$-rules for function application and pairs. Finally,
%we implicitly assume that it respects purely syntactic and structural
%equality. This includes equality under lambda ($\xi$-rule).

%Crucially, being judgemental, this presentation is agnostic in the
%notion of equality actually implemented. Indeed, our typing and
%equality judgements do not rely on a ``propositional'' equality. This
%freedom is a key point in favour of the wide applicability of our
%proposal. This judgemental presentation must be read as a
%\emph{specification}: our proposal works with any propositional
%equality satisfying this specification. Moreover, our lightweight
%requirements do not endanger decidability of equality-checking.
%Obviously, when implementing our technology in an existing
%type-theory, some opportunities arise. We will present some of them
%along the course of the paper.


%\begin{figure}

%\[\stkc{
%% %% Reflexivity
%% \Rule{\Gamma \vdash \Bhab{x}{T}}
%%      {\Gamma \vdash \Bhab{x \equiv x}{T}}
%% \qquad
%% %% Symmetry
%% \Rule{\Gamma \vdash \Bhab{x \equiv y}{T}}
%%      {\Gamma \vdash \Bhab{y \equiv x}{T}}
%% \qquad
%% %% Transitivity
%% \Rule{\stkl{\Gamma \vdash \Bhab{z \equiv y}{T} \\
%%             \Gamma \vdash \Bhab{y \equiv z}{T} }}
%%      {\Gamma \vdash \Bhab{x \equiv z}{T}}
%% \\
%% Beta-reduction
\Rule{\stkl{\Gamma       \vdash \Type{\M{S}} \quad
            \Gamma ; \xS \vdash \Bhab{\M{t}}{\M{T}} \\
            \Gamma       \vdash \Bhab{\M{s}}{\M{S}}}}
     {\Gamma \vdash \Bhab{(\PLAM{\x}{\M{S}} \M{t})\:\M{s} \equiv \M{t}[\M{s}/\x]}{\M{T}[\M{s}/\x]}}
\\
%% Xi rule
%% \Rule{\Gamma       \vdash \Type{S} \quad
%%       \Gamma ; \xS \vdash \Bhab{t \equiv t'}{T}}
%%      {\Gamma \vdash \Bhab{(\PLAM{\x}{S} t) \equiv (\PLAM{\x}{S} t')}{\PIS{\xS} T}}
%% \\
%% Projections
\Rule{\stkl{\Gamma                 \vdash \Bhab{\M{s}}{\M{S}} \quad
            \Gamma ; \xS           \vdash \Bhab{\M{T}}{\Set} \\
            \Gamma ; \Bhab{\M{s}}{\M{S}}   \vdash \Bhab{\M{t}}{\M{T}[\M{s}/\x]}}}
     {\Gamma \vdash \Bhab{\fst{(\pair{\M{s}}{\M{t}}{\x.\M{T}})} \equiv \M{s}}{\M{S}}}
\qquad
\Rule{\stkl{\Gamma               \vdash \Bhab{\M{s}}{\M{S}} \quad
            \Gamma ; \xS         \vdash \Bhab{\M{T}}{\Set} \\
            \Gamma ; \Bhab{\M{s}}{\M{S}} \vdash \Bhab{\M{t}}{\M{T}[\M{s}/\x]}}}
     {\Gamma \vdash \Bhab{\snd{(\pair{\M{s}}{\M{t}}{\x.\M{T}})} \equiv \M{t}}{\M{T}[\M{s}/\x]}}
}\]

%
%\caption{Judgemental equality}
%\label{fig:judgemental-equality}
%
%\end{figure}



\begin{wstructure}
!!! Need Help !!!
<- Meta-theoretical properties
    <- Assuming a stratified discipline
    <> The point here is to reassert that dependent types are not evil, 
       there is no non-terminating type-checker, or such horrible lies <>
    -> Strongly normalising
        -> Every program terminates
    -> Type-checking terminates
    ???
\end{wstructure}


%This completes our presentation of the type theory. Assuming a
%stratified discipline of universe, the system we have described enjoy
%some very strong meta-theoretical properties. Unlike simply typed
%languages, such as Haskell, dependently-typed systems are
%\emph{strongly normalising}: every program that type-checks
%terminates. Moreover, type-checking is decidable and can therefore be
%implemented by a terminating algorithm.
%\note{Need some care here. Expansion would be good too. I wanted to
%  carry the intuition that we are not the bad guys with a
%  non-terminating type-checker.}

\subsection{Finite enumerations of tags}
\label{sec:finite-sets}

\begin{wstructure}
<- Motivation
    <- Finite sets could be encoded with Unit and Bool
        /> Hinder the ability to name things
    <- W-types considered harmful?
        ???
    -> For convenience
        <- Named elements
        <- Referring by name instead of code
        -> Types as coding presentation
            /> Also as coding representation!
\end{wstructure}

It is time for our first example of a \emph{universe}. You might
want to offer a choice of named constructors in your datatypes: we shall
equip you with sets of tags to choose from. Our plan
is to implement (by extending the theory, or by encoding) the signature
%
\[
  \Type{\EnumU}\qquad \Type{\EnumT{(\Bhab{\M{E}}{\EnumU})}}
\]
%
where some value \(E:\EnumU\) in the `enumeration universe' describes
a type of tag choices \(\EnumT{E}\). We shall need
some tags---valid identifiers, marked to indicate that
they are data, not variables scoped and substitutable---so we hardwire these
rules:
%
\[
%% UId
\Rule{\Gamma \vdash \Valid}
     {\Gamma \vdash \Type{\UId}}
\qquad
%% Tag
\Rule{\Gamma \vdash \Valid}
     {\Gamma \vdash \Bhab{\Tag{\V{s}}}{\UId}}\;\V{s}\: \mbox{a valid identifier}
\]
%
Let us describe enumerations as lists of tags, with signature:
%
\[
\Bhab{\NilE}{\EnumU}\qquad
\Bhab{\ConsE{(\Bhab{\M{t}}{\UId})}{(\Bhab{\M{E}}{\EnumU})}}{\EnumU}
\]
%
What are the \emph{values} in \(\EnumT{E}\)? Formally, we represent
the choice of a tag as a numerical index into \(E\), via new rules:
%
\[
%% Ze
\Rule{\Gamma \vdash \Valid}
     {\Gamma \vdash \Bhab{\Ze}{\EnumT{(\ConsE{\M{t}}{\M{E}})}}} 
\qquad
%% Su
\Rule{\Gamma \vdash \Bhab{\M{n}}{\EnumT{\M{E}}}}
     {\Gamma \vdash \Bhab{\Su{\M{n}}}{\EnumT{(\ConsE{\M{t}}{\M{E}})}}}
\]
%
However, we expect that in practice, you might rather refer to these
values \emph{by tag}, and we shall ensure that this is possible in due course.

%As a motivating example, we are now going to extend the type theory
%with a notion of finite set. One could argue that there is no need for
%such an extension: finite sets, just as any data-structure, can be
%encoded inside the type theory. A well-known example of such encoding
%is the Church encoding of natural numbers, which is isomorphic to
%finite sets. \note{Shall we talk about W-types encoding?}

%However, using encodings is impractical. In the case of finite sets,
%for instance, we would like to name the elements of the sets. Then, we
%need to be able to manipulate these elements by their name, instead of
%their encoding. While we are able to give names to encodings, it is
%extremely tedious to map the encodings back to a name. Whereas these
%objects have a structure, the structure is lost during the encoding,
%when they become anonymous inhabitants of a $\Pi$ or $\Sigma$-type.

%In the simply-typed world, we are used to see types as a coding
%presentation -- presentation of invariants, presentation of
%properties. In the dependently-typed world, we also learn to use types
%as a coding representation: finite sets being good citizens, they
%ought to be democratically represented at the type level. As we will
%see, this gives us the ability to name and manipulate them (this is
%were the democracy analogy goes crazy, I think).
%\note{Did I got the coding presentation vs. coding representation
%  story right? No.}

\begin{wstructure}
<- Implementation [figure]      
    <- Tags
        -> Purely informational token
    <- EnumU
        -> Universe of finite sets
    <- EnumT e
        -> Elements of finite set e
\end{wstructure}

%The specification of finite sets is presented in
%Figure~\ref{fig:typing-finite-set}. It is composed of three
%components. First, we define tags as inhabitants of the $\UId$ type. A
%tag is solely an informative token, used for diagnostic
%purposes. Finite sets inhabits the $\EnumU$ type. Unfolding the
%definition, we get that a finite set is a list of tags. Finally,
%elements of a finite set $\V{u}$ belong to the corresponding $\EnumT{\V{u}}$
%type. Intuitively, it corresponds to an index -- a number -- pointing
%to an element of $\V{u}$.

%\begin{figure}

%\[\stkc{
%% UId
\Rule{\Gamma \vdash \Valid}
     {\Gamma \vdash \Type{\UId}}
\qquad
%% Tag
\Rule{\Gamma \vdash \Valid}
     {\Gamma \vdash \Bhab{\Tag{s}}{\UId}}\;s \mbox{ unique identifier}
\\
%% EnumU
\Rule{\Gamma \vdash \Valid}
     {\Gamma \vdash \Type{\EnumU}} 
\qquad
%% EnumT
\Rule{\Gamma \vdash \Bhab{e}{\EnumU}}
     {\Gamma \vdash \Type{\EnumT{e}}} 
\\
%% NilE
\Rule{\Gamma \vdash \Valid}
     {\Gamma \vdash \Bhab{\NilE}{\EnumU}} 
\qquad
%% ConsE
\Rule{\Gamma \vdash \Bhab{t}{\UId} \quad
      \Gamma \vdash \Bhab{e}{\EnumU}}
     {\Gamma \vdash \Bhab{\ConsE{t}{e}}{\EnumU}}
\\
%% Ze
\Rule{\Gamma \vdash \Valid}
     {\Gamma \vdash \Bhab{\Ze}{\EnumT{\ConsE{t}{e}}}} 
\qquad
%% Su
\Rule{\Gamma \vdash \Bhab{n}{\EnumT{e}}}
     {\Gamma \vdash \Bhab{\Su{n}}{\EnumT{\ConsE{t}{e}}}}
}\]


%\caption{Typing rules for finite sets}
%\label{fig:typing-finite-set}

%\end{figure}


\begin{wstructure}
<- Equipment
    <- \spi operator
        <- Equivalent of Pi on finite sets
        <- First argument: (finite) domain
        <- Second argument: for each element of the domain, a co-domain
        -> Inhabitant of \spi: right-nested tuple of solutions
            <- Skip code for space reasons
    <- switch operator
        <- case analyses over x
        <- index into the \spi tuple to retrieve the corresponding result
\end{wstructure}

Enumerations come with further machinery. Each \(\EnumT{E}\) needs an
eliminator, allowing us to branch according to a tag choice. Formally,
whenever we need such new computational facilities, we add primitive
operators to the type theory and extend the judgmental equality with
their computational behavior. However, for compactness and readability, we
shall write these operators as functional programs (much as we
model them in Agda).

We first define the `small product' $\SYMBspi$ operator:
%
\[\stk{
%% spi
\spi{}{}: \PITEL{\V{E}}{\EnumU}\PITEL{\V{P}}{\EnumT{\V{E}} \To \Set} \To \Set \\
\begin{array}{@{}l@{\:}l@{\;\;\mapsto\;\;}l}
\spi{\NilE}{& \V{P}}        & \Unit \\
\spi{(\ConsE{\V{t}}{\V{E}})}{& \V{P}} & \TIMES{\V{P}\: \Ze}{\spi{\V{E}}{\LAM{\V{x}} \V{P}\: (\Su{\V{x}})}}
\end{array}
}\]
%
This builds a right-nested
tuple type, packing a $\V{P}\:\V{i}$ value for each $\V{i}$ in the given
domain. The step case exposes our notational convention that
binders scope rightwards as far as possible. These tuples are `jump tables', tabulating dependently
typed functions.
We give this functional
interpretation---the eliminator we need---by the
$\SYMBswitch$ operator, which, unsurprisingly, iterates projection:
%
\[\stk{
%% switch
\begin{array}{@{}ll}
\SYMBswitch : \PITEL{\V{E}}{\EnumU}
               \PITEL{\V{P}}{\EnumT{\V{E}} \To \Set} \To
              \spi{\V{E}}{\V{P}} \To
               \PITEL{\V{x}}{\EnumT{\V{E}}} \To \V{P}\: \x
\end{array} \\
\begin{array}{@{}l@{\:\mapsto\:\:}l}
\switch{(\ConsE{\V{t}}{\V{E}})}{\V{P}}{\V{b}}{\Ze}      & \fst{\V{b}} \\
\switch{(\ConsE{\V{t}}{\V{E}})}{\V{P}}{\V{b}}{(\Su{\V{x}})} & \switch{\V{E}}{(\LAM{\V{x}} \V{P}
  (\Su{\V{x}}))}{(\snd{\V{b}})}{\V{x}}
\end{array}
}\]

%Again, there is a clear equivalent in the full-$\Set$ world: function
%application. The operational behaviour of $\F{switch}$ is
%straightforward: $\V{x}$ is peeled off as we move deeper inside the nested
%tuple $\V{b}$. When $\V{x}$ equals $\Ze$, we simply return the value we are
%pointing to.

\begin{wstructure}
<- Equivalent to having a function space over finite sets
    /> Made non-obvious by low-level encodings
        <- General issue with codes
             -> Need to provide an attractive presentation to the user
    -> Types seem to obfuscate our reading
        <- Provide ``too much'' information
        /> False impression: information is actually waiting to be used more widely
        -> See next Section
\end{wstructure}

The $\SYMBspi$ and $\SYMBswitch$ operators deliver dependent
elimination for finite enumerations, but are rather awkward to
use directly. We do not write the range for a \(\LAMBINDER\)-abstraction, so
it is galling to supply \(\V{P}\) for functions defined by $\SYMBswitch$.
Let us therefore find a way to recover the tedious details of the
encoding from types.

%, they also
%come with a notion of finite function space. However, we had to
%extract that intuition from the type, by a careful reading. This seems
%to contradict our argument in favour of types for coding
%representation. Here, we are overflown by low-level, very precise type
%information.

%However, our situation is significantly different from the one we
%faced with encoded data: while we were suffering from a crucial lack
%of information, we are now facing too much information, hence losing
%focus. This is a general issue with the usage of codes, as they convey
%much more information than what the developer is willing to see. 

%As we will see in the following section, there exists a cure to this
%problem. In a nutshell, instead of being overflown by typing
%information, we will put it at work, automatically. The consequence is
%that, in such system, working with codes is \emph{practical}: one
%should not be worried by information overload, but how to use it as
%much as possible. Therefore, we should not be afraid of using codes for
%practical purposes.


\subsection{Type propagation}
\label{sec:type-propagation}

\begin{wstructure}
<- Bidirectional type-checking [ref. Turner,Pierce]
    -> Separating type-checking from type synthesis
    <- Type checking: push terms into types
        <- Example: |Pi S T :>: \ x . t| allows us to drop annotation on lambda
    <- Type synthesis: pull types out of terms
        <- Example: |x : S l- x :<: S| gives us the type of x
\end{wstructure}

Our approach to tidying the coding cruft is deeply rooted in the
bidirectional presentation of type checking from Pierce and
Turner~\cite{pierce:bidirectional-tc}. They divide
type inference into two communicating components. In
\emph{type synthesis}, types are \emph{pulled} out of terms. A
typical example is a variable in the context:
%
\[
\Rule{\G ; \xS ; \Delta \vdash \Valid}
     {\G ; \xS ; \Delta \vdash \Bhab{\V{x}}{\M{S}}}
\]
%
Because the context stores the type of the variable, we can extract the
type whenever the variable is used.

On the other hand, in the \emph{type checking} phase, types are
\emph{pushed} into terms. We are handed a type together with a term,
our task consists of checking that the type admits the term. In doing
so, we can and should use the information
provided by the type. Therefore, we can relax our requirements on the
term. Consider \(\LAMBINDER\)-abstraction:
%
\[
\Rule{\G       \vdash \Type{\M{S}} \quad
      \G ; \xS \vdash \Bhab{\M{t}}{\M{T}}}
     {\G \vdash \Bhab{\PLAM{\x}{\M{S}} \M{t}}{\PIS{\xS} \M{T}}}
\]
%
The official rules require an annotation specifying the domain.
However, in type \emph{checking}, the \(\Pi\)-type we push in determines
the domain, so we can drop the annotation.

\begin{wstructure}
<- Formalisation: type propagation
    <- Motivation
        -> High-level syntax
            -> exprIn: types are pushed in
                <- Subject to type *checking*
            -> exprEx: types are pulled from
                <- Subject to type *synthesis*
        -> Translated into our low-level type theory
        -> Presented as judgements
    -> Presentation mirrors typing rule of [figure] 
        -> Ignore identical judgements
\end{wstructure}

We adapt this idea, yielding a \emph{type
propagation} system, whose purpose is to elaborate compact
\emph{expressions} into the terms of our underlying type theory, much
as in the definition of Epigram
1~\cite{mcbride.mckinna:view-from-the-left}.  We divide expressions
into two syntactic categories: $\exprIn$ into which types are pushed,
and $\exprEx$ from which types are extracted. In the
bidirectional spirit, the $\exprIn$ are subject to type
\emph{checking}, while the $\exprEx$---variables and elimination
forms---admit type \emph{synthesis}. We embed $\exprEx$ into
$\exprIn$, demanding that the synthesised type coincides with the type
proposed. The other direction---only necessary to apply
abstractions or project from pairs---takes a type annotation.

% As the presentation largely
%mirrors the inference rules of the type theory, we will ignore the
%judgments that are identical. We refer our reader to the associated
%technical report~\cite{chapman:desc-tech-report} for the complete
%system of rules.

\begin{wstructure}
<- Type synthesis [figure]
    <- Pull a type out of an exprEx
    <- Result in a full term, together with its type
    -> Do *not* need to specify types
        -> Extracting a term from the context
        -> Function application
        -> Projections
\end{wstructure}

Type synthesis (Fig.~\ref{fig:type-synthesis}) is the \emph{source} of
types. It follows the \(\exprEx\) syntax, delivering both the
elaborated term and its type. Terms and expressions never mix: e.g.,
for application, we instantiate the range with the \emph{term}
delivered by checking the argument \emph{expression}. Hardwired
operators are checked as variables.

\begin{figure}
\[\stkc{
%% Form
\boxed{\Gamma \Vdash \propag{\exprEx}{\pull{\CN{term}}{\CN{type}}}}
\\
\\
%% Reversal
\Rule{\Gamma \Vdash \propag{\push{T}{\Set}}
                           {T'} \quad
      \Gamma \Vdash \propag{\push{t}{T'}}
                           {t'}}
     {\Gamma \Vdash \propag{(\Bhab{t}{T})}
                           {\pull{t'}{T'}}} 
\\
%% Context
\Rule{\Gamma ; \xS ; \Delta \vdash \Valid}
      {\Gamma ; \xS ; \Delta \Vdash \propag{\x}
                                           {\pull{\x}{\M{S}}}}
\qquad
%% Application
\Rule{\stkl{\Gamma \Vdash \propag{\M{f}}
                                 {\pull{\M{f}\M{'}}{\PIS{\xS}{\M{T}}}} \\
            \Gamma \Vdash \propag{\push{\M{s}}{\M{S}}}
                                 {\M{s'}}}}
     {\Gamma \Vdash \propag{\M{f}\: \M{s}}{\pull{\M{f'}\: \M{s'}}{\M{T} [\M{s'}/\x]}}} 
\\
%% First projection
\Rule{\Gamma \Vdash \propag{\M{p}}
                           {\pull{\M{p'}}{\SIGMAS{\xS}{\M{T}}}}}
     {\Gamma \Vdash \propag{\fst{\M{p}}}
                           {\pull{\fst{\M{p'}}}{\M{S}}}} \qquad 
%% Second projection
\Rule{\Gamma \Vdash \propag{\M{p}}
                           {\pull{\M{p'}}{\SIGMAS{\xS}{\M{T}}}}}
     {\Gamma \Vdash \propag{\snd{\M{p}}}
                           {\pull{\snd{\M{p'}}}{\M{T} [\fst{\M{p'}}/\x]}}}
}\]

\caption{Type synthesis}
\label{fig:type-synthesis}
\end{figure}


\begin{wstructure}
<- Type checking [figure]
    <- Push a type in an exprIn
    <- Result in a full term
    -> *Use* the type to build the term!
        -> Domain and co-domain propagation for Pi and Sigma
        -> Translation of 'tags into EnumTs
        -> Translation of ['tags ...] into EnumUs
        -> Finite function space into switch
\end{wstructure}

Dually, type checking judgments (Fig.~\ref{fig:type-checking})
are \emph{sinks} for types. From an
$\exprIn$ and a type pushed into it, they elaborate a low-level
term, extracting information from the type. Note that we inductively ensure the following `sanity checks':
%
\[\stkc{
\Gamma\Vdash\propag{e}{\pull{t}{T}} \Rightarrow
  \Gamma\vdash t:T \\
\Gamma\Vdash\push{\propag{e}{t}}{T} \Rightarrow
  \Gamma\vdash t:T
}\]

Canonical set-formers are \emph{checked}: we could exploit
\(\Set:\Set\) to give them synthesis rules, but this would prejudice
our future stratification plans. Note that abstraction and pairing are
free of annotation, as promised. Most of the propagation rules are
unremarkably structural: we have omitted some mundane rules which just
follow the pattern, e.g., for \(\UId\).

\begin{figure}
\[\stkc{
%% Form
\boxed{\Gamma \Vdash \propag{\push{\CN{exprIn}}{\CN{type}}}{\CN{term}}} 
\\
\\
%% Set in Set
%% \Axiom{\Gamma \Vdash \propag{\push{\Set}{\Set}}
%%                             {\Set}}
%% \\
%% Prop
%% \Rule{\Gamma \Vdash \propag{\push{q}{\Prop}}
%%                            {q'}}
%%      {\Gamma \Vdash \propag{\push{\prf{q}}{\Set}}
%%                            {\prf{q'}}}
%% \qquad
%% True
%% \Axiom{\Gamma \Vdash \propag{\push{\True}{\Prop}}
%%                             {\True}}
%% \\
%% Pi
%% \Rule{\Gamma \Vdash \propag{\push{S}{\Set}}
%%                            {S'} \quad
%%       \Gamma \Vdash \propag{\push{T}{S' \To \Set}}
%%                            {T'}}
%%      {\Gamma \Vdash \propag{\push{\PiTy{S}{T}}{\Set}}
%%                            {\PiTy{S'}{T'}}} 
%% \\
%% %% Sigma
%% \Rule{\Gamma \Vdash \propag{\push{S}{\Set}}
%%                            {S'} \quad
%%       \Gamma \Vdash \propag{\push{T}{S' \To \Set}}
%%                            {T'}}
%%      {\Gamma \Vdash \propag{\push{\SigmaTy{S}{T}}{\Set}}
%%                            {\SigmaTy{S'}{T'}}}
%% \\
%% Lambda
\Rule{\Gamma ; \xS \Vdash \propag{\push{\V{t}}{\V{T}}}
                                 {\V{t'}}}
     {\Gamma \Vdash \propag{\push{\LAM{\x} \V{t}}{\PIS{\xS}{\V{T}}}}
                           {\PLAM{\x}{\V{S}} \V{t'}}} 
\\
%% Pair
\Rule{\stkl{ \Gamma \Vdash \propag{\push{\V{s}}{\V{S}}}
                                  {\V{s'}} \\
             \Gamma \Vdash \propag{\push{\V{t}}{\V{T}[\V{s'}/\V{x}]}}
                                  {\V{t'}}}}
     {\Gamma \Vdash \propag{\push{\pair{\V{s}}{\V{t}}{}}{\SIGMAS{\xS}{\V{T}}}}
                           {\pair{\V{s'}}{\V{t'}}{\V{T}}}}
\\
%% EnumU
%% \Axiom{\Gamma \Vdash \propag{\push{\EnumU}{\Set}}
%%                             {\EnumU}} 
%% \qquad
%% %% EnumT
%% \Rule{\Gamma \Vdash \propag{\push{e}{\EnumU}}
%%                            {e'}}
%%      {\Gamma \Vdash \propag{\push{\EnumT{e}}{\Set}}
%%                            {\EnumT{e'}}}
%% \\
%% Tag
\Axiom{\Gamma \Vdash \propag{\push{\etag{t_0}}{\EnumT{(\ConsE{\etag{t_1}}{\V{e}})}}}
                            {\Ze}}\;\etag{t_0} = \etag{t_1}
\\
\Rule{\Gamma \Vdash \propag{\push{\etag{t_0}}{\EnumT{\V{e}}}}
                           {\V{n}}}
     {\Gamma \Vdash \propag{\push{\etag{t_0}}{\EnumT{(\ConsE{\etag{t_1}}{\V{e}})}}}
                            {\Su{\V{n}}}}\;\etag{t_0} \neq \etag{t_1}
\\
%% EnumU
\Axiom{\Gamma \Vdash \propag{\push{[]}{\EnumU}}
                            {\NilE}}
\\
\Rule{\Gamma \Vdash \propag{\push{\V{ts}}{\EnumU}}
                                 {\V{cs}}}
     {\Gamma \Vdash \propag{\push{[ \etag{t_1},\: \V{ts} ]}{\EnumU}}
                                 {\ConsE{\etag{t_1}}{\V{cs}}}}
\\
%% Switch
\Rule{\Gamma \Vdash \propag{\push{\V{t}}{\spi{\V{e}}{\V{P}}}}
                           {\V{t'}}}
     {\Gamma \Vdash \begin{array}{@{}l} 
                        \propag{\push{\V{t}}{\PI{\V{x}}{\EnumT{\V{e}}} \V{P}\:\V{x}}}
                               {\\ \PLAM{\V{x}}{(\EnumT{\V{e}})} \switch{\V{e}}{\V{P}}{\V{t'}}{\V{x}}}
                    \end{array}}\;\mbox{$\V{t}$ is $[]$ or $[\_,\_]$}
%% \\
%% Conversion
%% \Rule{\Gamma \Vdash \propag{s}
%%                            {\pull{s'}{S}} \quad 
%%       \Gamma \Vdash \push{S \equiv T}{\Set}}
%%      {\Gamma \Vdash \propag{\push{s}{T}}
%%                            {s'}}
}\]

\caption{Type checking}
\label{fig:type-checking}
\end{figure}

However, we also add abbreviations. We write \(\spl{\M{f}}\),
pronounced `uncurry \(\M{f}\)' for the function which takes a pair and
feeds it to \(\M{f}\) one component at a time, letting us name
them individually. Now, for the finite enumerations, we go to work.

Firstly, we present the codes for enumerations as right-nested tuples
which, by our LISP convention, we write as unpunctuated lists of tags
\(\sqr{\etag{t_0}\ldots\etag{t_n}}\).
Secondly, we can denote an element \emph{by its
name}: the type pushed in allows us to recover the numerical
index. We retain the numerical forms to facilitate
\emph{generic} operations and ensure that shadowing is punished
fittingly, not fatally.
Finally, we express functions from enumerations as tuples.
Any tuple-form, \(\void\) or \(\pair{\_}{\_}{}\), is accepted by the
function space---the generalised product---if it is accepted by the
small product. Propagation fills in the appeal to $\SYMBswitch$,
copying the range information.

Our interactive development tools also perform the
reverse transformation for intelligible output.
The encoding of any specific enumeration is thus hidden by these
translations. Only, and rightly, in enumeration-generic programs is the
encoding exposed.

\begin{wstructure}
<- Summary
    -> Not a novel technique [refs?]
        /> Used as a boilerplate scrapper
    -> Make dealing with codes *practical*
        <- Example: Finite sets/finite function space
        -> We should not restrain our self in using codes
            <- We know how to present them to the user
-> Will extend this machinery in further sections
\end{wstructure}

%In this section, we have developed a type propagation system based on
%bidirectional type-checking. Using bidirectional type-checking as a
%boilerplate scrapper is a well-known
%technique~\cite{pierce:bidirectional-tc,
%  xi:bidirectional-tc-bound-array, chlipala:strict-bidirectional-tc}
%\note{Everybody ok with the citations?}. In our case, we have shown
%how to instrument bidirectionality to rationalise the expressivity of
%dependent types. We have illustrated our approach with finite sets. We
%have abstracted away the low-level presentation of finite sets,
%offering a convenient syntax instead.

%This example teaches us that we should not be afraid of codes, as soon
%as type information is available. We have shown how to rationalise
%this information in a formal presentation. Hence, we have shown that
%programming with such objects is practical. As we introduce more codes
%in our theory, we will show how to extend the framework we have
%developed so far.

Our type propagation mechanism does no constraint
solving, just copying, so it is just the thin end of the elaboration wedge.
It can afford us this `assembly
language' level of civilisation as \(\EnumU\) universe
specifies not only the \emph{representation} of the
low-level values in each set as bounded numbers, but also the
\emph{presentation} of these values as high-level tags. To encode only
the former, we should merely need the \emph{size} of enumerations, but
we extract more work from these types by making them more informative.
We have also, \emph{en passant}, distinguished enumerations which have
the same cardinality but describe distinct notions:
\(\EnumT{\sqr{\etag{\CN{red}}\,\etag{\CN{blue}}}}\) is not
\(\EnumT{\sqr{\etag{\CN{green}}\,\etag{\CN{orange}}}}\).


%%%%%%%%%%%%%%%%%%%%%%%%%%%%%%%%%%%%%%%%%%%%%%%%%%%%%%%%%%%%%%%%
%% A Universe of simple data-types
%%%%%%%%%%%%%%%%%%%%%%%%%%%%%%%%%%%%%%%%%%%%%%%%%%%%%%%%%%%%%%%%

\section{A Universe of Inductive Datatypes}
\label{sec:universe-desc}

\begin{wstructure}
<- Why starting with simple datatypes
    <- For pedagogical purposes
        <- datatypes as we know them every day
        /> Target dependent types
    -> Cut down version of Induction Recursion
        -> Presentation evolves independently as we go to dependent types
\end{wstructure}

In this section, we describe an implementation of inductive types, as
we know them from ML-like languages. By working with familiar datatypes,
we hope to focus on the delivery mechanism, warming up gently to the
indexed datatypes we really want.  % Our encoding is inspired by
Dybjer and Setzer's closed formulation of
induction-recursion~\cite{dybjer:axiom-ir}, but without the
`-recursion'.  An impredicative Church-style encoding of datatypes is
not adequate for dependently typed programming, as although such
encodings present data as non-dependent eliminators, they do not
support dependent \emph{induction}~\cite{geuvers:induction-not-derivable}.
Whilst the \(\LAMBINDER\)-calculus captures all that data can \emph{do},
it cannot ultimately delimit all that data can \emph{be}.

\subsection{The power of $\Sigma$}

\begin{wstructure}
<- The duality of Sigma
    <- Sigma generalises sum over arbitrary arities
        -> \Sigma A B == \Sigma_{x : A} B x
    <- Sigma generalises product to have a dependant second component
        -> \Sigma A B == (x : A) \times (B x)
\end{wstructure}

In dependently typed languages, $\Sigma$-types can be interpreted as
two different generalisations. This duality is reflected in the
notation we can find in the literature. The notation
$\blue{\Sigma}_{\x : \M{A}} (\M{B}\: \x)$ stresses that
$\Sigma$-types are `dependent sums', generalising sums over
arbitrary arities, where simply typed languages have finite sums.

On the other hand, our choice, $\SIGMA{\x}{\M{A}}
(\M{B}\:\x)$, emphasises that $\Sigma$-types generalise products,
with the type of the second component depending on the value of the
first. Simply typed languages do not express such relative validity.

\begin{wstructure}
<- datatypes in the simply-typed world
    -> "sums-of-product"
        <- Sum of constructors
        <- Product of arguments
<- datatypes in the dependently-typed world
    -> "sigmas-of-sigmas"
    /> Need ability to manipulate these sigmas
        -> Define a Code for datatypes
        -> Together with a sigma-based Interpretation
\end{wstructure}

In ML-like languages, datatypes are presented as a
\emph{sum-of-products}. A datatype is defined by a finite sum of
constructors, each carrying a product of arguments. To embrace
these datatypes, we have to capture this grammar.
With dependent types, the notion of sum-of-products translates into
\emph{sigmas-of-sigmas}.


\subsection{The universe of descriptions}
\label{sec:desc-universe}

\begin{wstructure}
<- Introduction to Universe construction
    <- Define a Code
        -> Name objects
    <- Define an Interpretation of codes into the type theory
        -> Give a semantics to objects
    -> Ability to manipulate code
    -> Ability to compute with these objects
\end{wstructure}

While sigmas-of-sigmas can give a \emph{semantics} for the
sum-of-products structure in each node of the tree-like values in a
datatype, we need to account for the recursive structure which
ties these nodes together. We do this by
constructing a \emph{universe}~\cite{martin-lof:itt}. Universes
are ubiquitous in dependently typed
programming~\cite{benke:universe-generic-prog, oury:power-of-pi},
but here we take them as the foundation of our notion
of datatypes.

%The key idea behind universe construction is our ability to make names
%by defining new types. These names are called \emph{codes}. By
%defining a set of codes, we define the syntax of a language. To give
%it a semantics, we \emph{interpret} these codes back into the type
%theory. Hence, codes act as labels, while the type theory provides the
%computational machinery. Codes being mere labels, we can inspect them,
%hence regaining structural information. We can also compute over them:
%deriving new codes from others, or even functions on them.

\begin{wstructure}
<- Justification of the code 
    <- [both figures]: cannot be read separately
    <- Mimic the standard grammar of datatypes description
        -> Just as we already know it
        <- '\Sigma for making sigmas-of-sigmas
        <- 'indx for exhibiting the functoriality
            -> For recursive arguments
        <- '1 for end of description
\end{wstructure}

To add inductive types to our type theory, we build a universe of
datatype \emph{descriptions} by implementing the signature presented
in Figure~\ref{fig:desc_universe}, with codes mimicking the grammar of
datatype declarations. We can read a description $\V{D}:\Desc^{n}$ as
a `pattern functor' on $\Set^{n}$, with $\SYMBdescop{\V{D}}$ its
action on an object, \(\V{X}\), soon to be instantiated
recursively. The superscripts indicate the $\Set$-levels at which we
\emph{expect} these objects in a stratified system. This is but an
informal notation, to give a flavour of the stratified
presentation. Note that the functors so described are \emph{strictly
  positive}, by construction.

Descriptions are sequential structures ending in $\DUnit$,
indicating the empty tuple. To build sigmas-of-sigmas, we provide a
$\SYMBDSigma$ code, interpreted as a $\Sigma$-type. To request a
recursive component, we have $\DIndx{\V{D}}$, where \(\V{D}\)
describes the rest of the node. These codes give us
sigmas-of-sigmas with recursive places. An equivalent, more algebraic presentation
could be given, as illustrated in Section~\ref{sec:indexing-desc}.

\begin{figure}
\[\stk{
\begin{array}{@{}ll}
\Desc^n 
    &: \Set^{n+1}\\
\DUnit 
    &: \Desc^{n}\\
\DSigma{(\Bhab{\V{S}}{\Set^{n}})}{(\Bhab{\V{D}}{\V{S} \To \Desc^{n}})} 
    &: \Desc^{n}\\
\DIndx{(\Bhab{\V{D}}{\Desc^{n}})} 
    &: \Desc^{n} \\
\end{array}
\smallskip\\
\begin{array}{l@{\V{X}\:\mapsto\:}l}     
\multicolumn{2}{l}{\descop{\_\:}{} : \Desc^{n} \To \Set^{n} \To \Set^{n}} \\
 \descop{\DUnit}{} &  \Unit \\
 \descop{\DSigma{\V{S}}{\V{D}}}{} &
     \SIGMAS{\Bhab{\V{s}}{\V{S}}}{\descop{\V{D}\,\V{s}}{\!\V{X}}}  \\
\descop{\DIndx{\V{D}}}{}  &  \TIMES{\V{X}}{\descop{\V{D}}{\V{X}}}
\end{array}
}\]

\caption{Universe of Descriptions}
\label{fig:desc_universe}
 
\end{figure}

We admit to being a little coy,
writing of `implementing a signature' without clarifying how. A viable
approach would simply be to extend the theory with constants for the
constructors and an operator for \(\descop{\V{D}}\). In
Section~\ref{sec:desc-levitate}, you will see what we do instead.
Meanwhile, let us gain some intuition by developing examples.

\subsection{Examples}
\label{sec:desc-examples}

\begin{wstructure}
<- Nat
    <- Sum of zero, suc
    <- zero case: done
    <- suc case: leave open and done
    -> NatD Z = 1 + Z
\end{wstructure}

We begin with the natural numbers, now working in the high-level
expression language of Section~\ref{sec:type-propagation}, exploiting
type propagation.
%
\[\stk{
\NatD : \Desc^n \\
\NatD \mapsto \DSigma{\EnumT{\sqr{\NatZero\: \SYMBNatSuc }}}
                     {\sqr{ \DUnit \quad (\DIndx{\DUnit}) }}
}\]
%
Let us explain its construction. First, we use $\SYMBDSigma$ to
give a choice between the $\NatZero$ and $\SYMBNatSuc$ constructors.
What follows depends on this choice, so we write the function
computing the rest of the description in tuple notation.  In the
$\NatZero$ case, we reach the end of the description. In the
$\SYMBNatSuc$ case, we attach one recursive argument and close the
description. Translating the \(\Sigma\) to a binary sum, we have
effectively described the functor:
%
\[    \NatD\: \V{Z} \mapsto \Unit \mathop{\D{+}} \V{Z}    \]
Correspondingly, we can see the injections to the sum:
\[
\sqr{\NatZero}:\descop{\NatD}Z \qquad
\sqr{\NatSuc{(\Bhab{z}{Z})}}:\descop{\NatD}Z
\]

\begin{wstructure}
<- List
    <- Sum of nil, cons
    <- nil case: done
    <- cons case: product of X with leave open and done
    -> ListD X Z = 1 + X * Z
\end{wstructure}

The pattern functor
for lists needs but a small change:
%
\[\stk{
\ListD : \Set^n \To \Desc^n \\
\ListD \: \V{X} \mapsto
 \DSigma{\EnumT{\sqr{ \ListNil\: \SYMBListCons }}}
         {\sqr{ \DUnit \quad (\DSigma{\V{X}}{\LAM{\_} \DIndx{\DUnit}}) }}
}\]
%
The $\SYMBNatSuc$ constructor becomes
$\SYMBListCons$, taking an $\V{X}$ followed by a
recursive argument. This code describes the following functor:
%
\[    \ListD\: \V{X}\: \V{Z} \mapsto \Unit \mathop{\D{+}} \V{X} \D{\ensuremath{\times}} \V{Z}     \]

\begin{wstructure}
<- Tree
    <- sum of leaf, node
    <- leaf case: done
    <- node case: product of X with two leave open and done
    -> TreeD X Z = 1 + X * Z * Z
\end{wstructure}

Of course, we are not limited to one recursive argument. Here are
the node-labelled binary trees:
%
\[\stk{
\TreeD : \Set^n \To \Desc^n \\
\begin{array}{@{}l@{}l}
\TreeD \: \V{X} \mapsto 
    \DSigma{ & \EnumT{\sqr{ \TreeLeaf\: \TreeNode }} \\}
           { & \sqr{ \DUnit \quad
                     (\DIndx{(\DSigma{\V{X}}{\LAM{\_}\DIndx{\DUnit}})})} }
\end{array}
}\]
%
Again, we are one evolutionary step away from $\ListD$. However,
instead of a single call to the induction code, we add another. The
interpretation of this code corresponds to the following functor:
%
\[    \TreeD\: \V{X}\: \V{Z} \mapsto \Unit \mathop{\D{+}} 
          \V{Z} \Prod \V{X}  \Prod \V{Z}     \]


\begin{wstructure}
<- Tagged description
    <- Form TDesc = List (UId x Desc) [equation]
    <- Follow usual sums-of-product presentation of datatype
        <- Finite set of constructors
        <- Then whatever you want
    -> Any Desc datatype can be turned into this form
        -> No loss of expressive power
        /> Guarantee a ``constructor form''
\end{wstructure}

From the examples above, we observe that datatypes are defined by a
$\SYMBDSigma$ whose first argument enumerates the constructors. We
call codes fitting this pattern \emph{tagged} descriptions. Again,
this is a clear reminder of the sum-of-products style. Any
description can be forced into this style with a singleton constructor
set. We characterise tagged descriptions thus:
\[\stk{
 \TagDesc^n : \Set^{n+1} \\
 \TagDesc^n \mapsto \SIGMA{\V{E}}{\EnumU} (\spi{\V{E}}{\LAM{\_} \Desc^n})
\smallskip\\
\SYMBtoDesc : \TagDesc^n \To \Desc^n \\
\SYMBtoDesc \mapsto
\spl{\LAM{\V{E}}\LAM{\V{D}}
\DSigma{\EnumT{\V{E}}}{(\F{switch}\:\V{E}\:(\LAM{\_}\Desc^n)\:\V{D})}}
}\]
It is not such a stretch to expect that the familiar datatype
declaration might desugar to the \emph{definitions} of a tagged description.

\begin{wstructure}
<- Fictive object [figure 'data Desc']
    -> Must be read as a type signature
    -> See further for its actual implementation
        <- Subject to our levitation exercise
\end{wstructure}

\subsection{The least fixpoint}
\label{sec:desc-fix-point}

\begin{wstructure}
<- Build the fixpoint of functors
    <- See examples: need to build their initial algebra
    -> Extend the type theory with Mu/Con [figure]
        <- Straightforward definition of a fixpoint
            <- Interpret D with (Mu D) as sub-objects
\end{wstructure}

\newcommand{\mude}[1]{\D{\(\mu^{\!+}\)}\:#1}

So far, we have built pattern functors with our \(\Desc\) universe.
Being polynomial functors, they all admit a least fixpoint, which we
now construct by \emph{tying the knot}: the element type abstracted
by the functor is now instantiated recursively:
%
\[
\Rule{\Gamma \vdash \Bhab{\M{D}}{\Desc^n}}
     {\Gamma \vdash \Bhab{\Mu{\M{D}}}{\Set^n}} \qquad
\Rule{\Gamma \vdash \Bhab{\M{D}}{\Desc^n} \quad 
      \Gamma \vdash \Bhab{\M{d}}{\descop{\M{D}}{(\Mu{\M{D}})}}}
     {\Gamma \vdash \Bhab{\Con{\M{d}}}{\Mu{\M{D}}}}
\]
Tagged descriptions are very common, so we abbreviate:
\[
  \mude{}:\TagDesc^n \To \Set^n \qquad \mude{\V{T}}\mapsto\Mu{(\toDesc{\V{T}})}
\]

\begin{wstructure}
<- Elimination on Mu
    <- We are used to foldD : \forall X. (desc D X -> X) -> mu D -> X
        /> Not dependent
        -> Cannot express some (which one again?) properties
    -> Develop a dependent induction
        <- Everywhere/All
        <- Induction
    -> *Generic*
    ???
\end{wstructure}

We can now build datatypes and their elements, e.g.:
\[\stk{
\Nat \mapsto \Bhab{\mude{\pair{\sqr{\NatZero\: \SYMBNatSuc }}
                               {\sqr{\DUnit \quad (\DIndx{\DUnit})}}{}}}
                  {\Set^n}
\\
\Con{\sqr{\NatZero}}:\Nat \qquad
\Con{\sqr{\NatSuc{(\Bhab{n}{\Nat})}}}:\Nat
}\]

But how shall we compute with our data?  We should expect an
elimination principle. Following a categorical intuition, we might
provide the `fold', or `iterator', or `catamorphism':
%
\[
\F{cata} : \PITEL{\V{D}}{\Desc^n}
           \PI{\V{T}}{\Set^n}
           (\descop{\V{D}}{\V{T}} \To \V{T}) \To 
           \Mu{\V{D}} \To \V{T} 
\]

\newcommand{\SYMBind}{\F{ind}}

However, iteration is inadequate for \emph{dependent} computation.
We need \emph{induction} to write functions whose type depends on inductive
data. Following \citet{benke:universe-generic-prog}, we adopt the following:
%
\[\stk{
\F{ind} : \stk{ \PITEL{\V{D}}{\Desc^n}
                \PI{\V{P}}{\Mu{\V{D}} \To \Set^k}         \\
               (\PI{\V{d}}{\descop{\V{D}}{(\Mu{\V{D}})}}       
                \All{\V{D}}{(\Mu{\V{D}})}{\V{P}}{\V{d}} \To \V{P} (\Con{\V{d}})) \To \\
               \PI{\V{x}}{\Mu{\V{D}}} \V{P} \V{x} 
} \\
\F{ind}\, \V{D}\, \V{P}\, \V{m}\, (\Con{\V{d}}) \mapsto 
    \V{m}\, \V{d}\, (\all{\V{D}}{(\Mu{\V{D}})}{\V{P}}
                           {(\F{ind}\, \V{D}\, \V{P}\, \V{m})}
                           {\V{d}})
}\]
%
Here, $\All{\V{D}}{\V{X}}{\V{P}}{\V{d}}$ states that
$\Bhab{\V{P}}{\V{X} \To \Set^k}$ holds for every subobject $\V{x}:\V{X}$
in \(\V{D}\), and \(\all{\V{D}}{\V{X}}{\V{P}}{\V{p}}{\V{d}}\) is a
`dependent map', applying some
\(\Bhab{\V{p}}{\PIS{\Bhab{\x}{\V{X}}}\V{P}\,\V{x}}\) to each \(\x\)
contained in \(\V{d}\). The definition (including an extra case,
introduced soon) is in
Figure~\ref{fig:all-predicates}.\footnote{To pass the termination
  checker, we had to inline the definition of $\SYMBall$ into
  $\SYMBind$ in our Agda model. A simulation argument shows that the
  definition presented here terminates if the inlined version
  does. Hence, although not directly structural, this definition is
  indeed terminating.} So, $\SYMBind$ is our first
operation generic over descriptions, albeit hardwired.  Any
datatype we define comes with induction.

Note that the very same functors \(\SYMBdescop{\M{D}}\) also admit greatest
fixpoints: we have indeed implemented coinductive types this way, but
that is another story.

\begin{figure*}

\[
\begin{array}{ll}
%%
\stk{
\begin{array}{@{}ll}
\SYMBAll : & \PITEL{\V{D}}{\Desc^n}
             \PITEL{\V{X}}{\Set^n}
             \PITEL{\V{P}}{\V{X} \To \Set^k} \\
           & \PI{\V{xs}}{\descop{\V{D}}{\V{X}}} 
             \Set^k 
\end{array} \\
\begin{array}{@{}l@{}l@{\:\mapsto\:\:}l}
\All{\DUnit}{& \V{X}}{\V{P}}{\Void} &
    \Unit \\
\All{(\DSigma{\V{S}}{\V{D}})}{& \V{X}}{\V{P}}{\pair{\V{s}}{\V{d}}{}} &
    \All{(\V{D}\: \V{s})}{\V{X}}{\V{P}}{\V{d}} \\
\All{(\DIndx{\V{D}})}{& \V{X}}{\V{P}}{\pair{\V{x}}{\V{d}}{}} &
    \TIMES{\V{P}\: \V{x}}{\All{\V{D}}{\V{X}}{\V{P}}{\V{d}}} \\
\All{(\DHindx{\V{H}}{\V{D}})}{& \V{X}}{\V{P}}{\pair{\V{f}}{\V{d}}{}} &
    \TIMES{(\PI{\V{h}}{\V{H}} \V{P}\: (\V{f}\: \V{h}))}
          {\All{\V{D}}{\V{X}}{\V{P}}{\V{d}}}
\end{array}
}
&
%%
\stk{
\begin{array}{@{}ll}
\SYMBall : & \PITEL{\V{D}}{\Desc^n}
             \PITEL{\V{X}}{\Set^n}
             \PITEL{\V{P}}{\V{X} \To \Set^k} \\
           & \PITEL{\V{p}}{\PI{\V{x}}{\V{X}} \V{P}\: \V{x}}
             \PI{\V{xs}}{\descop{\V{D}}{\V{X}}} 
             \All{\V{D}}{\V{X}}{\V{P}}{\V{xs}} 
\end{array} \\
\begin{array}{@{}l@{}l@{\:\mapsto\:\:}l}
\all{\DUnit}{& \V{X}}{\V{P}}{\V{p}}{\Void} &
    \Void \\
\all{(\DSigma{\V{S}}{\V{D}})}{& \V{X}}{\V{P}}{\V{p}}{\pair{\V{s}}{\V{d}}{}} &
    \all{(\V{D}\: \V{s})}{\V{X}}{\V{P}}{\V{p}}{\V{d}} \\
\all{(\DIndx{\V{D}})}{& \V{X}}{\V{P}}{\V{p}}{\pair{\V{x}}{\V{d}}{}} &
    \pair{\V{p}\: \V{x}}
         {\all{\V{D}}{\V{X}}{\V{P}}{\V{p}}{\V{d}}}{} \\
\all{(\DHindx{\V{H}}{\V{D}})}{& \V{X}}{\V{P}}{\V{p}}{\pair{\V{f}}{\V{d}}{}} &
    \pair{\LAM{\V{h}} \V{p}\: (\V{f}\: \V{h})}
         {\all{\V{D}}{\V{X}}{\V{P}}{\V{p}}{\V{d}}}{}
\end{array}
\end{array}
}
\]

\caption{Defining and collecting inductive hypotheses}
\label{fig:all-predicates}

\end{figure*}


\subsection{Extending type propagation}

\begin{wstructure}
<- Extending type propagation
    <- datatype declaration turns into definitions
        -> Straightforward translation to Desc
        -> Creation of a variable referring to the structure
    <- Labelled Mu
        /> Just mention the possibility of labelling, no details required
        -> For the user, objects have names rather than Mu of codes
    <- Push Mu to an applied name [figure]
        -> Direct integration into the type propagation machinery
    -> Coded presentation is practical
        <- The user never see a code
\end{wstructure}


We now have low level machinery to build and manipulate inductive
types. Let us apply cosmetic surgery
to reduce the syntactic overhead. We extend type checking of
expressions:
%
\[
\Rule{\Gamma\Vdash\propag{\push{\etag{c}}{\EnumT{\M{E}}}}{n} \quad
\Gamma\Vdash  
\propag{
  \push{\sqr{\vec{t}}}
   {\descop{\M{D}\: n}{(\Mu{(\DSigma{\EnumT{\M{E}}}{\M{D}})})}}}
        {\M{t'}}}
     {\Gamma\Vdash\propag{\push{\etag{c}\: \vec{\M{t}}}
                   {\Mu{(\DSigma{\EnumT{\M{E}}}{\M{D}})}}}
             {\Con{\pair{n}{\M{t'}}{}}}}
\]
%
Here $\etag{c}\:\vec{t}$ denotes a tag `applied' to a sequence
of arguments, and \(\sqr{\vec{t}}\) that sequence's repackaging as
a right-nested tuple. Now we can just write data directly.
\[
\NatZero:\Nat\qquad \NatSuc{(\Bhab{n}{\Nat})}:\Nat
\]
Once again, the type explains the legible presentation, as well as
the low-level representation.

We may also simplify appeals to induction by type propagation, as we
have done with functions from pairs and enumerations.
\[
\Rule{\Gamma\Vdash\stk{
\propag{\push{\M{f}}
             {\PI{\V{d}}{\descop{\M{D}}{(\Mu{\M{D}})}}       
                \All{\M{D}}{(\Mu{\M{D}})}{(\PLAM{\x}{\Mu{\M{D}}}\M{P})}{\V{d}} \To \M{P}[\Con{\V{d}}/\x]\\}}
       {\M{f'}}}}
{\Gamma\Vdash
\propag{\push{\sind{\M{f}}}
             {\PIS{\Bhab{\x}{\Mu{\M{D}}}}\M{P} }}
       {\F{ind}\:\M{D}\:(\PLAM{\x}{\Mu{\M{D}}}\M{P})\:\M{f'}}}
\]
This abbreviation is no substitute for the dependent pattern
matching to which we are entitled in a high-level language built on top
of this theory~\cite{goguen:pattern-matching}. It does at least make
`assembly language' programming mercifully brief, albeit hieroglyphic.
\[\stk{
\F{plus}:\Nat\To\Nat\To\Nat \\
\F{plus}\mapsto\sind{\spl{\sqr{(\LAM{\_}\LAM{\_}\LAM{\V{y}}\V{y})
  \quad(\LAM{\_}\spl{\LAM{\V{h}}\LAM{\_}\LAM{\V{y}}
    \NatSuc{(\V{h}\:\V{y})}})}}}
}\]

This concludes our introduction to the universe of datatype descriptions.
We have encoded sum-of-products datatypes from the simply-typed world
as data and equipped them with computation. We have also made sure
to hide the details by type propagation.


%%%%%%%%%%%%%%%%%%%%%%%%%%%%%%%%%%%%%%%%%%%%%%%%%%%%%%%%%%%%%%%%
%% Levitating the universe of descriptions
%%%%%%%%%%%%%%%%%%%%%%%%%%%%%%%%%%%%%%%%%%%%%%%%%%%%%%%%%%%%%%%%

\section{Levitating the Universe of Descriptions}
\label{sec:desc-levitate}

In this section, we will fulfil our promises and show how we implement
the signatures, first for the enumerations, and then for the codes of
the $\Desc^n$ universe.  Persuading these programs to perform was a perilous
pedagogical peregrination for the protagonist.  Our method was indeed to
hardwire constants implementing the signatures specified above, in the
first instance, but then attempt to replace them, step by step, with
\emph{definitions}: ``Is \(2+2\) still \(4\)?'', ``No, it's a loop!''.
But we did find a way, so now we hope to convey to you the
dizzy feeling of levitation, without the falling.



\subsection{Implementing finite enumerations}

\begin{wstructure}
<- Recall typing rules of 1st section
    -> Make clear they were just promises
    -> Can be implemented now
        <- Simply List UId
\end{wstructure}

In Section~\ref{sec:finite-sets}, we specified the
finite sets of tags. We are going to implement (at every universe level)
the $\EnumU$ type former and its constructors. Recall:
%
\[\Type{\EnumU}^n\qquad\Bhab{\NilE}{\EnumU}\qquad
\Bhab{\ConsE{(\Bhab{\M{t}}{\UId})}{(\Bhab{\M{E}}{\EnumU})}}{\EnumU}
\]
%
The $\NilE$ and $\SYMBConsE$ constructors are just the `nil' and
`cons' or ordinary lists, with elements from \(\UId\).
Therefore, we
implement:
%
\[
\EnumU \mapsto \Mu{(\ListD\: \UId)}
\qquad
\NilE \mapsto \ListNil
\qquad
\ConsE{\M{t}}{\M{E}} \mapsto \ListCons{\M{t}}{\M{E}}
\]


\begin{wstructure}
<- Consequences
    -> Type theory doesn't need to be extended with EnumU, NilE, and ConsE
        <- EnumU == Mu EnumUD
        <- NilE, ConsE are just tags
    -> Do not need a specific \spi eliminator
        <- \spi is an instance of the generic eliminator
            <- Code?
    -> Anything else remains the same (switch, EnumT, 0, 1+)
\end{wstructure}

Let us consider the consequences. We find that the type
theory does not need a special type former $\EnumU$,
or special constructors $\NilE$ and $\SYMBConsE$. 
Moreover, the $\spi{\M{E}}{\M{P}}$ operator, computing tuple types
of \(\M{P}\)s by recursion on \(\M{E}\) need not be hardwired: we can
just use the generic $\F{ind}$ operator, as we would for any ordinary
program.

Note, however, that the universe decoder \(\EnumT{\M{E}}\) \emph{is}
hardwired, as are the primitive \(\Ze\) and \(\Su\) that
we use for low-level values, and indeed the \(\SYMBswitch\) operator.
We cannot dispose of data altogether! We have, however, gained
the ordinariness of the enumeration \emph{codes}, and hence of generic
programs which manipulate them. Our next step is similar: we are going to
condense the entire naming scheme of datatypes \emph{into itself}.

%% Apart from these changes, we are left with implementing the other
%% components of finite sets. This consists of $\EnumT$, $\Ze$, $\Su$, as
%% well as the $\F{switch}$ operator. Note that the actual implementation
%% of $\EnumU$ does not influence our implementation of $\EnumT$: be they
%% hard-coded or codes in the $\Desc$ universe, the $\EnumU$ objects
%% behave the same. We have witnessed this effect when carrying the
%% operation in Epigram, moving from an hard-coded presentation to a
%% self-hosted one. Absolutely no change to the $\EnumT$ objects was
%% required.

\begin{wstructure}
<- Summary of the operation
    <- The content of the type theory is exactly the same
        <- before == after
    /> type naming scheme condenses
        <- Replace named constructors by codes in the universe of datatypes
    -> Our next step is a similar move (in essence)
        /> Condenses the entire naming scheme of datatypes
\end{wstructure}

%In this section, we have replaced a low-level presentation of finite
%sets by a self-hosted one, expressed in the universe of
%descriptions. However, formally, the content of the type theory
%remains unchanged: objects that were present before the modification
%are still there. Conversely, we have not introduced any spurious
%object.

%If not on the content, this modification had an effect on the
%names. The type naming scheme of the type theory has condensed: named
%type formers ($\EnumU$) and constructors ($\NilE$ and $\ConsE$) are
%now replaced by codes and their fixpoint in the universe of
%descriptions. In essence,

\subsection{Implementing descriptions}

\begin{wstructure}
<- Realising our promises
    <- We are going to implement Desc
    /> Desc is itself a datatype
        <- Same situation as EnumU
            <- We want to benefit from generic operations
        -> It ought to be encoded in itself
\end{wstructure}

The set of codes, $\Desc$, is already some sort of datatype; as with $\EnumU$,
we ought to be able to describe it, coding of $\Desc^n$ in $\Desc^{n+1}$, spiralling
upwards. Hence, this code would be a first-class citizen, born with
the generic equipment of datatypes.

\subsubsection{First attempt}

\begin{wstructure}
<- A partial implementation
    <- '1 and 'indx are easy
    <- 'sigma is partially doable
        /> lack the ability to do an higher-order inductive call
    -> Show partial code [figure]
\end{wstructure}

Our first attempt gets stuck quite quickly:
%
\[\stk{
\DescD^n : \Desc^{n+1} \\
\DescD^n \mapsto
  \tada{\stkm{\DUnit\\ \SYMBDSigma\\ \SYMBDIndx}}
       {\stkm{ \DUnit                                \\
         \DSigma{\Set^n}{\LAM{\V{S}} \SHED}      \\
         \DIndx{\DUnit}    }                    } \\
%\begin{array}{@{}ll}
%\DescD \mapsto \DSigma{\!}{\!} & (\EnumT{[ \DUnit, \DSigma{\!}{\!}, \DIndx{\!} ]})  \\
%                               & \left[\begin{array}{l}
%                                   \DUnit                                \\
%                                   \DSigma{\Set}{\LAM{\V{S}} \SHED}      \\
%                                   \DIndx{\DUnit}                        \\
%                                 \end{array}
%                                 \right]
%\end{array}
}\]
%
Let us explain where we stand. Much as
we have done so far, we first offer a constructor choice from
$\DUnit$, $\SYMBDSigma$, and $\SYMBDIndx$. You may notice
that the `tagged' notation we have used for the $\Desc^n$ constructors now
fits the facts: these were actually the tags we are defining.
For $\DUnit$, we immediately reach the end of the description. For
$\SYMBDIndx$, there is a single recursive argument. Describing
$\SYMBDSigma$ is problematic. Recall the specification of
$\SYMBDSigma$:
%
\[
\DSigma{(\Bhab{\V{S}}{\Set^n})}{(\Bhab{\V{D}}{\V{S} \To \Desc^n})} : \Desc^n\\
\]
%
So, we first pack a  $\Set^n$, $\V{S}$, as well we might when working in
$\Desc^{n+1}$. We should then like
a recursive argument \emph{indexed} by $\V{S}$, but
that is an \emph{exponential}, and our presentation so far delivers only
sums-of-products. To code our universe, we must first enlarge it!


\subsubsection{Second attempt}

\begin{wstructure}
<- Extending the universe of description
    -> With higher-order induction
    <- Intuition: index elements in X by H, and go on reading
        -> indx is isomorph to hindx for H = 1
    /> Keep indx
        <- First order!
        -> Extensionally equal to hindx 1
        /> Practically, definitional equality on Sigma/Pi cannot cope with it
    -> Show DescD code
\end{wstructure}

In order to capture a notion of higher-order induction, we add a code
$\SYMBDHindx$ that takes an indexing set $\V{H}$. This amounts to give
a recursive subobject for each element of $H$.
%
\[\stk{
    \DHindx{(\Bhab{\M{H}}{\Set^n})}{(\Bhab{\M{D}}{\Desc^n})} : \Desc^n
    \smallskip \\
    \descop{\DHindx{\V{H}}{\V{D}}}{\V{X}}     
        \mapsto 
            \TIMES{(\V{H} \To \V{X})}{\descop{\V{D}}{\V{X}}} \\
}\]


Note that up to isomorphism, $\SYMBDIndx$ is subsumed by
$\DHindx{\Unit}{\!}$. However, the apparent duplication has some
value.  Unlike its counterpart, $\SYMBDIndx$ is first-order: we prefer
not to demand dummy functions from \(\Unit\) in ordinary data,
e.g. \(\NatSuc{(\LAM{\_}n)}\). It is na{\"\i}ve to imagine that up to
isomorphism, any representation of data will do.  First-order
representations are finitary by construction, and thus admit a richer,
componentwise decidable equality than functions may in
general possess.\footnote{E.g., extensionally, there is one function in
\(\EnumT{\void}\To\Nat\); intensionally, there is a countable infinitude
which it is dangerous to identify definitionally.}

We are now able to describe our universe of datatypes:
%
\[\stk{
\DescD^n : \Desc^{n+1} \\
\DescD^n \mapsto\tada
{\stkm{ \DUnit\\ \SYMBDSigma\\ \SYMBDIndx\\ \SYMBDHindx}}
{\stkm{
\DUnit                                            \\
                                   \DSigma{\Set^n}{\LAM{\V{S}} \DHindx{\V{S}}{\DUnit}}   \\
                                   \DIndx{\DUnit}                                    \\
                                   \DSigma{\Set^n}{\LAM{\_} \DIndx{\DUnit}}
}}
}\]
%
The $\DUnit$ and $\SYMBDIndx$ cases remain unchanged, as expected. We
successfully describe the $\SYMBDSigma$ case via
the higher-order induction, branching on $\V{S}$. The $\SYMBDHindx$ case
just packs a $\Set^n$ with a recursive argument.

At a first glance, we have achieved our goal. We have described the
codes of the universe of descriptions. The fixpoint of \(\descop{\DescD^n}\)
is a datatype just like $\Desc^n$, in $\Set^{n+1}$. Might we be
so bold as to take \(\Desc^n \mapsto \Mu{\DescD^n}\) as the levitating
definition? If we do, we shall come down with a bump! 
To complete our levitation, just as in the magic trick, requires
hidden assistance. Let us explain the problem and reveal the `invisible
cable' which fixes it.

\subsubsection{Final move}

\begin{wstructure}
<- Subtlety: translation of [ ... ]
    -> Let us do it manually
        -> Code with problem for the motive of switch
\end{wstructure}

The definition \(\Desc^n \mapsto \Mu{\DescD^n}\) is circular,
but the offensive recursion is concealed by a prestidigitation.

Expanding \(\toDesc{-}\) and propagating types as in
Figure~\ref{fig:type-checking} reveals the
awful truth:
\[
\Desc^n\mapsto
\Mu{\stk{(\DSigma{\EnumT{\sqr{ \DUnit\:\: \SYMBDSigma\:\: \SYMBDIndx\:\: \SYMBDHindx}}\\}
    {\;\SYMBswitch\:\sqr{ \DUnit\:\: \SYMBDSigma\:\: \SYMBDIndx\:\: \SYMBDHindx}\:(\LAM{\_}\Desc^{n+1})\\
\;\sqr{\stkm{
\DUnit                                            \\
                                   \DSigma{\Set^n}{\LAM{\V{S}} \DHindx{\V{S}}{\DUnit}}   \\
                                   \DIndx{\DUnit}                                    \\
                                   \DSigma{\Set^n}{\LAM{\_} \DIndx{\DUnit}}
}}})}}
\]
The recursion shows up only because we must specify the return type
of the general-purpose \(\F{switch}\), and it is computing a \(\Desc^{n+1}\)!
Although type propagation
allows us to hide this detail \emph{when defining a function}, we cannot
readily suppress this information and check types
when \(\F{switch}\) is fully applied.

\begin{wstructure}
<- The magician trick
    <- Our problem is to give a motive for switch
        /> We perfectly know what it ought to be: \_ -> DescD
    -> Solution: extend the type theory with a special purpose switchD
        -> Only extension required to the type theory!
        -> Hidden away to the user by the syntactic sugar
            -> Sufficient to ensure unavailability as a raw operator
            <- Another instance of type propagation
\end{wstructure}

%What happens if we unfold the definition? We ought to build the following term:
%
%\[
%\PLAM{\V{x}}{(\EnumT{\V{E}})} \switch{\V{E}}{(\LAM{\_} %\Mu{\DescD})}{\V{\pi^f}}{\V{x}}
%\]
%
%But this is quite problematic. We are still in the process of
%constructing $\DescD$, and the motive of $\F{switch}$ is abruptly
%begging for this very same $\DescD$. Despite our willingness, we
%cannot materialise such motive. However, we perfectly know what the
%motive is.

We are too close to give up now. If only we did not need to
supply that return type, especially when we know what it must be!
We eliminate the recursion by \emph{specialising} \(\F{switch}\):
%
\[
\F{switchD} : \PITEL{\V{E}}{\EnumU}   \To
                (\spi{\V{E}}{\LAM{\_} \Desc^m}) \To
                \EnumT{\V{E}} \To \Desc^m
\]
%
The magician's art rests here, in this extension. We conceal it
behind a type propagation rule for \(\F{switchD}\) which we apply
with higher priority than for \(\F{switch}\) in general.
%
\[
\Rule{\Gamma \Vdash
  \propag{\push{\sqr{\vec{t}}}{\spi{\M{E}}{\PLAM{x}{\EnumT{E}}\Desc^m}}}
                           {\M{t'}}}
     {\Gamma \Vdash
\propag{\push{\sqr{\vec{t}}}{\EnumT{\M{E}}\To\Desc^m}}
 {\switchD{\M{E}}{\M{t'}}}}
\]
As a consequence, our definition above now propagates without
introducing recursion. Of course, by pasting together the declaration
of \(\Desc^n\) and its internal copy, we have made it appear in its own
type. Hardwired as a trusted \emph{fait accompli}, this creates no regress,
although one must assume the definition to recheck it.

Our Agda model does not formalise the $\SYMBswitchD$
construction. Instead, we exhibit the isomorphism between declared
and encoded descriptions. Here, $\SYMBswitchD$ lets us collapse
this isomorphism, operationally identifying defined and coded
descriptions.

There are other ways to achieve a sufficient specialisation to avoid a
recursive code, e.g., extending $\Desc^n$ with specialised codes for
\emph{finite} sums and products, pushing the $\SYMBswitch$ into the
interpretation of codes, rather than the code itself. Here, we prefer
not to add codes to $\Desc^n$ which are otherwise unmotivated.
%%
%\[\stk{
%    \Dsigma{(\Bhab{\V{E}}{\EnumU})}
%           {(\Bhab{\V{B}}{\spi{\V{E}}{\LAM{\_}{\Desc^n}}})} : \Desc^n
%    \smallskip \\
%    \descop{\Dsigma{\V{E}}{\V{B}}}{\V{X}}     
%        \mapsto 
%            \SIGMAS{\Bhab{\V{x}}{\EnumT{\V{E}}}}
%                   %{\descop{\switch{\V{E}}{(\LAM{\_}{\Desc})}{\V{B}}{\V{x}}}{\V{X}}}    
%}\]
%%
%Using $\SYMBDsigma$ to switch over the $\Desc$ constructors, we avoid
%mention of $\Desc$ while levitating it. The $\SYMBswitch$ operator is
%pushed into the interpretation function, at which point $\Desc$ indeed
%exists.

%% * There exists alternative
%% **   <- All we want is to avoid mention of Desc
%% **   -> Add a code for finite sigma : $\sigma : (e : \EnumU) -> \pi e \Desc -> \Desc$
%% **       <- Take a finite set in argument
%% **       <- Interpreted as a switch
%% **       -> No more need to mention Desc during its construction
%% **           <- Pushed into intrepretation, at which time it exists

\begin{wstructure}
<- Generic programming now!
    <- Desc is just data
        -> Can be manipulated
    <- Free induction scheme on Desc
        -> Ability to inspect datatypes
        -> Ability to program on datatypes
\end{wstructure}


We have levitated \(\Desc\) at every level. Beyond its pedagogical
value, this exercise has several practical outcomes. First, it
confirms that each $\Desc$ universe is just plain data. As any piece of
data, it can therefore be inspected and manipulated. Moreover, it is
expressed in a $\Desc$ universe. As a consequence, it is equipped,
for free, with an induction principle. So, our ability to inspect and
program with $\Desc$ is not restricted to a meta-language: we have
the necessary equipment to program with \emph{data}, so we can program over
datatypes. \emph{Generic programming is just programming}.


\subsection{The generic catamorphism}

\begin{wstructure}
<- Making cata
    <- Present the type signature
    <- Starts with a call to generic induction
        <- induction on Desc!
        /> Show types at hand
        -> Explain how to use inductive hypothesis
    <- Implement the 'replace' function
    -> Dependent-typeless catamorphism 
\end{wstructure}

In Section~\ref{sec:desc-fix-point}, we hardwired a dependent
$\F{ind}$uction principle, but sometimes,
iteration suffices. Let us construct the catamorphism.

\newcommand{\cata}{\F{cata}}

We proceed by induction on the \emph{data} in \(\Mu{\V{D}}\):
the non-dependent return type $\V{T}$ is readily propagated.
Given a node $\V{xs}$
and the induction hypotheses, the method ought to build an element of
$\V{T}$. Provided that we know how to make an element of
$\descop{\V{D}}{\V{T}}$, this step will be performed by the algebra
$\V{f}$. Let us take a look at this jigsaw:
%
\[\stk{
\cata : \PITEL{\V{D}}{\Desc}
           \PI{\V{T}}{\Set}
           (\descop{\V{D}}{\V{T}} \To \V{T}) \To 
           \Mu{\V{D}} \To \V{T} \\
\cata\: \V{D}\: \V{T}\: \V{f} \mapsto
  \sind \LAM{\V{xs}}\LAM{\V{hs}} \V{f}\: \SHED
}\]
% 
The hole remains: we have
\(\Bhab{\V{xs}}{\descop{\V{D}}{\Mu{\V{D}}}}\) and
\(\Bhab{\V{hs}}{\All{\V{D}}{\Mu{\V{D}}}{(\LAM{\_} \V{T})}{\V{xs}}}\)
to hand, and we need a \(\descop{\V{D}}{\V{T}}\).
Now, $\V{xs}$ has the right shape, but its
components have the wrong type. However, for each such
component, $\V{hs}$ holds the corresponding value
in $\V{T}$.  We need a function to \(\F{replace}\) the former with the latter:
this pattern
matching sketch yields an induction on \(\V{D}\). We fill the hole
with \(\F{replace}\: \V{D}\:(\Mu{\V{D}})\: \V{T}\: \V{xs}\: \V{hs}\).
%
\[\stk{
\F{replace} : \stk{\PITEL{\V{D}}{\Desc}
                   \PITEL{\V{X},\V{Y}}{\Set}\\
                   \PI{\V{xs}}{\descop{\V{D}}{\V{X}}} 
                   \All{\V{D}}{\V{X}}{(\LAM{\_}\V{Y})}{\V{xs}} \To
                   \descop{\V{D}}{\V{Y}}} \\
\begin{array}{@{}l@{\:}l@{\:}l@{\:\:\mapsto\:\:}l}
\F{replace}\: \DUnit &          \V{X}\: \V{Y}\: \Void&          \Void   &
    \Void  \\
\F{replace}\: (\DSigma{\V{S}}{\V{D}})& \V{X}\: \V{Y}\: \pair{\V{s}}{\V{d}}{}& \V{d'}  &
    \pair{\V{s}}{\F{replace}\: (\V{D}\: \V{s})\: \V{X}\: \V{Y}\: \V{d}\: \V{d'}}{} \\
\F{replace}\: (\DIndx{\V{D}})&     \V{X}\: \V{Y}\: \pair{\V{x}}{\V{d}}{}& \pair{\V{y}}{\V{d'}}{} &
    \pair{\V{y}}{\F{replace}\: \V{D}\: \V{X}\: \V{Y}\: \V{d}\: \V{d'}}{} \\
\F{replace}\: (\DHindx{\V{H}}{\V{D}})& \V{X}\: \V{Y}\: \pair{\V{f}}{\V{d}}{}& \pair{\V{g}}{\V{d'}}{} &
 \pair{\V{g}}{\F{replace}\: \V{D}\: \V{X}\: \V{Y}\: \V{d}\: \V{d'}}{}
\end{array}
}\]
%

%% The astute reader will have been struck by the type of $\F{replace}$:
%% it is \emph{almost} the morphism part -- sometimes called \emph{map}
%% -- of the functor $\V{D}$ from $\V{X}$ to $\V{Y}$ in $\Set$. Just as
%% the $\F{induction}$ is the dependent version of $\cata$, $\F{replace}$
%% is the dependent version of the map, which uses the induction
%% hypotheses. For space reason, we will not present the non-dependent
%% map. It can be found in the Agda model.

\begin{wstructure}
<- Deriving generic functions
    <- Taking a Desc and computing a function
        <- Desc comes equipped with an induction principle
        -> Ability to compute more functions from it
            -> More generic functions
    <- Inspecting datatypes
        <- All described byu a Desc code
        -> Ability to explore the code
            <- Desc equipped with an induction principle
            -> Build new objects based on that structure
\end{wstructure}

We have shown how to derive a generic
operation, \(\cata\), from a pre-existing generic operation,
\(\F{ind}\), by
manipulating descriptions as data: the catamorphism is just a
function taking each $\Desc$ value to a datatype specific
operation. This is polytypic programming, as in
PolyP~\cite{jansson:polyp}, made ordinary.

%% Moreover, the $\F{replace}$ function demonstrates the benefit of an
%% approach based on universes. The datatypes living in the universe of
%% descriptions, we are able to \emph{inspect} them. As shown by
%% $\F{replace}$, it is easy to explore these structures, as well as
%% building new ones.

\subsection{The generic free monad}
\label{sec:desc-free-monad}

\begin{wstructure}
<- A generic program: the free monad construction
    <- Recall free monad construction in Haskell
        -> Based on a functor F
    <- Note that the free monad construction is itself defined by a functor
        -> Extract it
\end{wstructure}

In this section, we try a more ambitious generic operation. Given a
functor---a signature of operations represented as a tagged
description---we build its free monad, extending the signature with
variables and substitution.
%
\newcommand{\FMFreeMonad}{\D{FreeMonad}}
\newcommand{\FMFreeMonadD}{\D{FreeMonadD}}
\newcommand{\FMVar}{\C{Var}}
\newcommand{\FMComposite}{\C{Composite}}
%
Let us recall this construction in, say, Haskell. Given a
functor \texttt{f}, the free monad over \texttt{f} is given
thus:
%
\[
\texttt{data FreeMonad f x = Var x | Op (f (FreeMonad f x))}
\]
%
Provided \texttt{f} is an instance of \texttt{Functor}, we may
take \texttt{Var} for \texttt{return} and use \texttt{f}'s
\texttt{fmap} to define \texttt{>>=} as substitution.

Being an inductive type, $\FMFreeMonad$ arises
by a pattern functor:
%
\[
\FMFreeMonadD\: \V{F}\: \V{X}\: \V{Z} \mapsto \V{X} \mathop{\D{+}} \V{F}\:\V{Z}
\]

\begin{wstructure}
    <- Encode it in the Desc world [equation]
        <- F is the Desc we start with
        <- The free monad functor is what we have just defined
        <- [\_]* : Desc -> Set -> Desc
           [\_]* D X = 'cons ['var ('sigma X (\_ -> '1))] D
        -> Mu does the fixpoint
\end{wstructure}

Our construction takes the functor as a
tagged description, and given a set $\V{X}$ of variables, computes the
tagged description of the free monad pattern functor.
%
\[\stk{
\FreeMonad{\_} : \TagDesc \To \Set \To \TagDesc \\
\FreeMonad{\pair{\V{E}}{\V{D}}{}}\:\V{X} \mapsto
    \pair{\pair{\DVar{}}{\V{E}}{}}
         {\pair{\DSigma{\V{X}}{\DUnit}}{\V{D}}{}}{}
}\]
%
We simply add a constructor, $\SYMBDVar$, making its arguments
$\DSigma{\V{X}}{\DUnit}$---just an element of $\V{X}$. 
$\V{E}$ and $\V{D}$ stay put, leaving the other constructors
unchanged. Unfolding the interpretation of this definition, we
find an extended sum, corresponding to the \(\V{X}\D{+}\) in
$\FMFreeMonadD$. Taking the fixpoint ties the knot and we have our
data.

\begin{wstructure}
<- A generic program: monadic substitution [equation]
    <- subst : \forall T X Y. mu ([T]* X) -> (X -> mu ([T]* Y)) -> mu ([T]* Y)
        -> Using Fold
\end{wstructure}

\newcommand{\subst}{\F{subst}}
\newcommand{\apply}{\F{apply}}

Now we need the operations. As expected, \(\LAM{\x}\DVar{\x}\)
plays the r\^ole of \return, making variables terms. Meanwhile,
\bind is indeed \emph{substitution}, which we now
implement generically, making use of $\cata$. Let us write the type,
and start filling in the blanks:
%
\[\stk{
\begin{array}{@{}ll}
\subst : & \PITEL{\V{D}}{\TagDesc}
           \PI{\V{X}, \V{Y}}{\Set} 
           (\V{X} \To \mude{(\FreeMonad{\V{D}}{\V{Y}})}) \To \\
         & \mude{(\FreeMonad{\V{D}}{\V{X}})} \To
           \mude{(\FreeMonad{\V{D}}{\V{Y}})} 
\end{array} \\
\subst\: \V{D}\: \V{X}\: \V{Y}\: \V{\sigma} \mapsto
  \cata\: (\toDesc{(\FreeMonad{\V{D}}{\V{X}})})\: 
          (\mude{(\FreeMonad{\V{D}}{\V{Y}})})\: 
          \SHED
}\]
%
We are left with implementing the algebra of the catamorphism. Its
role is to catch appearances of $\DVar{\V{x}}$ and replace them by
$\V{\sigma}\: \V{x}$. This corresponds to the following definition:
%
\[\stk{
\begin{array}{@{}ll}
\apply : & \PITEL{\V{D}}{\TagDesc} 
           \PI{\V{X}, \V{Y}}{\Set} 
           (\V{X} \To \mude{(\FreeMonad{\V{D}}{\V{Y}})}) \To \\
         & \descop{\toDesc{(\FreeMonad{\V{D}}{\V{X}})}}{(\mude{(\FreeMonad{\V{D}}{\V{Y}})})}
           \To  \mude{(\FreeMonad{\V{D}}{\V{Y}})}
\\
\end{array} \\
\begin{array}{@{}l@{\:\mapsto\:\:}l}
\apply\: \V{D}\: \V{X}\: \V{Y}\: \V{\sigma}\: \sqr{\DVar{\V{x}}}   & \V{\sigma}\: \V{x}                   \\
\apply\: \V{D}\: \V{X}\: \V{Y}\: \V{\sigma}\: \pair{\V{c}}{\V{xs}}{} & \Con{\pair{\V{c}}{\V{xs}}{}}
\end{array}
}\]

\begin{wstructure}
    -> Consequences
        <- We have free monad datatype
            <- Term + variables
        <- We have monad operations
            <- Return / var
            <- Substitution / bind
\end{wstructure}

We complete the hole with $\apply\: \V{D}\: \V{X}\: \V{Y}\:
\V{\sigma}$. Every tagged
description can be seen as a signature of operations: we can uniformly
add a notion of variable, building a new type from an old one, then
providing the substitution structure.

\begin{wstructure}
<- Deriving new data-structure and functions on them
    <- Computing the Free Monad of a datatype
        <- Derive new data-structure from previous one
            <- It is just code
        /> New data-structure comes with some equipment
    <- Computing new functions on computed datatypes
        <- If data comes with structure, we ought to be able to capture it
            <- Induction on Desc
            -> Ability to compute over data
\end{wstructure}

%% Candidate for removal:
%% With the free monad construction, we have seen two kinds of generic
%% operations. Firstly, we have derived a new data-structure from another
%% one: we make the free monad from its underlying functor. To do so, we
%% crucially rely on the fact that datatypes are nothing but codes. We
%% are therefore entitled to modify this code and, in this case, extend
%% it. Extending a datatype might give rise to a more structured object,
%% as was the case here.  So, secondly, we have equipped this new
%% datatype with its inherent structure: the \bind\ and
%% \return\ operations. We have been able to build them as generic
%% functions.

\subsection{Skyhooks all the way up?}

%% * Desc in Desc
%% ** /> Blur the lines between implementation and reflection
%% ** /> Abusing the paradoxical nature of Set : Set

In this section, we have seen how to \emph{levitate}
descriptions. Although our theory, as presented here, takes
$\Bhab{\Set}{\Set}$, our annotations indicate how a stratified theory
could code each level from above.  We do not rely on the paradoxical
nature of $\Bhab{\Set}{\Set}$ to flatten the hierarchy of descriptions
and fit large inside small.  We shall now be more precise about what
we have done.

%% * Summary 
%% ** <- What is the equipment for making data-types
%% ** <- What is reflected, what is implemented
%% ** <- (table 1)

Let us first clarify the status of the implementation. The kit for
making datatypes is presented in Table~\ref{tab:sumup-operators}. For
each operation, we describe its role and its status, making clear
which components are self-described and which ones are actually
implemented.

%% * Paradox
%% ** <- How do we bottom out in a stratified setting?
%% ** <- Spiral
%% ***    <- Encoding of DescD_n : Desc_{n+1}
%% ***    <- Desc_n = Mu DescD_n : Set_{n+1}
%% ** <- Agda model: Desc_42
%% ***    <- No dependent pattern-matching
%% ***    <- No IR
%% ***    <- Normal universes
%% ***    -> Straight Agda (UTT)
%% ** <- ``Self-encoding'' only in a level polymorphic sense
%% ***    -> Agda model: set poly

In a stratified system, the `self-encoded' nature of $\Desc$ appears
only in a set polymorphic sense: the principal type of the encoded
description generalises to the type of $\Desc$ itself. We encode this
much in our set polymorphic model in Agda and in our Coq model,
crucially relying on typical ambiguity~\cite{harper:implicit-universe}.
We step outside current technology only to replace the declared $\Desc$
with its encoding.

Even this last step, we can approximate within a standard predicative
hierarchy. Fix a top level, perhaps \(42\). We may start by declaring
$\Desc^{42}:\Set^{43}$. We can then construct $\DescD^{41}:\Desc^{42}$
and thus acquire an encoded $\Desc^{41}$. Although $\Desc^{41}$ is
encoded, not declared, it includes the relevant descriptions,
including $\DescD^{40}$. We can thus build the tower of descriptions
down to $\Desc^0$, encoding every level below the top.  Description of
descriptions forms a `spiral', rather than a circle. We have modelled
this process exactly in Agda, without any appeal to dependent
pattern matching, induction-recursion, or set polymorphism. All it
takes to build such a sawn-off model of encodings is inductive
definition and a cumulative predicative hierarchy of set levels.

\begin{table}

{
\begin{center}
\begin{tabular}{|c|c|c|c|}
\hline
Object                & Role                        & Status \\
\hline
\hline
$\EnumU$              & Build finite sets           & Levitated \\
\hline
$\Desc$               & Describe pattern functors   & Levitated \\
\hline
$\SYMBdescop{\_}$     & Interpret descriptions      & Hardwired \\
\hline
$\SYMBMu$, $\SYMBCon$ & Define, inhabit fixpoints   & Hardwired \\
\hline
$\SYMBind$, $\SYMBAll$, $\SYMBall$  
                      & Induction principle         & Hardwired \\
\hline
\end{tabular}
\end{center}
}

\caption{Summary of constructions on Descriptions}
\label{tab:sumup-operators}

\end{table}


%%%%%%%%%%%%%%%%%%%%%%%%%%%%%%%%%%%%%%%%%%%%%%%%%%%%%%%%%%%%%%%%
%% Indexing descriptions
%%%%%%%%%%%%%%%%%%%%%%%%%%%%%%%%%%%%%%%%%%%%%%%%%%%%%%%%%%%%%%%%

\section{A Universe of Inductive Families}
\label{sec:indexing-desc}

\newcommand{\vtup}[2]{\bigRedBracket{\begin{array}{@{}#1@{}}#2\end{array}}}

\begin{wstructure}
!!! Need Help !!!
<- Motivation
    <- Desc: expressivity of simply-typed datatypes: inductive types
        <- Values do not influence types
    /> Example: Vectors
        <- Cannot be defined by just induction
            <- Vectors of all size need to be defined at the *same* time
            -> Defined as a *family* of types
                -> Index
        -> I -> IDesc I: Inductive family
    ???
\end{wstructure}

So far, we have explored the realm of inductive types, building on
intuition from ML-like datatypes, using type dependency as a
descriptive tool in $\Desc$ and its interpretation. Let us now make
dependent types the object as well as the means of our study.

Dependent datatypes provide a way to work at higher level of
precision \emph{a priori}, reducing the sources of failure we
might otherwise need to manage. For the perennial
example, consider \emph{vectors}---lists indexed by length. By
making length explicit in the type, we can prevent hazardous
operations (the type of `head' demands vectors of length
$\NatSuc{\V{n}}$) and offer stronger guarantees (pointwise
addition of $\V{n}$-vectors yields an $\V{n}$-vector).

However, these datatypes are not \emph{individually} inductive. For
instance, we have to define the whole \emph{family} of vectors
mutually, in one go. In dependently typed languages, the basic grammar
of datatypes is that of inductive families. To capture this grammar,
we must account for \emph{indexing}.

%% \subsection{Desc, atomically}
%% \label{sec:idesc-atomic-desc}

%% \begin{wstructure}
%% [Outdated: type former presentation instead]
%% <- Adding hindx have introduced some duplication
%%     <- indx == hindx 1
%%     -> We can factor out commonalities 
%%         /> Obtain an equivalent presentation
%%         /> Still embeddable (refer to the Agda model)
%% \end{wstructure}

%% \begin{wstructure}
%% <- Also replacing '1 by 'const  [figure]
%%     <- For convenience
%%         <- 'const X equivalent to 'sigma X (\_ -> '1)
%%         /> Easier to abstract
%%             <- Extensionally same
%%             /> 'const more useful in practice
%% \end{wstructure}

%% Before moving on to indexed descriptions, we have to carry out some
%% maintenance work on descriptions. We presented $\Desc$ as the grammar
%% of inductive types. Hence, the codes closely follow this grammar. In
%% the following, we adopt an alternative presentation. With
%% $\DSigma{\!}{\!}$, we are actually \emph{quoting} a standard
%% type-former, namely

%% $$\Bhab{\Sigma}{\PI{\V{S}}{\Set} \PI{\V{S}}{\Set} \Set}$$

%% In the alternative presentation, we go further and present all our
%% codes as quotations of standard type-formers. This presentation is
%% shown in Figure~\ref{fig:type-former-desc}. The reader will notice
%% that we replace $\DUnit$ by a more general $\DConst{\!}$ code. Whereas
%% $\DUnit$ was interpreted as the unit set, $\DConst{\V{X}}$ is
%% interpreted as $\V{X}$, for any $\Bhab{\V{X}}{\Set}$. Extensionally,
%% $\DConst{\V{X}}$ and $\DSigma{\V{X}}{\DUnit}$ are equivalent. However,
%% $\DConst{\!}$ is more succinct. More importantly, $\DConst$ is
%% \emph{first-order}, unlike its equivalent encoding. From a
%% definitional perspective, we are giving more opportunities to the
%% type-system, hence reducing the burden on the programmer. For the same
%% reason, we introduce $\DProd{\!}{\!}$ that overlaps with
%% $\DSigma{\!}{\!}$.

%% This reorganisation is strictly equivalent to the previous one
%% (Fig.~\ref{fig:hindx_desc}). Just as the previous version, it is also
%% self-descriptive. We refer the reader to the companion technical
%% report for details. In this finer-grained presentation, we can define
%% $\DIndx{\!}$ and $\DHindx{\!}{\!}$ as follow:

%% \[\begin{array}{l@{\:\mapsto\:\:}l}
%% \DIndx{\V{D}}         & \DProd{\DId}{\V{D}}                      \\
%% \DHindx{\V{H}}{\V{D}}     & \DProd{(\DPi{\V{H}}{(\LAM{\_} \DId)})}{\V{D}}
%% \end{array}
%% \]

%% Consequently, the examples previously developed can be
%% straightforwardly translated into this new presentation. For example,
%% here is the new definition of $\NatD$:

%% \[\stk{
%% \NatD : \Desc \\
%% \NatD \mapsto \DSigma{(\EnumT{[ \NatZero, \NatSuc{\!} ]})}
%%                      {[ \DUnit \quad \DId ]}
%% }\]


%% In the following, we adopt this last version as our de
%% facto universe of inductive types. In particular, we are going to
%% evolve this presentation into an indexed one.

%% \note{Shall we talk about the Type Theory being Desc Zero? or such story?}

%% \begin{figure}

%% \[\stk{
%% \begin{array}{ll}
%% \stk{
%% \data \Desc : \Set \where                                      \\
%% \;\;\begin{array}{@{}l@{\::\:\:}l@{\quad}l}
%%     \DId            & \Desc                                    \\
%%     \DConst{\!}     & \Set \To \Desc                           \\
%%     \DProd{\!}{\!}  & \PI{\V{D}, \V{D'}}{\Desc} \Desc          \\
%%     \DSigma{\!}{\!} & \PI{\V{S}}{\Set} \PIS{\V{S} \To \Desc} \Desc \\
%%     \DPi{\!}{\!}    & \PI{\V{S}}{\Set} \PIS{\V{S} \To \Desc} \Desc 
%% \end{array}
%% }
%% \vspace{0.2in}
%% \\
%% \stk{
%% \descop{\_\:}{} : \Desc \To \Set \To \Set \\
%% \begin{array}{@{}l@{\:=\:\:}ll}
%% \descop{\DId}{\V{X}}          &  \V{X}                                           \\
%% \descop{\DConst{\V{Z}}}{\V{X}}    &  \V{Z}                                           \\
%% \descop{\DProd{\V{D}}{\V{D'}}}{\V{X}} &  \TIMES{\descop{\V{D}}{\V{X}}}{\descop{\V{D}\V{'}}{\V{X}}}       \\
%% \descop{\DSigma{\V{S}}{\V{D}}}{\V{X}} &  \SIGMA{\V{s}}{\V{S}} \descop{\V{D}\: \V{s}}{\V{X}}          \\
%% \descop{\DPi{\V{S}}{\V{D}}}{\V{X}}    &  \PI{\V{s}}{\V{S}} \descop{\V{D}\: \V{s}}{\V{X}}            
%% \end{array}
%% }
%% \end{array}
%% }\]

%% \caption{Universe of descriptions based on Type-formers}
%% \label{fig:type-former-desc}

%% \end{figure}

\subsection{The universe of indexed descriptions}

\begin{wstructure}
<- Labelling Id
    <- We had: data Desc : Set -> Set
    -> We want: data IDesc : (I -> Set) -> Set
        <- Indexed functor (?)
        -> It is sufficient to label Id
            <- Where the functor is built
\end{wstructure}


\newcommand{\DotTo}{\mathop{\blue{\dot{\rightarrow}}}}

We presented the $\Desc$ universe as a
grammar of strictly positive endofunctors on $\Set$ and developed
inductive types by taking a fixpoint. To describe inductive families
indexed by some $\Bhab{\V{I}}{\Set}$, we play a similar game with
endofunctors on the category $\Set^{\V{I}}$,
families of sets \(\V{X},\V{Y}:\V{I}\To\Set\) for objects, and for morphisms,
families of functions in \(\V{X}\DotTo\V{Y}\), defined pointwise:
\[
\V{X}\DotTo\V{Y} \mapsto \PI{\V{i}}{\V{I}}\V{X}\:\V{i}\To\V{Y}\:\V{i}
\]

An \emph{indexed functor} in $\Set^{\V{I}}\To\Set^{\V{J}}$ has the
flavour of a device driver, characterising `responses' to a given
request in \(\V{J}\) where we may in turn make `subrequests' at indices
chosen from \(\V{I}\). When we use indexed functors to define inductive
families of datatypes,
\(\V{I}\) and \(\V{J}\) coincide: we explain how to make a node fit a given
index, including subnodes at chosen indices. E.g., if we are asked for a
vector of length 3, we choose to ask in turn for a tail of length 2.

To code up valid notions of response to a given request, we introduce
$\SYMBIDesc$ and its interpretation:
%
\[\stk{
\IDesc{(\Bhab{\V{I}}{\Set})} : \Set \smallskip \\
\idescop{\_}{}{} : _{\PI{\V{I}}{\Set}} \IDesc{\V{I}} \To (\V{I} \To \Set) \To \Set    \\
}\]

An \(\IDesc{\V{I}}\) specifies just \emph{one} response, but a
request-to-response \emph{function},
$\V{R}:\V{I} \To \IDesc{\V{I}}$, yields a strictly positive endofunctor
\[
  \LAM{\V{X}} \LAM{\V{i}} \idescop{\V{R}\:\V{i}}{\V{I}}{\V{X}} :
  \Set^{\V{I}} \To \Set^{\V{I}}
\]
whose fixpoint we then take:
%
\[\stkl{
\Rule{\Gamma \vdash \Bhab{\V{I}}{\Set} \qquad
      \Gamma \vdash \Bhab{\V{R}}{\V{I} \To \IDesc{\V{I}}}}
     {\Gamma \vdash \Bhab{\SYMBIMu_{\V{I}}{\V{R}}}{\V{I}\To\Set}} \qquad
\\
\Rule{\begin{array}{l@{\qquad}l}
          \Gamma \vdash \Bhab{\V{I}}{\Set} &
          \Gamma \vdash \Bhab{\V{R}}{\V{I} \To \IDesc{\V{I}}} \\
          \Gamma \vdash \Bhab{\V{i}}{\V{I}} &
          \Gamma \vdash \Bhab{\V{x}}{\idescop{\V{R}\:\V{i}}{\V{I}}{(\SYMBIMu_{\V{I}}{\V{R}})}}
      \end{array}}
     {\Gamma \vdash \Bhab{\Con{\V{x}}}{\IMu{\V{I}}{\V{R}}{\V{i}}}}
}\]

\newcommand{\upgrade}{\F{upgrade}}
\newcommand{\inductionI}{\F{indI}}
\newcommand{\cataI}{\F{cataI}}

We define the $\SYMBIDesc$ grammar in Figure~\ref{fig:idesc},
delivering only \emph{strictly positive} families. As well as
indexing our descriptions, we have refactored a little, adopting
a more compositional algebra of codes, where $\Desc$ is
biased towards the right-nested tuples. We now have
\(\DVar{i}\) for recursive `subrequests' at a chosen index \(i\),
with tupling by right-associative
\(\DProd{}{}\) and higher-order branching
by \(\DPi{\!}{\!}\).  Upgrade your old $\Desc$
to a trivially indexed $\IDesc{\Unit}$ as follows!
\[\begin{array}{@{}ll}
\upgrade :\Desc & \To\IDesc{\Unit} \\
\upgrade\:\DUnit & \mapsto \DConst{\Unit} \\
\upgrade\:(\DSigma{\V{S}}{\V{D}}) &
   \mapsto \DSigma{\V{S}}{\LAM{\V{s}}\upgrade\:(\V{D}\:\V{s})} \\
\upgrade\:(\DIndx{\V{D}}) & \mapsto
  \DProd{\DVar{\Void}}{\upgrade\:\V{D}} \\
\upgrade\:(\DHindx{\V{H}}{\V{D}}) & \mapsto
  \DProd{(\DPi{\V{H}}{\LAM{\_}\DVar{\Void}})}{\upgrade\:\V{D}} \\
\end{array}\]

To deliver induction for indexed datatypes, we need the `holds everywhere'
machinery. We present $\SYMBAllI$ and $\SYMBallI$ in
Figure~\ref{fig:allI-predicates}, with a twist---where
$\Desc$ admits the $\SYMBall$ construction, $\SYMBIDesc$ is \emph{closed}
under it! The $\SYMBAllI$
operator for a description indexed on \(\V{I}\) is strictly positive in
turn, and has a description indexed on
 some \(\SIGMA{\V{i}}{\V{I}}{\V{X}\: \V{i}}\).
Induction on indexed descriptions is then hardwired thus:
%
\[\stk{
\begin{array}{@{}ll}
\inductionI : & _{\PI{\V{I}}{\Set}}
                   \PITEL{\V{R}}{\V{I} \To \IDesc{\V{I}}}
                   \PI{\V{P}}{(\SIGMA{\V{i}}{\V{I}}{\IMu{\V{I}}{\V{R}}{\V{i}}}) \To \Set} \\
                 & (      \PITEL{\V{i}}{\V{I}} 
                          \PI{\V{xs}}{\idescop{\V{R}\: \V{i}}{\V{I}}{(\SYMBIMu_{\V{I}}{\V{R}})}} \\
                 & \   \idescop{\AllI{}
                                     {(\V{R}\: \V{i})}
                                     {(\SYMBIMu_{\V{I}}{\V{R}})}
                                     {\V{xs}}}
                               {}
                               {\V{P}} \To
                       \V{P}\: \pair{\V{i}}{\Con{\V{xs}}}{}) \To \\
                 & \PITEL{\V{i}}{\V{I}}
                   \PI{\V{x}}{\IMu{\V{I}}{\V{R}}{\V{i}}}
                   \V{P}\: \pair{\V{i}}{\V{x}}{}
\end{array} \\
\inductionI\: \V{R}\: \V{P}\: \V{m}\: \V{i}\: (\Con{\V{xs}}) \mapsto 
    \V{m}\: \V{i}\: \V{xs}\: (\allI{}
                                    {\V{R}\: \V{i}}
                                    {(\SYMBIMu_{\V{I}}{\V{R}})}
                                    {\V{P}}
                                    {(\spl{\LAM{\V{i}}\LAM{\V{xs}} \inductionI\: \V{R}\: \V{P}\: \V{m}})}
                                    {\V{xs}})
}\]
%
The generic catamorphism, $\cataI$, is constructed from $\inductionI$
as before. Its type becomes more elaborated, to deal with the
indexing:
%
\[
\begin{array}{@{}l@{}l}
\cataI :& \PITEL{\V{I}}{\Set}
          \PITEL{\V{R}}{\V{I} \To \IDesc{\V{I}}} \\
        & \PI{\V{T}}{\V{I} \To \Set}
          (\PI{\V{i}}{\V{I}}{\idescop{\V{R}\: \V{i}}{}{\V{T}} \To \V{T}\: \V{i}}) \To
          \SYMBIMu_{\V{I}}{\V{R}} \DotTo \V{T}
\end{array}
\]


\begin{figure*}

\[
\begin{array}{ll}
%%
\stk{
\begin{array}{@{}ll}
\SYMBAllI : & _{\PI{\V{I}}{\Set}}
              \PITEL{\V{D}}{\IDesc{\V{I}}}
              \PI{\V{X}}{\V{I} \To \Set} \\
            & \idescop{\V{D}}{\V{I}}{\V{X}} \To
              \IDesc{(\SIGMA{\V{i}}{\V{I}}{\V{X}\: \V{i}})}
\end{array} \\
\begin{array}{@{}l@{}l@{\:\mapsto\:\:}l}
\AllI{\:}{(\DVar{\V{i}})}{& \V{X}}{\V{x}} &
    \DVar{\pair{\V{i}}{\V{x}}{}} \\
\AllI{\:}{(\DConst{\V{K}})}{& \V{X}}{\V{k}} &
    \DConst{\Unit} \\
\AllI{\:}{(\DProd{\V{D}}{\V{D'}})}{& \V{X}}{\pair{\V{d}}{\V{d'}}{}} &
    \DProd{\AllI{}{\V{D}}{\V{X}}{\V{d}}}{\AllI{}{\V{D'}}{\V{X}}{\V{d'}}} \\
\AllI{\:}{(\DSigma{\V{S}}{\V{D}})}{& \V{X}}{\pair{\V{s}}{\V{d}}{}} &
    \AllI{}{(\V{D}\: \V{s})}{\V{X}}{\V{d}} \\
\AllI{\:}{(\DPi{\V{S}}{\V{D}})}{& \V{X}}{\V{f}} &
    \DPi{\V{S}}{\LAM{\V{s}} \AllI{}{(\V{D}\: \V{s})}{\V{X}}{(\V{f}\: \V{s})}}
\end{array}
}
&
%%
\stk{
\begin{array}{@{}ll}
\SYMBallI : & _{\PI{\V{I}}{\Set}}
              \PITEL{\V{D}}{\IDesc{\V{I}}}
              \PITEL{\V{X}}{\V{I} \To \Set} 
              \PI{\V{P}}{(\SIGMA{\V{i}}{\V{I}}{\V{X}\: \V{i}}) \To \Set} \\
            & (\PI{\V{x}}{\SIGMA{\V{i}}{\V{I}}{\V{X}\: \V{i}}} \V{P}\: \V{x}) \To
              \PI{\V{xs}}{\idescop{\V{D}}{\V{I}}{\V{X}}} 
              \idescop{\AllI{}{\V{D}}{\V{X}}{\V{xs}}}{}{\V{P}}
\end{array} \\
\begin{array}{@{}l@{}l@{\:\mapsto\:\:}l}
\allI{\:}{(\DVar{\V{i}})}{& \V{X}}{\V{P}}{\V{p}}{\V{x}} &
    \V{p}\: \pair{\V{i}}{\V{x}}{} \\
\allI{\:}{(\DConst{\V{K}})}{& \V{X}}{\V{P}}{\V{p}}{\V{k}} &
    \void \\
\allI{\:}{(\DProd{\V{D}}{\V{D'}})}{& \V{X}}{\V{P}}{\V{p}}{\pair{\V{d}}{\V{d'}}{}} &
    \pair{\allI{}{\V{D}}{\V{X}}{\V{P}}{\V{p}}{\V{d}}}
         {\allI{}{\V{D'}}{\V{X}}{\V{P}}{\V{p}}{\V{d'}}}{} \\
\allI{\:}{(\DSigma{\V{S}}{\V{D}})}{& \V{X}}{\V{P}}{\V{p}}{\pair{\V{s}}{\V{d}}{}} &
    \allI{}{(\V{D}\: \V{s})}{\V{X}}{\V{P}}{\V{p}}{\V{d}} \\
\allI{\:}{(\DPi{\V{S}}{\V{D}})}{& \V{X}}{\V{P}}{\V{p}}{\V{f}} &
    \LAM{\V{a}}\allI{}{(\V{D}\: \V{a})}{\V{X}}{\V{P}}{\V{p}}{(\V{f}\: \V{a})}
\end{array}
\end{array}
}
\]

\caption{Indexed induction predicates}
\label{fig:allI-predicates}

\end{figure*}


\begin{figure}

\[\stk{\begin{array}{@{}ll}
\IDesc{(\Bhab{\V{I}}{\Set})} &: \Set \\
\DVar{(\Bhab{\V{i}}{\V{I}})} &: \IDesc{\V{I}} \\
\DConst{(\Bhab{\V{A}}{\Set})} &:\IDesc{\V{I}}       \\
\DProd{(\Bhab{\V{D}}{\IDesc{\V{I}}})}{(\Bhab{\V{D}}{\IDesc{\V{I}}})}
  & :\IDesc{\V{I}}       \\
\DSigma{(\Bhab{\V{S}}{\Set})}{(\Bhab{\V{D}}{\V{S}\To\IDesc{\V{I}}})}
& : \IDesc{\V{I}}  \\
\DPi{(\Bhab{\V{S}}{\Set})}{(\Bhab{\V{D}}{\V{S}\To\IDesc{\V{I}}})}
& : \IDesc{\V{I}}  \\
\end{array}\smallskip \\
\idescop{\_\:}{}{} :_{\PI{\V{I}}{\Set}} \IDesc{\V{I}} \To (\V{\V{I}} \To \Set) \To \Set                  \\
\begin{array}{@{}l@{\V{X}}@{\:\mapsto\:\:}ll}
\idescop{\DVar{\V{i}}}{\V{I}}{&}      &  \V{X}\: \V{i}                                           \\
\idescop{\DConst{\V{K}}}{\V{I}}{&}    &  \V{K}                                                   \\
\idescop{\DProd{\V{D}}{\V{D'}}}{\V{I}}{&} &  \TIMES{\idescop{\V{D}}{\V{I}}{\V{X}}}{\idescop{\V{D'}}{\V{I}}{\V{X}}}       \\
\idescop{\DSigma{\V{S}}{\V{D}}}{\V{I}}{&} &  \SIGMA{\V{s}}{\V{S}} \idescop{\V{D}\: \V{s}}{\V{I}}{\V{X}}                    \\
\idescop{\DPi{\V{S}}{\V{D}}}{\V{I}}{&}    &  \PI{\V{s}}{\V{S}} \idescop{\V{D}\: \V{s}}{\V{I}}{\V{X}}            
\end{array}
}
\]

\caption{Universe of indexed descriptions}
\label{fig:idesc}

\end{figure}



\subsection{Examples}
\label{sec:idesc-examples}

\paragraph{Natural numbers:}

\begin{wstructure}
<- Nat
    -> [equation]
    <- Non-indexed types lives in IDesc 1
        -> This applies to all previous examples
\end{wstructure}

For basic reassurance, we \(\upgrade\:\NatD\):
%
\[\stk{
\upgrade\:\NatD : \IDesc{\Unit} \\
\upgrade\:\NatD \mapsto \DSigma{(\EnumT{\sqr{\NatZero\: \SYMBNatSuc}})}
                     {\sqr{(\DConst{\Unit}) \; 
                           (\DProd{\DVar{\Void}}{\DConst{\Unit}})}}
}\]
%
Note that trailing \(\Unit\)'s keep our right-nested, \(\void\)-terminated
tuple structure, and with it our elaboration machinery.
We can similarly \(\upgrade\) any inductive type.
Moreover, \(\IDesc{I}\) can now code a bunch of mutually
inductive types, if \(I\) enumerates the
bunch~\cite{paulin:habilitation, yakushev:mutual-def}.



\paragraph{Indexed descriptions:}

\begin{wstructure}
<- Levitation [figure]
    <- Following Desc encoding
        /> Note: simple datatype
            -> Live in IDesc 1
    -> Behind the scene, relies on the special purpose switchD
\end{wstructure}

Note that $\IDesc{\V{I}}$ is a plain inductive type, parametrised
by \(\V{I}\), but indexed trivially.
%
\[\stk{
\IDescD : \PI{\V{I}}{\Set} \IDesc{\Unit} \\
\IDescD\: \V{I} \mapsto \SYMBDSigma \\
\quad
 \EnumT\vtup{r}{\SYMBDVar\\
                \SYMBDConst\\
                \DProd{}{}\\
                \SYMBDSigma\\
                \SYMBDPi}
         \; \vtup{l@{}l}{
  (\DProd{\DConst{\V{I}} &}{\DConst{\Unit}})                  \\
  (\DProd{\DConst{\Set}  &}{\DConst{\Unit}})                  \\
  (\DProd{\DVar{\Void}}{\DProd{\DVar{\Void}&}{\DConst{\Unit}}})  \\
  (\DSigma{\Set}{\LAM{\V{S}}
     \DProd{( \DPi{\V{S}}{\LAM{\_} \DVar{\Void}}) &}{\DConst{\Unit}}})     \\
  (\DSigma{\Set}{\LAM{\V{S}}
     \DProd{( \DPi{\V{S}}{\LAM{\_} \DVar{\Void}}) &}{\DConst{\Unit}}})     \\
                                   }
}\]

Therefore, this universe is self-describing and can be
levitated. As before, we rely on a special purpose $\F{switchID}$
operator to build the finite function $\bigRedBracket{\ldots}$
without mentioning \(\SYMBIDesc\).

\paragraph{Vectors:}

\newcommand{\VecD}{\F{VecD}}
\newcommand{\VecNil}{\etag{\CN{vnil}}}
\newcommand{\SYMBVecCons}{\etag{\CN{vcons}}\xspace}
\newcommand{\VecCons}[2]{\SYMBVecCons\:#1\:#2}

So far, our examples live in $\IDesc{\Unit}$, with no interesting
indexing. Let us at least have vectors. Recall
that the constructors $\VecNil$ and $\SYMBVecCons$ are defined only for
$\NatZero$ and $\NatSuc$ respectively:
%
\[
\stk{
\data \D{Vec}\: \PITEL{\V{X}}{\Set} : \PI{\V{i}}{\Nat} \Set \where \\
\;\;\begin{array}{@{}l@{\::\:\:}l@{\quad}l}
    \VecNil          & \D{Vec}\:\V{X}\:{\NatZero}   \\
    \SYMBVecCons & _{\PI{\V{n}}{\Nat}}\V{X} \To \D{Vec}\:{\V{X}}\:{\V{n}} \To \D{Vec}\:{\V{X}}\:{(\NatSuc{\V{n}})}
\end{array}
}
\]

One way to code constrained datatypes is to appeal to a suitable
notion of propositional equality \(\PropEq\) on indices. The
constraints are expressed as `Henry Ford' equations in the datatype.
For vectors:
%
\[\stk{
\VecD : \Set \To \Nat \To \IDesc{\Nat} \\
\VecD\: \V{X}\: \V{i} \mapsto \SYMBDSigma\\
\quad
\EnumT{\vtup{r}{\VecNil\\ \SYMBVecCons}}
\; \vtup{r}{
                            (\DConst{(\NatZero\PropEq\V{i})}) \\
 ( \DSigma{\Nat}{\LAM{\V{n}}
   \DProd{\DConst{\V{X}}}
     {\DProd{\DVar{\V{n}}}{\DConst{(\NatSuc{\V{n}}\PropEq\V{i})}})}}
                          }
}\]

You may choose $\VecNil$ for any index you like as long as it is
$\NatZero$; in the $\SYMBVecCons$ case, the length of the tail is
given explicitly, and the index $\V{i}$ must be one more. Our previous
\(\Unit\)-terminated tuple types can now be seen as the trivial case
of constraint-terminated tuple types, with elaboration supplying the
witnesses when trivial.

In this paper, we remain anxiously agnostic about
propositional equality. Any will do, according to
conviction; many variations are popular. The
homogeneous identity type used in Coq is ill-suited to
dependent types, but its heterogeneous variant (forming equations
regardless of type) allows the translation of pattern
matching with structural recursion to
\(\F{indI}\)~\cite{goguen:pattern-matching}. The
extensional equality of \citet{altenkirch:ott} also sustains the translation.

\begin{wstructure}
!!! Need Help !!!
<- Brady optimisation: forcing
    <- Source to source translation
    <- Able to remove some constraints
    -> Example: Fin [figure]
    ??? More technical detail needed
\end{wstructure}

However, sometimes, the equations are redundant. 
Looking back at $\D{Vec}$, we find that the equations constrain
the choice of constructor and stored tail index retrospectively.
But \emph{inductive families need not store their
  indices}~\cite{brady:index-inductive-families}!  If we
analyse the incoming index, we can tidy our description of $\D{Vec}$
as follows:
%
\[\stk{
\VecD \:\PITEL{\V{X}}{\Set} : \Nat \To \IDesc{\Nat} \\
\begin{array}{@{}lll}
\VecD\:\V{X}\: \NatZero     & \mapsto & \DConst{\Unit} \\
\VecD\:\V{X}\: (\NatSuc{\V{n}}) & \mapsto &
 \DProd{\DConst{\V{X}}}{\DVar{\V{n}}}
\end{array}
                                       
}\]
%
The constructors and equations have simply disappeared. A similar
example is $\SYMBFin$ (bounded numbers), specified by:
%
\[
\stk{
\data \SYMBFin : \PI{\V{n}}{\Nat} \Set \where \\
\;\;\begin{array}{@{}l@{\::\:\:}l@{\quad}l}
    \FinZero      & _{\PI{\V{n}}{\Nat}}\Fin{(\NatSuc{\V{n}})}   \\
    \SYMBFinSuc   & _{\PI{\V{n}}{\Nat}}\Fin{\V{n}} \To \Fin{(\NatSuc{\V{n}})}
\end{array}
}\]
%
In this case, we can eliminate equations but not constructors, since both
$\FinZero$ and $\SYMBFinSuc$ both target $\SYMBNatSuc$:
%
\[\stk{
\FinD : \Nat \To \IDesc{\Nat} \\
\begin{array}{@{}lll}
\FinD\: \NatZero         & \mapsto & \DSigma{\EnumT{\Void}}{\Void} \\
\FinD\: (\NatSuc{\V{n}}) & \mapsto & \DSigma{\EnumT{\sqr{\FinZero\: \SYMBFinSuc}}}
                                            {\sqr{(\DConst{\Unit})\: (\DVar{\V{n}})}}
\end{array}
}\]

This technique of extracting information by case analysis on indices
applies to descriptions exactly where Brady's `forcing' and
`detagging' optimisations apply in compilation. They eliminate just
those constructors, indices and constraints which are redundant even
in \emph{open} computation. In \emph{closed} computation, where proofs
can be trusted, all constraints are dropped.


\paragraph{Tagged indexed descriptions:}

\newcommand{\SYMBmuide}{\D{\({\mu}^{\!+}\)}\xspace}
\newcommand{\muide}[2]{\SYMBmuide\!\!_{#1}\:#2}

Let us reflect this index analysis technique.
We can divide a description of tagged indexed data in two: first, the
constructors that do not depend on the index; then, the constructors
that do. The non-dependent part mirrors the definition for non-indexed
descriptions. The index-dependent part simply indexes the choice of
constructors by $\V{I}$. Hence, by inspecting the index, it is
possible to vary the `menu' of constructors.
%
\[
\begin{array}{@{}l@{\:\mapsto\:\:}l}
 \TagIDesc{\V{I}}  & \TIMES{\ATagIDesc{\V{I}}}{\ITagIDesc{\V{I}}} \\
 \ATagIDesc{\V{I}} & \SIGMA{\V{E}}{\EnumU} \PI{\V{i}}{\V{I}} \spi{\V{E}}{\LAM{\_} \IDesc{\V{I}}} \\
 \ITagIDesc{\V{I}} & 
     \SIGMA{\V{F}}{\V{I} \To \EnumU} \PI{\V{i}}{\V{I}} \spi{(\V{F}\: \V{i})}{\LAM{\_} \IDesc{\V{I}}} 
\end{array}
\]

\begin{wstructure}
<- Vectors
    Do we treat them in the end? 
    What can we say here we haven't with Fin?
\end{wstructure}

In the case of a tagged $\D{Vec}$, for instance, for the index
$\NatZero$, we would only propose the constructor
$\ListNil$. Similarly, for $\NatSuc{n}$, we would only propose the
constructor $\SYMBListCons$.

We write $\toIDesc{\V{D}}\:\V{i}$ to denote the $\IDesc{\V{I}}$
computed from the tagged indexed description $\V{D}$ at index
$\V{i}$. Its expansion is similar to the definition of \(\SYMBtoDesc\)
for tagged descriptions, except that it must also append the two parts.
We again write $\muide{\V{I}}{\V{D}}$ for
$\IMu{\V{I}}{(\toIDesc{\V{D}})}$.

\paragraph{Typed expressions:}

\begin{wstructure}
<- Hutton's razor
    <- Types
        <- 'Nat
        <- 'Bool
    <- Term [figure]
        <- val : Val 'a -> 'a  for Val : Ty -> Set, mapping to Nat and Bool
        <- cond : 'Bool -> a -> a -> a
        <- plus : 'Nat -> 'Nat -> 'Nat
        <- le : 'Nat -> 'Nat -> 'Bool
\end{wstructure}

%% Types
\newcommand{\Ty}{\D{Ty}}
\newcommand{\Ebool}{\etag{\CN{bool}}}
\newcommand{\Enat}{\etag{\CN{nat}}}

%% Constructors
\newcommand{\SYMBEval}{\etag{\CN{val}}\xspace}
\newcommand{\SYMBEvar}{\etag{\CN{var}}\xspace}
\newcommand{\Eval}[1]{\SYMBEval\:#1}
\newcommand{\SYMBEcond}{\etag{\CN{cond}}\xspace}
\newcommand{\Econd}[3]{\SYMBEcond\:#1\:#2\:#3}
\newcommand{\SYMBEplus}{\etag{\CN{plus}}\xspace}
\newcommand{\Eplus}[2]{\SYMBEplus\:#1\:#2}
\newcommand{\SYMBEle}{\etag{\CN{le}}\xspace}
\newcommand{\Ele}[2]{\SYMBEle\:#1\:#2}

%% Index mapper (terminology?)
\newcommand{\SYMBVal}{\F{Val}\xspace}
\newcommand{\Val}[1]{\SYMBVal\:#1}
\newcommand{\SYMBVar}{\F{Var}\xspace}
\newcommand{\Var}[2]{\SYMBVar\: #1\: #2}

%% Hutton expressions
\newcommand{\HExprD}{\F{ExprD}}
\newcommand{\HExprAD}{\F{ExprAD}}
\newcommand{\HExprID}{\F{ExprID}}
\newcommand{\HExprVarD}[1]{\C{ExprD}_{\F{Var},#1}}
\newcommand{\HExprFreeD}{\C{ExprD}^{\C{Free}}}
\newcommand{\HExprAFreeD}{\C{ExprAD}^{\C{Free}}}

We are going to define a syntax for a small language with
two types, natural numbers and booleans:
%
\[
\Ty \mapsto \EnumT{\sqr{\Enat\: \Ebool}}
\]

\newcommand{\plusHost}{\mathop{\green{+_{\mathrm{H}}}}}
\newcommand{\leHost}{\mathop{\green{\leq_{\mathrm{H}}}}}


This language has values, conditional expression, addition and
comparison. Informally, their types are:
%
\[
\begin{array}{l@{\::\:\:}l}
\SYMBEval            & \Val{\V{ty}} \To \V{ty} \\
\SYMBEcond           & \Ebool \To \V{ty} \To \V{ty} \To \V{ty}  \\ 
\end{array}
\qquad
\begin{array}{l@{\::\:\:}l}
\SYMBEplus           & \Enat \To \Enat \To \Enat                           \\
\SYMBEle             & \Enat \To \Enat \To \Ebool                          \\
\end{array}
\]
%
The function $\SYMBVal$ interprets object language types in the
host language, so that arguments to $\SYMBEval$ fit their
expected type.
%
\[\stk{
\SYMBVal : \Ty \To \Set \\
\begin{array}{@{}l@{\:\mapsto\:\:}l}
\Val{\Enat}   & \Nat \\
\Val{\Ebool}  & \Bool
\end{array}
}\]
%
We take $\Nat$ and $\Bool$ to represent natural numbers and Booleans
in the host language, equipped with addition $\plusHost$ and
comparison $\leHost$.

We express our syntax as a tagged indexed description, indexing over
object language types $\Ty$. We note that some constructors are always
available, namely $\SYMBEval$ and $\SYMBEcond$. On the other hand,
$\SYMBEplus$ and $\SYMBEle$ constructors are index-dependent, with
$\SYMBEplus$ available just when building a $\Enat$, $\SYMBEle$ just
for $\Ebool$. The code, below, reflects this intuition, with the first
component uniformly offering $\SYMBEval$ and $\SYMBEcond$, the second
selectively offering $\SYMBEplus$ or $\SYMBEle$.
%
%%% \begin{figure}
%
\[\stk{
\stk{
\HExprD : \TagIDesc{\Ty} \\
\HExprD \mapsto \sqr{ \HExprAD , \HExprID } \\
} \smallskip\\
\stk{
\HExprAD : \ATagIDesc{\Ty} \\
\HExprAD \mapsto \vtup{l}{
   {\vtup{r}{\SYMBEval\\ \SYMBEcond \,}} \red{,} \;
      \LAM{\V{ty}}
      \vtup{l@{\:}l}{
      \DProd{\DConst{(\Val{\V{ty}})}&}{\DConst{\Unit}} \\
      \DProd{\DProd{\DVar{\Ebool}}{\DProd{\DVar{\V{ty}}}{\DVar{\V{ty}}}}&}
        {\DConst{\Unit}} \\
     }
   }
\smallskip\\
} 
\\
\stk{
\HExprID : \ITagIDesc{\Ty} \\
\HExprID \mapsto \vtup{l}{
                   \vtup{r}{\sqr{\SYMBEplus} \\ \sqr{\SYMBEle}} \red{,} \;
  \LAM{\_} \sqr{\DProd{\DProd{\DVar{\Enat}}{\DVar{\Enat}}}{\DConst{\Unit}}}
                   }
}
}\]

%%%\caption{Syntax of typed expressions}
%%%\label{fig:hexpr-full}

%%%\end{figure}

\newcommand{\evalH}{\F{eval}_{\green{\Downarrow}}}
\newcommand{\evalOne}{\F{eval}_{\green{\downarrow}}}

Given the syntax, let us supply the semantics. We implement an
evaluator as a catamorphism:
%
\[\stk{
\evalH : \PI{\V{ty}}{\Ty} 
         \muide{\Ty}{\HExprD}\: \V{ty} \To
         \Val{\V{ty}} \\
\evalH\: \V{ty}\: \V{term} \mapsto \cataI_{\Ty} \:
                                 (\toIDesc{\HExprD})\: 
                                 \SYMBVal\: 
                                 \evalOne\: 
                                 \V{ty}\: 
                                 \V{term}
}\]
%
To finish the job, we must supply the algebra which implements a single
step of evaluation, given subexpressions evaluated already.
%
\[\stk{
\evalOne : \PI{\V{ty}}{\Ty}
 \idescop{(\toIDesc{\HExprD})\:\V{ty}}{\Ty}{\SYMBVal}
           \To {\Val{\V{ty}}} \\
\begin{array}{@{}l@{}c@{}l@{\:\mapsto\:\:}l}
\evalOne\: & \_\: & (\SYMBEval\;\V{x})                                            & \V{x} \\
\evalOne\: & \_\: & (\SYMBEcond\:\BoolTrue\:\V{x}\:\_)   & \V{x} \\
\evalOne\: & \_\: & (\SYMBEcond\:\BoolFalse\:\_\:\V{y})  & \V{y} \\
\evalOne\: & \Enat\: & (\SYMBEplus\:\V{x}\:\V{y})   & \V{x} \plusHost \V{y} \\
\evalOne\: & \Ebool\: & (\SYMBEle\:\V{x}\:\V{y})  & \V{x} \leHost \V{y} 
\end{array}
}\]

\begin{wstructure}
    /> Closed term
        <- only constants and operations on them
        -> Extend Val with Var : Ty -> Set, mapping to EnumU
            -> Open term
            -> Language of well-typed terms
                <- By construction
\end{wstructure}

Hence, we have a type-safe syntax and a tagless interpreter for our
language, in the spirit
of~\citet{augustsson.carlsson:dependent.interpreter}, with help from
the generic catamorphism. However, so far, we are only able to define
and manipulate \emph{closed} terms. Adding variables, it is possible
to build and manipulate \emph{open} terms, that is, terms in a
context. We shall get this representation, for free, thanks to the
\emph{free indexed monad} construction.


\subsection{Free indexed monad}

\begin{wstructure}
<- Variation on a theme: free imonad construction
    <- Recall existence of generic free monad construction
    -> Present its generalisation to IDesc [equation]
        <- \I -> IDesc I as describing an indexed endofunctor
        <- Free monad construction
    -> Still a suitable, generic notion of substitution
        <- show type signature
        <- show implementation?? (space! space!)
\end{wstructure}

In Section~\ref{sec:desc-free-monad}, we have built a free monad
operation for simple descriptions. The process is similar in the
indexed world. Namely, given an indexed functor, we derive the indexed
functor coding its free monad: \note{pwm: Whoa there. Maybe we should
  say something about IMonads in general before we get to this point?}
%
\[\stk{
\begin{array}{ll}
\FreeIMonad{\_}{} : & _{\PI{\V{I}}{\Set}}
                     \PITEL{\V{R}}{\TagIDesc{\V{I}}} 
                     \PITEL{\V{X}}{\V{I} \To \Set}\To 
                      \TagIDesc{\V{I}}
\end{array} \\
\FreeIMonad{\pair{\V{E}}{\V{F}}{}}{\V{I}}{\V{R}} \mapsto
    \pair{\pair{\ListCons{\SYMBDVar}{(\fst{\V{E}})}} 
               {\LAM{\V{i}}
                \pair{\DConst{(\V{R}\: \V{i})}}
                     {(\snd{\V{E}})\: \V{i}}{}}{}}
         {\V{F}}{}
}\]


\newcommand{\substI}{\F{substI}}


Just as in the universe of descriptions, this construction comes with
an obvious \return and a substitution operation, the \bind. Its
definition is the following:
%
\[\stk{
\begin{array}{@{}ll}
\substI : & _{\PI{\V{I}}{\Set}}
            \PI{\V{X}, \V{Y}}{\V{I} \To \Set}
            \PITEL{\V{R}}{\TagIDesc{\V{I}}} \\
          & (\V{X} \DotTo 
             \muide{\V{I}}{(\FreeIMonad{\V{R}}{\V{I}}{\V{Y}})}) \To 
             \muide{\V{I}}{(\FreeIMonad{\V{R}}{\V{I}}{\V{X}})} \DotTo
             \muide{\V{I}}{(\FreeIMonad{\V{R}}{\V{I}}{\V{Y}})}
\end{array} \\
\substI\: \V{X}\: \V{Y}\: \V{R}\: \V{\sigma}\: \V{i}\: \V{t} \mapsto \\
\qquad    \cataI_{\V{I}}\: (\toIDesc{\FreeIMonad{\V{R}}{}{\V{X}}})\:
                      (\muide{\V{Y}}{(\FreeIMonad{\V{R}}{}{\V{Y}})})\:
                      (\F{applyI}\: \V{R}\: \V{X}\: \V{Y}\: \V{\sigma})\:
                      \V{i}\:
                      \V{t} 
}\]
% 
where  $\F{applyI}$ is defined as follows:
%
\[\stk{
\begin{array}{@{}ll}
\F{applyI} : & _{\PI{\V{I}}{\Set}}
            \PITEL{\V{R}}{\TagIDesc{\V{I}}}
            \PI{\V{X}, \V{Y}}{\V{I} \To \Set} \\
          & (\V{X} \DotTo \muide{\V{I}}{(\FreeIMonad{\V{R}}{\V{I}}{\V{Y}})}{}) \To \\
          & \idescop{\toIDesc{\FreeIMonad{\V{R}}
                                         {\V{I}}
                                         {\V{X}}}}
                    {\V{I}}
                    {\muide{\V{I}}{(\FreeIMonad{\V{R}}{\V{I}}{\V{Y}})}} \DotTo 
            \muide{\V{I}}{(\FreeIMonad{\V{R}}{\V{I}}{\V{Y}})}{}
\end{array} \\
\begin{array}{@{}l@{\:\mapsto\:\:}l}
\F{applyI}\: \V{R}\: \V{X}\: \V{Y}\: \V{\sigma}\: \V{i}\: \pair{\SYMBDVar}{\V{x}}{}   & \V{\sigma}\: \V{i}\: \V{x}                   \\
\F{applyI}\: \V{R}\: \V{X}\: \V{Y}\: \V{\sigma}\: \V{i}\: \pair{\V{c}}{\V{ys}}{} & \Con{\pair{\V{c}}{\V{ys}}{}}
\end{array}
}\]
 
The subscripted types corresponds to implicit arguments that can be
automatically inferred, hence do not have to be typed in. Let us now
consider two examples of free indexed monads.


\paragraph{Typed expressions:}

\begin{wstructure}
    /> Closed term
        <- only constants and operations on them
        -> Extend Val with Var : Ty -> Set, mapping to EnumU
            -> Open term
            -> Language of well-typed terms
                <- By construction
\end{wstructure}

\newcommand{\Ctxt}{\D{Context}}
\newcommand{\SYMBCtxtEmpty}{\C{[]}\xspace}
\newcommand{\CtxtEmpty}{\SYMBCtxtEmpty}
\newcommand{\SYMBCtxtSnoc}{\C{snoc}\xspace}
\newcommand{\CtxtSnoc}[2]{\SYMBCtxtSnoc\:#1\:#2}
\newcommand{\SYMBEnv}{\F{Env}}
\newcommand{\Env}[1]{\SYMBEnv\: #1}
\newcommand{\SYMBlookup}{\F{lookup}}
\newcommand{\lookup}[4]{\SYMBlookup\: #1\: #2\: #3\: #4}

In the previous section, we presented a language of closed
arithmetic expressions. Using the free monad construction, we are
going to extend this construction to open terms. An open term is
defined with respect to a context, represented by a snoc-list of
types:
%
\[
\begin{array}{@{}l@{\::\:\:}l@{\quad}l}
\Ctxt           & \Set \\
\SYMBCtxtEmpty  & \Ctxt \\
\SYMBCtxtSnoc   & \Ctxt \To \Ty \To \Ctxt
\end{array}
\]
%
An environment realises the context, packing a value for each type:
%
\[
\stk{
\SYMBEnv : \Ctxt \To \Set \\
\begin{array}{@{}l@{\:\:\mapsto\:\:}l}
\Env{\CtxtEmpty}                & \Unit \\
\Env{(\CtxtSnoc{\V{G}}{\V{S}})} & \TIMES{\Env{\V{G}}}{\Val{\V{S}}}
\end{array}
}\]
%
In this setting, we define type variables, $\SYMBVar$ by:
%
\[\stk{
\Var{}{} : \Ctxt \To \Ty \To \Set \\
\begin{array}{@{}ll@{\:\:\mapsto\:\:}l}
\Var{\CtxtEmpty}{& \V{T}}                & 
    \Void \\
\Var{(\CtxtSnoc{\V{G}}{\V{S}})}{& \V{T}} & 
    \SUM{(\Var{\V{G}}{\V{T}})}{(\V{S} \PropEq \V{T})}
\end{array}
}\]
%
While $\SYMBVal$ maps the type to the corresponding host type,
$\SYMBVar$ indexes a value in the context, obtaining a proof that the
types match. The $\SYMBlookup$ function precisely follow this
semantics:
%
\[\stk{
\SYMBlookup : \PI{\V{G}}{\Ctxt} 
          \Env{\V{G}} \To 
          \PI{\V{T}}{\Ty} 
          \Var{\V{G}}{\V{T}} \To
          \Val{\V{T}} \\
\begin{array}{@{}l@{}l@{}l@{}l@{}lll}
\lookup{& (\CtxtSnoc{\V{G}}{.T})}{& \pair{\V{g}}{\V{t}}{}}{& \V{T}}{& (\SumRight{\C{refl}})} & \mapsto & \V{t} \\
\lookup{& (\CtxtSnoc{\V{G}}{\V{S}})}{& \pair{\V{g}}{\V{t}}{}}{& \V{T}}{& (\SumLeft{\V{x}})} & \mapsto & \lookup{\V{G}}{\V{g}}{\V{T}}{\V{x}} 
\end{array}
}\]

\newcommand{\SYMBEmpty}{\F{Empty}\xspace}
\newcommand{\Empty}[1]{\SYMBEmpty\:#1}

\newcommand{\SYMBopenTerm}{\F{openTm}\xspace}
\newcommand{\openTerm}[1]{\SYMBopenTerm\: #1}
\newcommand{\closeTerm}{\F{closeTm}}

\newcommand{\update}{\F{update}}

Consequently, taking the free monad of \(\HExprD\) by \(\SYMBVar\:
\V{G}\), we obtain the language of open terms in a context \(\V{G}\):
%
\[
\openTerm{\V{G}} \mapsto \FreeIMonad{\HExprD}{\Ty}{(\SYMBVar\:\V{G})}
\]
%
In this setting, the language of closed terms corresponds to the free
monad assigning an empty set of values to variables
%
\[
\closeTerm \mapsto \FreeIMonad{\HExprD}{\Ty}{\SYMBEmpty}
\quad
\mbox{where}
\quad
\stk{
\SYMBEmpty : \Ty \To \Set \\
\begin{array}{@{}l@{\:\:\mapsto\:\:}l}
\Empty{\_}   & \Zero \\
\end{array}
}\]
%
Allowing variables from an empty set is much like forbidding variables,
so \(\closeTerm\) and \(\HExprD\) describe isomorphic
datatypes. Correspondingly, you can update an old \(\HExprD\) to a shiny
\(\closeTerm\):
%
\[\stk{
\update : \muide{\Ty}{\HExprD} \DotTo \muide{\Ty}{\closeTerm} \\
\begin{array}{@{}l@{}l}
\update\: \V{ty}\: \V{tm} \mapsto \cataI_{\Ty}\: & (\toIDesc{\HExprD})\:
                                                  (\muide{\Ty}{\closeTerm})\: \\
                                                & (\LAM{\_} \LAM{\pair{\V{tag}}{\V{tm}}{}} \Con{\pair{\Su{\V{tag}}}{\V{tm}}{}})\:
                                                  \V{ty}\:
                                                  \V{tm}
\end{array}
}\]
% 
The other direction of the isomorphism is straightforward, the
$\SYMBDVar$ case being impossible. Therefore, we are entitled to
reuse the $\evalH$ function to define the semantics of $\closeTerm$.

\newcommand{\discharge}{\F{discharge}}

Now we would like to give a semantics to the open term language. We
proceed in two steps: first, we substitute variables by their value in
the context; then, we evaluate the resulting closed term. Thanks to
$\evalH$, the second problem is already solved. Let us focus on
substituting variables from the context. Again, we can subdivide this
problem: first, discharging a single variable from the context; then,
applying this $\discharge$ function on every variables in the term.

The $\discharge$ function is relative to the required type and a
context of the right type. Its action is to map values to themselves,
and variables to their value in context. This corresponds to the
following function:
%
\[\stk{
\begin{array}{@{}ll}
\discharge : & \PI{\V{G}}{\Ctxt}
               \Env{\V{G}} \To 
               \Var{\V{G}}{} \DotTo
               \muide{\Ty}{\closeTerm}
\end{array} \\
\begin{array}{@{}l@{\:\mapsto\:\:}l}
\discharge\: \V{G}\: \V{g}\: \V{ty}\: \V{v} &
    \Con{\pair{\SYMBEval}{\lookup{\V{G}}{\V{g}}{\V{ty}}{\V{v}}}{}}
\end{array}
}\]

\begin{wstructure}
            /> Then, perform subst everywhere in the term
                -> Show type [code]
                /> This is a bind!?
                -> There is some more structure 
                    -> We should try to get it
\end{wstructure}

\newcommand{\substH}{\F{substExpr}}

We are now left with applying $\discharge$ over all variables of the
term.  We simply have to fill in the right arguments to $\substI$, the
type guiding us:
%
\[
\stk{
\begin{array}{@{}ll}
\substH  : & \PI{\V{G}}{\Ctxt} \\
           & (\Var{\V{G}}{} \DotTo
              \muide{\Ty}{\closeTerm}) \DotTo \\
          & \muide{\Ty}{(\openTerm{\V{G}})} \DotTo 
            \muide{\Ty}{\closeTerm}
\end{array} \\
\substH\: \V{G}\:
          \V{ty}\:          
          \V{g}\:
          \V{\sigma}\: 
          \V{tm} \mapsto  
\substI_{\Ty}\:
               (\SYMBVar\: \V{G})\: 
               \SYMBEmpty\:
               \HExprD\: 
               \V{\sigma}\:
               \V{ty}\:
               \V{tm}
}\]

Hence completing our implementation of the open terms
interpreter. Without much effort, we have described the syntax of a
well-typed language, together with its semantics.


\paragraph{Indexed descriptions:}

An interesting instance of free monad is $\SYMBIDesc$ itself. Indeed,
$\SYMBDVar$ is nothing but the \return. The remaining constructors form
the carrier functor, trivially indexed by $\Unit$. The signature functor
is described as follow:
%
\[\stk{
\IDescFreeD : \ATagIDesc{\Unit} \\
\begin{array}{@{}ll}
\IDescFreeD \mapsto \bigRedBracket{\begin{array}{l}
                                \sqr{\SYMBDConst\:\:
                                     \DProd\:\:
                                     \SYMBDSigma\:\:
                                     \SYMBDPi} \red{,}\\
                                  \LAM{\_}\bigRedBracket{\begin{array}{l}
                                        \DConst{\Set}               \\
                                        \DProd{\DVar{\Void}}{\DVar{\Void}}  \\
                                        \DSigma{\Set}{(\LAM{\V{S}} \DPi{\V{S}}{(\LAM{\_} \DVar{\Void})})} \\
                                        \DSigma{\Set}{(\LAM{\V{S}} \DPi{\V{S}}{(\LAM{\_} \DVar{\Void})})}
                                    \end{array}}\end{array}}
\end{array}
}\]
%
We get $\IDesc{\V{I}}$ by extending the signature with variables from \(\V{I}\):
%
\[\stk{
\IDescD : \PI{\V{I}}{\Set} \TagIDesc{\Unit} \\
\IDescD\: \V{I} \mapsto \FreeIMonad{\red{[}\IDescFreeD\red{,[}\LAM{\_}\sqr{}\red{,}\LAM{\_}\red{[]]]}}{\Unit}\LAM{\_}\V{I}
}\]

The fact that indexed descriptions are closed under substitution
is potentially of considerable utility, if we can exploit this fact:
\[
\idescop{\V{\sigma} \V{D}}{\V{J}}{\V{X}} 
    \mapsto 
        \idescop{\V{D}}
                {\V{I}}
                {\LAM{\V{i}}
                     {\idescop{\V{\sigma} \V{i}}
                              {\V{J}}
                              {\V{X}}}}
        \quad \mbox{where}\;\V{\sigma}:\V{I}\To\IDesc{\V{J}}
\]
By observing that a description can be decomposed via substitution, we
split its meaning into a superstructure of substructures, e.g. a
`database containing salaries', ready for traversal operations
preserving the former and targeting the latter.
 
%\newpage

%%%%%%%%%%%%%%%%%%%%%%%%%%%%%%%%%%%%%%%%%%%%%%%%%%%%%%%%%%%%%%%%
%% Discussion
%%%%%%%%%%%%%%%%%%%%%%%%%%%%%%%%%%%%%%%%%%%%%%%%%%%%%%%%%%%%%%%%

\section{Discussion}


In this paper, we have presented a universe of datatypes for a
dependent type theory. We started from an unremarkable type theory
with dependent functions and tuples, but relying on few other
assumptions, especially where propositional equality is concerned.  We
added finite enumeration sufficient to account for constructor choice,
and then we built coding systems, first (as a learning experience) for
simple ML-like inductive types, then for the indexed inductive
families which dependently typed programmers in Agda, Coq and Epigram
take for granted. We adopt a bidirectional type propagation mechanism
to conceal artifacts of the encoding, giving a familiar and practicable
constructor-based presentation to data.

Crucially to our approach, we ensure that the codes describing
datatypes inhabit a datatype with a code. In a stratified setting, we
avoid paradox by ensuring that this type of codes lives uniformly one
level above the types the codes describe. The adoption of ordinary
data to describe types admits datatype-generic operations implemented
just by ordinary programming. In working this way, we make
considerable use of type equality modulo open computation, silently
specialising the types of generic operations as far as the datatype
code for any given usage is known.

\subsection{Related work in Generic Programming}

\begin{wstructure}
!!! Need Help !!!
<- Comparison with Induction Recursion
    ???
\end{wstructure}


\begin{wstructure}
!!! Need Help !!!
<- Related Work
    <- Generic in simply-typed functional languages
        <- PolyP \cite{jansson:polyp}
        <- Generic Haskell \cite{hinze:generic-haskell}
        <- Scratch your boilerplate \cite{spj:syb}
\end{wstructure}

Generic programming is a vast topic. We refer our reader to
\citet{garcia:generic-comparative-study} for a broad overview of
generic programming in various languages. For Haskell alone, there is
a myriad of proposals: 
\citet{hinze:generic-approach-comparative} and
\citet{rodriguez:generic-libs-comparative} provide useful comparative
surveys.

Our approach follows the polytypic programming style, as initiated
by PolyP~\cite{jansson:polyp}. Indeed, we build generic functions by
induction on pattern functors, exploiting type-level computation
to avoid the preprocessing phase: our
datatypes are, natively, nothing but codes.

We have the \emph{type-indexed datatypes} of Generic
Haskell~\cite{hinze:generic-haskell} for free.
From one datatype, we can compute
others and equip them with relevant structure: the free monad
construction provides one example. Our approach to encoding datatypes
as data also sustains \emph{generic
  views}~\cite{holdermans:generic-view}, allowing us to rebias the
presentation of datatypes conveniently. Tagged descriptions,
giving us a sum-of-sigmas view, are a natural example.

Unlike Generic Haskell, we do not support polykinded
programming~\cite{hinze:polytypic-polykinded}. Our descriptions are
limited to endofunctors on $\Set^I$. Whilst indexing is known
to be sufficient to \emph{encode} a large class of higher-kinded
datatypes~\citep{DBLP:conf/ifip2-1/AltenkirchM02}, we should rather
hope to work in a more compositional style. We are free to write
higher-order programs manipulating codes, but is not yet clear whether
that is sufficient to deliver abstraction at higher kinds.
Similarly, it will be interesting to see whether
arity-generic programming~\cite{weirich:arity-generic} arises
just by computing with our codes, or whether a richer abstraction is
called for.

The Scrap Your Boilerplate~\cite{spj:syb} (SYB) approach to generic
programming offers a way to construct generic functions, based on
dynamic type-testing via the $\CN{Typeable}$
type class. SYB cannot compute types from codes, but its dynamic character
does allow a more flexible \emph{ad hoc} approach to generic data
traversal. By maintaining the correspondence between codes and types
whilst supporting arbitrary inspection of codes, we pursue the same
flexibility statically. The substitutive character of $\SYMBIDesc$
may allow us to observe and exploit \emph{ad hoc} substructural
relationships in data, but again, further work is needed if we
are to make a proper comparison.

\begin{wstructure}
    <- Generic in dependent types
        <- Norell \cite{norell:msc-thesis}
        <- Polytypic prog in Coq \cite{verbruggen:polytype-coq}
        <- Universes for generic prog \cite{benke:universe-generic-prog}
\end{wstructure}

\subsection{Generic Programming with Dependent Types}

Generic programming is not new to dependent
types. \citet{DBLP:conf/ifip2-1/AltenkirchM02} developed a universe
of polykinded types in Lego;
\citet{norell:msc-thesis} gave a formalisation of
polytypic programming in Alfa, a precursor to Agda;
\citet{verbruggen:polytype-prog-coq, verbruggen:polytype-coq} provided
a framework for polytypic programming in the Coq theorem
prover. However, these works aim at \emph{modelling} PolyP or Generic
Haskell in a dependently-typed setting for the purpose of proving
correctness properties of Haskell code. Our approach is different in
that we aim at building a foundation for datatypes, in a
dependently-typed system, for a dependently-typed system.

Closer to us is the work of \citet{benke:universe-generic-prog}. This
seminal work introduced the usage of universes for developing generic
programs. Our universes share similarities to theirs: our universe of
descriptions is similar to their universe of iterated induction, and
our universe of indexed descriptions is equivalent to their universe
of finitary indexed induction. This is not surprising, as we share the
same source of inspiration, namely induction-recursion.

However, we feel ready to offer a more radical prospectus. Their
approach is generative: each universe extends the base type theory
with both type formers and elimination rules. Thanks to levitation, we
rely only on a generic induction and a specialised $\SYMBswitchD$,
closing the type theory. We explore \emph{programming} with codes, but
also how to conceal the encoding when writing `ordinary' programs.


\subsection{Metatheoretical Status}

The \(\Set:\Set\) approach we have taken in this paper is convenient
from an experimental perspective, and it has allowed us to focus
primarily on the encoding of universes, leaving the question
of stratification (and with it, consistency, totality,
and decidability of type checking) to one side. However, we must
surely face up to the latter, especially since we have taken up the
habit of constructing `the set of all sets'. A proper account requires
a concrete proposal for a system of stratified universes which allows
us to make `level-polymorphic' constructions, and we are actively
pursuing such a proposal. We hope soon to have something to prove.

In the meantime, we can gain some confidence by systematically
embedding predicative fragments of our theory within
systems which already offer a universe hierarchy. We can, at the
very least, confirm that in UTT-style theories with conventional
inductive families of types~\citep{luo:utt}, as found in Coq
(and in Agda if one avoids experimental extensions), we build the
tower of universes we propose, cut off at an arbitrary height.
It is correspondingly clear that some such system can be made to
work, or else that other, longer-standing tools are troubled.

A metatheoretical issue open at time of writing concerns the
size of the \emph{index} set \(\V{I}\) in \(\IDesc{\V{I}}\). Both Agda
and recent versions of Coq allow inductive families with \emph{large}
indices, effectively allowing `higher-kind' fixpoints on
\(\Set^{\Set}\) and more. They retain the safeguard that the types of
\emph{substructures} must be as small as the inductively
defined superstructure. This liberalisation allows us large index sets
in our models, but whilst it offers no obvious route to paradox by
smuggling a large universe inside a small type, it is not yet known
to be safe. We can restrict \(\V{I}\) as necessary to avoid
paradox, provided \(\Unit\), used to index \(\SYMBIDesc\) itself, is
`small'.

\subsection{Further Work}

Apart from the need to nail down a stratified version of the system and
its metatheory, we face plenty of further problems and opportunities.
Although we have certainly covered Luo's criteria for inductive
families~\cite{luo:utt}, there are several dimensions in which to
consider expanding our universe.

Firstly, we seek to encompass
\emph{inductive-recursive} datatype families~\cite{dybjer:iir}, allowing
us to interleave the definition and interpretation of data in intricate
and powerful ways. This interleaving seems particularly useful when
reflecting the syntax of dependent type systems.

Secondly, we should very much like to extend our
universe with a codes for internal fixpoints, as
in~\cite{DBLP:conf/types/MorrisAM04}. The external knot-tying
approach we have taken here makes types like `trees with lists of subtrees'
more trouble than they should be. Moreover, if we allow the alternation
of least and greatest fixpoints, we should expect to gain types which are
not readily encoded with one external \(\SYMBIMu\).

Thirdly, it would be fascinating to extend our universe with dedicated
support for syntax with \emph{binding}, not least because a universe
with internal fixpoints has such a syntax. Harper and Licata have
demonstrated the potential for and of such an
encoding~\citep{DBLP:conf/icfp/LicataH09}, boldly encoding the invalid
definitions along with the valid. A more conservative strategy might
be to offer improved support for datatypes indexed by an extensible
context of free variables, with the associated free monad structure
avoiding capture as shown by~\citet{altenkirch:monadic-lambda}.

Lastly, we must ask how our new presentation of datatypes
should affect the tools we use to build software. It is not enough to
change the game: we must enable better play. If datatypes are data,
what is design?


%%%%%%%%%%%%%%%%%%%%%%%%%%%%%%%%%%%%%%%%%%%%%%%%%%%%%%%%%%%%%%%%
%% Acknowledgments
%%%%%%%%%%%%%%%%%%%%%%%%%%%%%%%%%%%%%%%%%%%%%%%%%%%%%%%%%%%%%%%%

\acks

We are grateful to Jos\'{e} Pedro Magalh\~{a}es for his helpful
comments on a draft of this paper. We are also grateful to the Agda
team, without which levitation would have been a much more perilous
exercise. J. Chapman was supported by the Estonian Centre of
Excellence in Computer Science, EXCS, financed by the European
Regional Development Fund. P.-\'{E}. Dagand, C. McBride and P. Morris
are supported by the Engineering and Physical Sciences Research
Council, Grants EP/G034699/1 and EP/G034109/1.


%%%%%%%%%%%%%%%%%%%%%%%%%%%%%%%%%%%%%%%%%%%%%%%%%%%%%%%%%%%%%%%%
%% Bibliography
%%%%%%%%%%%%%%%%%%%%%%%%%%%%%%%%%%%%%%%%%%%%%%%%%%%%%%%%%%%%%%%%


\begin{thebibliography}{47}
\providecommand{\natexlab}[1]{#1}
\providecommand{\url}[1]{\texttt{#1}}
\expandafter\ifx\csname urlstyle\endcsname\relax
  \providecommand{\doi}[1]{doi: #1}\else
  \providecommand{\doi}{doi: \begingroup \urlstyle{rm}\Url}\fi

\bibitem[Abel et~al.()Abel, Coquand, and Pagano]{DBLP:conf/tlca/AbelCP09}
A.~Abel, T.~Coquand, and M.~Pagano.
\newblock A modular type-checking algorithm for type theory with singleton
  types and proof irrelevance.
\newblock In \emph{TLCA}.

\bibitem[Adams(2006)]{DBLP:journals/jfp/Adams06}
R.~Adams.
\newblock Pure type systems with judgemental equality.
\newblock \emph{JFP}, 2006.

\bibitem[Altenkirch and McBride(2002)]{DBLP:conf/ifip2-1/AltenkirchM02}
T.~Altenkirch and C.~McBride.
\newblock Generic programming within dependently typed programming.
\newblock In \emph{Generic Programming}, 2002.

\bibitem[Altenkirch and Reus(1999)]{altenkirch:monadic-lambda}
T.~Altenkirch and B.~Reus.
\newblock Monadic presentations of lambda terms using generalized inductive
  types.
\newblock In \emph{Computer Science Logic}. 1999.

\bibitem[Altenkirch et~al.(2007)Altenkirch, McBride, and
  Swierstra]{altenkirch:ott}
T.~Altenkirch, C.~McBride, and W.~Swierstra.
\newblock Observational equality, now!
\newblock In \emph{PLPV}, 2007.

\bibitem[Augustsson and
  Carlsson(1999)]{augustsson.carlsson:dependent.interpreter}
L.~Augustsson and M.~Carlsson.
\newblock An exercise in dependent types: A well-typed interpreter.
\newblock {Available at
  \url{http://www.cs.chalmers.se/~augustss/cayenne/interp.ps}}, 1999.

\bibitem[Benke et~al.(2003)Benke, Dybjer, and
  Jansson]{benke:universe-generic-prog}
M.~Benke, P.~Dybjer, and P.~Jansson.
\newblock Universes for generic programs and proofs in dependent type theory.
\newblock \emph{Nordic Journal of Computing}, 2003.

\bibitem[Brady et~al.()Brady, Chapman, Dagand, Gundry, McBride, Morris, and
  Norell]{pigs:epigram}
E.~Brady, J.~Chapman, P.-E. Dagand, A.~Gundry, C.~McBride, P.~Morris, and
  U.~Norell.
\newblock An {E}pigram implementation.

\bibitem[Brady et~al.(2003)Brady, McBride, and
  McKinna]{brady:index-inductive-families}
E.~Brady, C.~McBride, and J.~McKinna.
\newblock Inductive families need not store their indices.
\newblock In \emph{TYPES}, 2003.

\bibitem[Cheney and Hinze(2003)]{cheney:gadt}
J.~Cheney and R.~Hinze.
\newblock First-class phantom types.
\newblock Technical report, Cornell University, 2003.

\bibitem[Coquand(1996)]{DBLP:journals/scp/Coquand96}
T.~Coquand.
\newblock An algorithm for type-checking dependent types.
\newblock \emph{SCP}, 1996.

\bibitem[Courant(2002)]{courant:explicit-universe}
J.~Courant.
\newblock Explicit universes for the calculus of constructions.
\newblock In \emph{TPHOLs}, 2002.

\bibitem[Danielsson(2010)]{nisse:asl}
N.~A. Danielsson.
\newblock The {A}gda standard library, 2010.

\bibitem[Dybjer(1991)]{dybjer:families}
P.~Dybjer.
\newblock Inductive sets and families in {M}artin-{L}{\"o}f's type theory.
\newblock In \emph{Logical {F}rameworks}. 1991.

\bibitem[Dybjer and Setzer(1999)]{dybjer:axiom-ir}
P.~Dybjer and A.~Setzer.
\newblock A finite axiomatization of inductive-recursive definitions.
\newblock In \emph{TLCA}, 1999.

\bibitem[Dybjer and Setzer(2000)]{dybjer:ir-initial-algebra}
P.~Dybjer and A.~Setzer.
\newblock Induction-recursion and initial algebras.
\newblock In \emph{Annals of Pure and Applied Logic}, 2000.

\bibitem[Dybjer and Setzer(2001)]{dybjer:iir}
P.~Dybjer and A.~Setzer.
\newblock Indexed induction-recursion.
\newblock In \emph{Proof Theory in Computer Science}. 2001.

\bibitem[Garcia et~al.(2003)Garcia, Jarvi, Lumsdaine, Siek, and
  Willcock]{garcia:generic-comparative-study}
R.~Garcia, J.~Jarvi, A.~Lumsdaine, J.~Siek, and J.~Willcock.
\newblock A comparative study of language support for generic programming.
\newblock In \emph{OOPSLA}, 2003.

\bibitem[Geuvers(2001)]{geuvers:induction-not-derivable}
H.~Geuvers.
\newblock Induction is not derivable in second order dependent type theory.
\newblock In \emph{TLCA}, 2001.

\bibitem[Goguen et~al.(2006)Goguen, McBride, and
  McKinna]{goguen:pattern-matching}
H.~Goguen, C.~McBride, and J.~McKinna.
\newblock Eliminating dependent pattern matching.
\newblock In \emph{Algebra, Meaning and Computation}. 2006.

\bibitem[Harper and Pollack()]{harper:implicit-universe}
R.~Harper and R.~Pollack.
\newblock Type checking with universes.
\newblock In \emph{TAPSOFT'89}.

\bibitem[Hinze(2000)]{hinze:polytypic-polykinded}
R.~Hinze.
\newblock Polytypic values possess polykinded types.
\newblock In \emph{MPC}. 2000.

\bibitem[Hinze et~al.(2002)Hinze, Jeuring, and L\"{o}h]{hinze:generic-haskell}
R.~Hinze, J.~Jeuring, and A.~L\"{o}h.
\newblock Type-indexed data types.
\newblock In \emph{MPC}, 2002.

\bibitem[Hinze et~al.(2007)Hinze, Jeuring, and
  L\"{o}h]{hinze:generic-approach-comparative}
R.~Hinze, J.~Jeuring, and A.~L\"{o}h.
\newblock Comparing approaches to generic programming in {H}askell.
\newblock In \emph{Datatype-Generic Programming}. 2007.

\bibitem[Holdermans et~al.(2006)Holdermans, Jeuring, L\"{o}h, and
  Rodriguez]{holdermans:generic-view}
S.~Holdermans, J.~Jeuring, A.~L\"{o}h, and A.~Rodriguez.
\newblock Generic views on data types.
\newblock In \emph{MPC}. 2006.

\bibitem[Jansson and Jeuring(1997)]{jansson:polyp}
P.~Jansson and J.~Jeuring.
\newblock Poly{P}---a polytypic programming language extension.
\newblock In \emph{POPL}, 1997.

\bibitem[L{\"a}mmel and {Peyton Jones}(2003)]{spj:syb}
R.~L{\"a}mmel and S.~{Peyton Jones}.
\newblock Scrap your boilerplate: a practical design pattern for generic
  programming.
\newblock In \emph{TLDI}, 2003.

\bibitem[Licata and Harper(2009)]{DBLP:conf/icfp/LicataH09}
D.~R. Licata and R.~Harper.
\newblock A universe of binding and computation.
\newblock In \emph{ICFP}, 2009.

\bibitem[Luo(1994)]{luo:utt}
Z.~Luo.
\newblock \emph{Computation and Reasoning}.
\newblock Oxford University Press, 1994.

\bibitem[Martin-L{{\"o}f}(1984)]{martin-lof:itt}
P.~Martin-L{{\"o}f}.
\newblock \emph{Intuitionistic Type Theory}.
\newblock Bibliopolis$\cdot$Napoli, 1984.

\bibitem[McBride and McKinna(2004)]{mcbride.mckinna:view-from-the-left}
C.~McBride and J.~McKinna.
\newblock The view from the left.
\newblock \emph{JFP}, 2004.

\bibitem[Morris(2007)]{morris:PhD}
P.~Morris.
\newblock \emph{Constructing Universes for Generic Programming}.
\newblock PhD thesis, University of Nottingham, 2007.

\bibitem[Morris and Altenkirch(2009)]{alti:lics09}
P.~Morris and T.~Altenkirch.
\newblock Indexed containers.
\newblock In \emph{LICS}, 2009.

\bibitem[Morris et~al.(2004)Morris, Altenkirch, and
  McBride]{DBLP:conf/types/MorrisAM04}
P.~Morris, T.~Altenkirch, and C.~McBride.
\newblock Exploring the regular tree types.
\newblock In \emph{TYPES}, 2004.

\bibitem[Morris et~al.(2009)Morris, Altenkirch, and Ghani]{morris:spf}
P.~Morris, T.~Altenkirch, and N.~Ghani.
\newblock A universe of strictly positive families.
\newblock \emph{IJCS}, 2009.

\bibitem[Norell(2002)]{norell:msc-thesis}
U.~Norell.
\newblock Functional generic programming and type theory.
\newblock Master's thesis, Chalmers University of Technology, 2002.

\bibitem[Norell(2007)]{norell:agda}
U.~Norell.
\newblock \emph{Towards a practical programming language based on dependent
  type theory}.
\newblock PhD thesis, Chalmers University of Technology, 2007.

\bibitem[Oury and Swierstra(2008)]{oury:power-of-pi}
N.~Oury and W.~Swierstra.
\newblock The power of {P}i.
\newblock In \emph{ICFP}, 2008.

\bibitem[Paulin-Mohring(1996)]{paulin:habilitation}
C.~Paulin-Mohring.
\newblock \emph{D{\'e}finitions inductives en th{\'e}orie des types d'ordre
  sup{\'e}rieur}.
\newblock th{\`e}se d'habilitation, ENS Lyon, 1996.

\bibitem[Pierce and Turner(1998)]{pierce:bidirectional-tc}
B.~C. Pierce and D.~N. Turner.
\newblock Local type inference.
\newblock In \emph{POPL}, 1998.

\bibitem[Rodriguez et~al.(2008)Rodriguez, Jeuring, Jansson, Gerdes, Kiselyov,
  and d.~S.~Oliveira]{rodriguez:generic-libs-comparative}
A.~Rodriguez, J.~Jeuring, P.~Jansson, A.~Gerdes, O.~Kiselyov, and B.~C.
  d.~S.~Oliveira.
\newblock Comparing libraries for generic programming in {H}askell.
\newblock In \emph{Haskell Symposium}, 2008.

\bibitem[{The Coq Development Team}()]{coq}
{The Coq Development Team}.
\newblock \emph{The Coq Proof Assistant Reference Manual}.

\bibitem[Verbruggen et~al.(2008)Verbruggen, de~Vries, and
  Hughes]{verbruggen:polytype-prog-coq}
W.~Verbruggen, E.~de~Vries, and A.~Hughes.
\newblock Polytypic programming in {C}oq.
\newblock In \emph{WGP}, 2008.

\bibitem[Verbruggen et~al.(2009)Verbruggen, de~Vries, and
  Hughes]{verbruggen:polytype-coq}
W.~Verbruggen, E.~de~Vries, and A.~Hughes.
\newblock Polytypic properties and proofs in {C}oq.
\newblock In \emph{WGP}, 2009.

\bibitem[Weirich and Casinghino(2010)]{weirich:arity-generic}
S.~Weirich and C.~Casinghino.
\newblock Arity-generic datatype-generic programming.
\newblock In \emph{PLPV}, 2010.

\bibitem[Xi et~al.(2003)Xi, Chen, and Chen]{xi:gadt}
H.~Xi, C.~Chen, and G.~Chen.
\newblock Guarded recursive datatype constructors.
\newblock In \emph{POPL}, 2003.

\bibitem[Yakushev et~al.(2009)Yakushev, Holdermans, L\"{o}h, and
  Jeuring]{yakushev:mutual-def}
A.~R. Yakushev, S.~Holdermans, A.~L\"{o}h, and J.~Jeuring.
\newblock Generic programming with fixed points for mutually recursive
  datatypes.
\newblock In \emph{ICFP}, 2009.

\end{thebibliography}

\bibliographystyle{abbrvnat}

% The bibliography should be embedded for final submission.
%\begin{thebibliography}{}
%\softraggedright
%\end{thebibliography}

\end{document}
