\section{The Type Theory}

\subsection{Base theory}

\begin{structure}
<- Presentation of the formalism
    <- Standard presentation
        -> No novelty here
    <- 3 judgments [equation]
        -> Context validity
        -> Typing judgements
        -> Equality judgements
\end{structure}

\[
\begin{array}{ll}
\Gamma\vdash\Valid & \mbox{\(\Gamma\) is a valid context, giving types to variables} \\
\Gamma\vdash \Bhab{t}{T} & \mbox{term \(t\) has type \(T\) in context \(\Gamma\)} \\
\Gamma\vdash \Bhab{s \equiv t}{T} & \mbox{\(s\) and \(t\) are equal at type \(T\) in context \(\Gamma\)} \\
\end{array}
\]

\begin{structure}
    <- Invariants [equation]
        -> By induction on derivations
\end{structure}

The systems of inference rules will be formulated to ensure that that the
following implications always hold by induction on derivations.
\[
\begin{array}{l@{\;\;\Rightarrow\;\;}l}
\Gamma\vdash \Bhab{t}{T}            & \Gamma\vdash\Valid \;\wedge\; \Gamma\vdash\Type{T} \\
\Gamma\vdash s \equiv \Bhab{t}{T}   & \Gamma\vdash \Bhab{s}{T} \;\wedge\; \Gamma\vdash \Bhab{t}{T} \\
\Gamma;\xS;\Delta \vdash J          & \Gamma\vdash \Bhab{s}{S} \;\Rightarrow\; 
                                           \Gamma;\Delta[s/x] \vdash J[s/x] \\
\end{array}
\]


\begin{structure}
    <- Judgemental equality
        <- Presentation independant of particular implementation choice
        -> Model in Agda, intensional
        -> Used in Epigram, OTT
\end{structure}

\begin{structure}
<- Context validity [figure]
    <- Not much to be said
\end{structure}

\begin{figure}

\[
%% Empty context validity
\Axiom{\vdash \Valid}
\qquad
%% Extend context
\Rule{\Gamma       \vdash \Type{S}}
     {\Gamma ; \xS \vdash \Valid}\;x\not\in\Gamma
\]

\caption{Context validity}
\label{fig:context-validity}
\end{figure}


\begin{structure}
<- Typing judgements [figure]
    <- Set in Set
        -> For simplicity of presentation
        -> Assume that a valid stratification can be inferred
            <- Harper-Pollack, Luo, Courant
        -> See later discussion
    <- Standard presentation of Pi and Sigma types
\end{structure}

\begin{figure}

\[\stkc{
%% Context
\Rule{\Gamma ; \xS ; \Delta \vdash \Valid}
     {\Gamma ; \xS ; \Delta \vdash \Bhab{\V{x}}{\V{S}}}
\qquad
%% Conversion
\Rule{\Gamma \vdash \Bhab{\V{s}}{\V{S}} \quad 
      \Gamma \vdash \Type{\V{S} \equiv \V{T}}}
     {\Gamma \vdash \Bhab{\V{s}}{\V{T}}}
\\
%% Girard's favorite
\Rule{\Gamma \vdash \Valid}
     {\Gamma \vdash \Type{\Set}}
\qquad
%% Pi-Sigma
\Rule{\Gamma       \vdash \Type{\V{S}} \quad
      \Gamma ; \xS \vdash \Type{\V{T}}}
     {\Gamma \vdash \Type{\PIS{\xS} \V{T}, \SIGMAS{\xS} \V{T}}}
\\
%% %% Prop
%% \Rule{\Gamma \vdash \Bhab{q}{\Prop}}
%%      {\Gamma \vdash \Type{\prf{q}}}
%% \qquad
%% %% True
%% \Rule{\Gamma \vdash \Valid}
%%      {\Gamma \vdash \Bhab{\True}{\Prop}}
%% \\
%% Unit
\Rule{\Gamma \vdash \Valid}
     {\Gamma \vdash \Bhab{\Unit}{\Set}}
\qquad
%% Void
\Rule{\Gamma \vdash \Valid}
     {\Gamma \vdash \Bhab{\Void}{\Unit}}
\\
%% Lambda
\Rule{\stkl{\Gamma       \vdash \Type{\V{S}} \\
            \Gamma ; \xS \vdash \Bhab{\V{t}}{\V{T}}}}
     {\Gamma \vdash \Bhab{\PLAM{\x}{\V{S}} \V{t}}{\PIS{\xS} \V{T}}}
\qquad
%% Application
\Rule{\stkl{\Gamma \vdash \Bhab{\V{f}}{\PIS{\xS} \V{T}} \\
            \Gamma \vdash \Bhab{\V{s}}{\V{S}}}}
     {\Gamma \vdash \Bhab{\V{f}\: \V{s}}{\V{T}[\V{s}/\V{x}]}} 
\\
%% Pair
\Rule{\Gamma       \vdash \Bhab{\V{s}}{S} \quad 
      \Gamma ; \xS \vdash \Bhab{\V{T}}{\Set}    \quad
      \Gamma       \vdash \Bhab{\V{t}}{\V{T}[\V{s}/\V{x}]}}
     {\Gamma \vdash \Bhab{\pair{\V{s}}{\V{t}}{\V{x}.\V{T}}}{\SIGMAS{\xS} \V{T}}}
\\
%% First projection
\Rule{\Gamma \vdash \Bhab{\V{p}}{\SIGMAS{\xS} \V{T}}}
     {\Gamma \vdash \Bhab{\fst{\V{p}}}{\V{S}}} 
\qquad
%% Second projection
\Rule{\Gamma \vdash \Bhab{\V{p}}{\SIGMAS{\xS} \V{T}}}
     {\Gamma \vdash \Bhab{\snd{\V{p}}}{\V{T}[\fst{\V{p}}/\V{x}]}}
\\
}\]


\caption{Typing judgements}
\label{fig:typing-judgements}

\end{figure}


\begin{structure}
<- Judgemental equality [figure]
    <- symmetry, reflexivity, and transitivity
    <- beta-rules for lambda and pair
    <- xi-rule for functions
    -> Agnostic in the notion of equality
        <- Doesn't rely on a ``propositional'' equality
\end{structure}


\begin{figure}

\[\stkc{
%% %% Reflexivity
%% \Rule{\Gamma \vdash \Bhab{x}{T}}
%%      {\Gamma \vdash \Bhab{x \equiv x}{T}}
%% \qquad
%% %% Symmetry
%% \Rule{\Gamma \vdash \Bhab{x \equiv y}{T}}
%%      {\Gamma \vdash \Bhab{y \equiv x}{T}}
%% \qquad
%% %% Transitivity
%% \Rule{\stkl{\Gamma \vdash \Bhab{z \equiv y}{T} \\
%%             \Gamma \vdash \Bhab{y \equiv z}{T} }}
%%      {\Gamma \vdash \Bhab{x \equiv z}{T}}
%% \\
%% Beta-reduction
\Rule{\stkl{\Gamma       \vdash \Type{\M{S}} \quad
            \Gamma ; \xS \vdash \Bhab{\M{t}}{\M{T}} \\
            \Gamma       \vdash \Bhab{\M{s}}{\M{S}}}}
     {\Gamma \vdash \Bhab{(\PLAM{\x}{\M{S}} \M{t})\:\M{s} \equiv \M{t}[\M{s}/\x]}{\M{T}[\M{s}/\x]}}
\\
%% Xi rule
%% \Rule{\Gamma       \vdash \Type{S} \quad
%%       \Gamma ; \xS \vdash \Bhab{t \equiv t'}{T}}
%%      {\Gamma \vdash \Bhab{(\PLAM{\x}{S} t) \equiv (\PLAM{\x}{S} t')}{\PIS{\xS} T}}
%% \\
%% Projections
\Rule{\stkl{\Gamma                 \vdash \Bhab{\M{s}}{\M{S}} \quad
            \Gamma ; \xS           \vdash \Bhab{\M{T}}{\Set} \\
            \Gamma ; \Bhab{\M{s}}{\M{S}}   \vdash \Bhab{\M{t}}{\M{T}[\M{s}/\x]}}}
     {\Gamma \vdash \Bhab{\fst{(\pair{\M{s}}{\M{t}}{\x.\M{T}})} \equiv \M{s}}{\M{S}}}
\qquad
\Rule{\stkl{\Gamma               \vdash \Bhab{\M{s}}{\M{S}} \quad
            \Gamma ; \xS         \vdash \Bhab{\M{T}}{\Set} \\
            \Gamma ; \Bhab{\M{s}}{\M{S}} \vdash \Bhab{\M{t}}{\M{T}[\M{s}/\x]}}}
     {\Gamma \vdash \Bhab{\snd{(\pair{\M{s}}{\M{t}}{\x.\M{T}})} \equiv \M{t}}{\M{T}[\M{s}/\x]}}
}\]


\caption{Judgemental equality}
\label{fig:judgemental-equality}

\end{figure}



\begin{structure}
!!! Need Help !!!
<- Meta-theoretical properties
    <- Assuming a stratified discipline
    <> The point here is to reassert that dependent types are not evil, 
       there is no non-terminating type-checker, or such horrible lies <>
    -> Strongly normalizing
        -> Every program terminates
    -> Type-checking terminates
    ???
\end{structure}

\subsection{Finite sets}

\begin{structure}
<- Motivation
    <- Finite sets could be encoded with Unit and Bool
        /> Hinder the ability to name things
    -> For convenience
        <- Named elements
        <- Referring by name instead of code
        -> Types as coding presentation
            /> Also as coding representation!
\end{structure}

\begin{structure}
<- Implementation [figure]      
    <- Tags
        -> Purely informational token
    <- EnumU
        -> Universe of finite sets
    <- EnumT e
        -> Elements of finite set e
\end{structure}

\begin{figure}

\[\stkc{
%% UId
\Rule{\Gamma \vdash \Valid}
     {\Gamma \vdash \Type{\UId}}
\qquad
%% Tag
\Rule{\Gamma \vdash \Valid}
     {\Gamma \vdash \Bhab{\Tag{s}}{\UId}}\;s \mbox{ unique identifier}
\\
%% EnumU
\Rule{\Gamma \vdash \Valid}
     {\Gamma \vdash \Type{\EnumU}} 
\qquad
%% EnumT
\Rule{\Gamma \vdash \Bhab{e}{\EnumU}}
     {\Gamma \vdash \Type{\EnumT{e}}} 
\\
%% NilE
\Rule{\Gamma \vdash \Valid}
     {\Gamma \vdash \Bhab{\NilE}{\EnumU}} 
\qquad
%% ConsE
\Rule{\Gamma \vdash \Bhab{t}{\UId} \quad
      \Gamma \vdash \Bhab{e}{\EnumU}}
     {\Gamma \vdash \Bhab{\ConsE{t}{e}}{\EnumU}}
\\
%% Ze
\Rule{\Gamma \vdash \Valid}
     {\Gamma \vdash \Bhab{\Ze}{\EnumT{\ConsE{t}{e}}}} 
\qquad
%% Su
\Rule{\Gamma \vdash \Bhab{n}{\EnumT{e}}}
     {\Gamma \vdash \Bhab{\Su{n}}{\EnumT{\ConsE{t}{e}}}}
}\]


\caption{Typing rules for finite sets}
\label{fig:typing-finite-set}

\end{figure}


\begin{structure}
<- Equipment
    <- \spi operator
        <- Equivalent of Pi on finite sets
        <- First argument: (finite) domain
        <- Second argument: for each element of the domain, a co-domain
        -> Inhabitant of \spi: right-nested tuple of solutions
            <- Skip code for space reasons
    <- switch operator
        <- case analyses over x
        <- index into the \spi tuple to retrieve the corresponding result
\end{structure}

\begin{figure}

\[\stkl{
%% spi
\F{$\pi$} : \PITEL{\V{e}}{\EnumU}
            \PITEL{\V{P}}{\EnumT{e} \To \Set} \To \Set 
\\
%% switch
\begin{array}{@{}ll}
\F{switch} : & \PITEL{\V{e}}{\EnumU}
               \PITEL{\V{P}}{\EnumT{e} \To \Set} \\
             & \PITEL{\V{b}}{\spi{e}{P}}
               \PITEL{\V{x}}{\EnumT{e}} \To P~x
\end{array}
}\]

\caption{Operators on finite sets}
\label{fig:operator-finite-set}

\end{figure}


\begin{structure}
<- Equivalent to having a function space over finite sets
    /> Made non-obvious by low-level encodings
        <- General issue with codes
             -> Need to provide an attractive presentation to the user
    -> Types seem to obfuscate our reading
        <- Provide ``too much'' information
        /> False impression: information is actually waiting to be used more widely
        -> See next Section
\end{structure}

\subsection{Type propagation}

\begin{structure}
<- Bidirectional type-checking [ref. Turner,Pierce]
    -> Separating type-checking from type synthesis
    <- Type checking: push terms into types
        <- Example: |Pi S T :>: \ x . t| allows us to drop annotation on lambda
    <- Type inference: pull types out of terms
        <- Example: |x : S l- x :<: S| gives us the type of x
\end{structure}

\begin{structure}
<- Formalization: type propagation
    <- Motivation
        -> High-level syntax
            -> exprIn: types are pushed in
                <- Subject to type *checking*
            -> exprEx: types are pulled from
                <- Subject to type *synthesis*
        -> Translated into our low-level type theory
        -> Presented as judgements
    -> Presentation mirors typing rule of [figure] 
        -> Ignore identitical judgements
\end{structure}

\begin{structure}
<- Type checking [figure]
    <- Push a type in an exprIn
    <- Result in a full term
    -> *Use* the type to build the term!
        -> Domain and co-domain propagation for Pi and Sigma
        -> Translation of 'tags into EnumTs
        -> Translation of ['tags ...] into EnumUs
        -> Finite function space into switch
\end{structure}

\begin{figure}

\[\stkc{
%% Form
\boxed{\Gamma \Vdash \propag{\push{\CN{exprIn}}{\CN{type}}}{\CN{term}}} 
\\
\\
%% Set in Set
%% \Axiom{\Gamma \Vdash \propag{\push{\Set}{\Set}}
%%                             {\Set}}
%% \\
%% Prop
%% \Rule{\Gamma \Vdash \propag{\push{q}{\Prop}}
%%                            {q'}}
%%      {\Gamma \Vdash \propag{\push{\prf{q}}{\Set}}
%%                            {\prf{q'}}}
%% \qquad
%% True
%% \Axiom{\Gamma \Vdash \propag{\push{\True}{\Prop}}
%%                             {\True}}
%% \\
%% Pi
%% \Rule{\Gamma \Vdash \propag{\push{S}{\Set}}
%%                            {S'} \quad
%%       \Gamma \Vdash \propag{\push{T}{S' \To \Set}}
%%                            {T'}}
%%      {\Gamma \Vdash \propag{\push{\PiTy{S}{T}}{\Set}}
%%                            {\PiTy{S'}{T'}}} 
%% \\
%% %% Sigma
%% \Rule{\Gamma \Vdash \propag{\push{S}{\Set}}
%%                            {S'} \quad
%%       \Gamma \Vdash \propag{\push{T}{S' \To \Set}}
%%                            {T'}}
%%      {\Gamma \Vdash \propag{\push{\SigmaTy{S}{T}}{\Set}}
%%                            {\SigmaTy{S'}{T'}}}
%% \\
%% Lambda
\Rule{\Gamma ; \xS \Vdash \propag{\push{\V{t}}{\V{T}}}
                                 {\V{t'}}}
     {\Gamma \Vdash \propag{\push{\LAM{\x} \V{t}}{\PIS{\xS}{\V{T}}}}
                           {\PLAM{\x}{\V{S}} \V{t'}}} 
\\
%% Pair
\Rule{\stkl{ \Gamma \Vdash \propag{\push{\V{s}}{\V{S}}}
                                  {\V{s'}} \\
             \Gamma \Vdash \propag{\push{\V{t}}{\V{T}[\V{s'}/\V{x}]}}
                                  {\V{t'}}}}
     {\Gamma \Vdash \propag{\push{\pair{\V{s}}{\V{t}}{}}{\SIGMAS{\xS}{\V{T}}}}
                           {\pair{\V{s'}}{\V{t'}}{\V{T}}}}
\\
%% EnumU
%% \Axiom{\Gamma \Vdash \propag{\push{\EnumU}{\Set}}
%%                             {\EnumU}} 
%% \qquad
%% %% EnumT
%% \Rule{\Gamma \Vdash \propag{\push{e}{\EnumU}}
%%                            {e'}}
%%      {\Gamma \Vdash \propag{\push{\EnumT{e}}{\Set}}
%%                            {\EnumT{e'}}}
%% \\
%% Tag
\Axiom{\Gamma \Vdash \propag{\push{\etag{t_0}}{\EnumT{(\ConsE{\etag{t_1}}{\V{e}})}}}
                            {\Ze}}\;\etag{t_0} = \etag{t_1}
\\
\Rule{\Gamma \Vdash \propag{\push{\etag{t_0}}{\EnumT{\V{e}}}}
                           {\V{n}}}
     {\Gamma \Vdash \propag{\push{\etag{t_0}}{\EnumT{(\ConsE{\etag{t_1}}{\V{e}})}}}
                            {\Su{\V{n}}}}\;\etag{t_0} \neq \etag{t_1}
\\
%% EnumU
\Axiom{\Gamma \Vdash \propag{\push{[]}{\EnumU}}
                            {\NilE}}
\\
\Rule{\Gamma \Vdash \propag{\push{\V{ts}}{\EnumU}}
                                 {\V{cs}}}
     {\Gamma \Vdash \propag{\push{[ \etag{t_1},\: \V{ts} ]}{\EnumU}}
                                 {\ConsE{\etag{t_1}}{\V{cs}}}}
\\
%% Switch
\Rule{\Gamma \Vdash \propag{\push{\V{t}}{\spi{\V{e}}{\V{P}}}}
                           {\V{t'}}}
     {\Gamma \Vdash \begin{array}{@{}l} 
                        \propag{\push{\V{t}}{\PI{\V{x}}{\EnumT{\V{e}}} \V{P}\:\V{x}}}
                               {\\ \PLAM{\V{x}}{(\EnumT{\V{e}})} \switch{\V{e}}{\V{P}}{\V{t'}}{\V{x}}}
                    \end{array}}\;\mbox{$\V{t}$ is $[]$ or $[\_,\_]$}
%% \\
%% Conversion
%% \Rule{\Gamma \Vdash \propag{s}
%%                            {\pull{s'}{S}} \quad 
%%       \Gamma \Vdash \push{S \equiv T}{\Set}}
%%      {\Gamma \Vdash \propag{\push{s}{T}}
%%                            {s'}}
}\]


\caption{Type checking}
\label{fig:type-checking}

\end{figure}


\begin{structure}
<- Type synthesis [figure]
    <- Pull a type out of an exprEx
    <- Result in a full term, together with its type
    -> Do *not* need to specify types
        -> Extracting a term from the context
        -> Function application
        -> Projections
\end{structure}

\begin{figure}

\[\stkc{
%% Form
\boxed{\Gamma \Vdash \propag{\exprEx}{\pull{\CN{term}}{\CN{type}}}}
\\
\\
%% Reversal
\Rule{\Gamma \Vdash \propag{\push{T}{\Set}}
                           {T'} \quad
      \Gamma \Vdash \propag{\push{t}{T'}}
                           {t'}}
     {\Gamma \Vdash \propag{(\Bhab{t}{T})}
                           {\pull{t'}{T'}}} 
\\
%% Context
\Rule{\Gamma ; \xS ; \Delta \vdash \Valid}
      {\Gamma ; \xS ; \Delta \Vdash \propag{\x}
                                           {\pull{\x}{\M{S}}}}
\qquad
%% Application
\Rule{\stkl{\Gamma \Vdash \propag{\M{f}}
                                 {\pull{\M{f}\M{'}}{\PIS{\xS}{\M{T}}}} \\
            \Gamma \Vdash \propag{\push{\M{s}}{\M{S}}}
                                 {\M{s'}}}}
     {\Gamma \Vdash \propag{\M{f}\: \M{s}}{\pull{\M{f'}\: \M{s'}}{\M{T} [\M{s'}/\x]}}} 
\\
%% First projection
\Rule{\Gamma \Vdash \propag{\M{p}}
                           {\pull{\M{p'}}{\SIGMAS{\xS}{\M{T}}}}}
     {\Gamma \Vdash \propag{\fst{\M{p}}}
                           {\pull{\fst{\M{p'}}}{\M{S}}}} \qquad 
%% Second projection
\Rule{\Gamma \Vdash \propag{\M{p}}
                           {\pull{\M{p'}}{\SIGMAS{\xS}{\M{T}}}}}
     {\Gamma \Vdash \propag{\snd{\M{p}}}
                           {\pull{\snd{\M{p'}}}{\M{T} [\fst{\M{p'}}/\x]}}}
}\]


\caption{Type synthesis}
\label{fig:type-synthesis}

\end{figure}


\begin{structure}
<- Summary
    -> Not a novel technique [refs?]
        /> Used as a boilerplate scrapper
    -> Make dealing with codes *practical*
        <- Example: Finite sets/finite function space
        -> We should not restrain ourself in using codes
            <- We know how to present them to the user
-> Will extend this machinery in further sections
\end{structure}
